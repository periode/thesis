\documentclass{article}

\usepackage[hyphens]{url}
\usepackage{csquotes}
\usepackage{fontspec}
\usepackage{graphicx}
\usepackage{listings}
\usepackage{authblk}
\usepackage[normalem]{ulem}
\usepackage[bottom]{footmisc}
\usepackage{hyperref}

\graphicspath{ {./images/} }

\defaultfontfeatures{Mapping=tex-text,Scale=1.00}
\setmainfont{Inter Light}
\setmonofont{Liberation Mono}
\linespread{1.50}
\sloppy

\lstset{
    basicstyle=\footnotesize\ttfamily,
    breaklines=true,
    frame=single
}

\begin{document}
\title{Critiques protocolaires d'Internet: Comparaison des projets IPFS et SecureScuttleButt}
\author{Pierre Depaz}
\affil{Paris-3 Sorbonne-Nouvelle - THALIM}
\maketitle

\section{Introduction}

Avec toutes ses implications économiques, sociales et politiques, l'Internet et le Web\cite{fielding_hypertext_2014} sont avant tout des protocoles de communication, c'est-à-dire un ensemble de règles permettant à deux parties ou plus de requérir et fournir des données dans un même réseau. Cette suite de protocoles voit le jour autour de TCP/IP à la fin des années 1960 sous l'égide de la recherche militaire étasunienne, tandis que les deux protocoles HTTP et HTTPS qui constituent l'infrastructure du Web sont développés et distribués par le Centre Européen de la Recherche Nucléaire, une institution publique de recherche. Ces deux documents sont mis-à-jour, avec la version 6 de l'Internet Protocol et la version 2 de l'HyperText protocol étant actuellement (2022) en cours d'adoption.

Pourtant, l'utilisation de ces protocoles ont découlé sur des utilisations bien différentes de leurs usages intialement envisagés—i.e. la sûreté des données en cas d'attaque militaire soviétique et l'accès à des articles et de la documentation de recherche en physique. Cette évolution est notamment documentée par Lawrence Lessig, dans son ouvrage \emph{Code and Other Laws of Cyberspace}\cite{lessig_code_1999}, en ce qu'il identifie différentes forces capables de faconner l'évolution de l'Internet et du Web: des forces légales, marchandes, sociales et technologiques\footnote{Des analyses notamment confirmées par Dominique Cardon.}.

Les dérives de surveillance, de limitation de partage et de monopole des applications issues des protocoles Internet et Web sont donc bien documentées. Face à celles-ci s'élèvent alors plusieurs types de critiques: critiques sémantiques, sous la forme de blogs, de livres, d'articles et de conférences; critiques légales, telles que les licences GPL ou Creative Commons ou les législations de la RGPD ou du DGA; ou encore critiques programmatiques, telles que les bloqueurs de publicités. Bien qu'il y ait eu des solutions légales avancées en réponse critiques à ces évolutions des usages des technologies d'Internet, telles que l'ensemble des licenses Creative Commons, dans la lignée de GPL et licenses copyleft, il n'en reste pas moins que les forces technologiques peuvent influencer fortement les possibilités d'agir des utilisateurs de ces dernières. Par example, Harsh Gupta s'interroge sur le manque de représentation des contients africains, sud-américains et asiatiques (respectivement 0\%, 0\% et 0\%) lors des délibération ayant pour objet l'implémentation de l'Encrypted Media Extensions. L'EME est un standard de communication pour contenus protégés par une propriété intellectuelle, une propriété intellectuelle de tradition exclusivement occidentale désormais établie en tant que vérité technique plutôt que réglementation économico-politique. Dans ce cas-là, il semble que le protocole en lui-même comporte une capacité d'influence et de détermination du comportement de l'utilisateur.

Ces différentes critiques sont donc toutes des manières d'exposer limitations et alternatives à un objet donné à un moment donné, se focalisant souvent sur un ou plusieurs points majoritaires. La critique sémantique est argumentative, et offre des stratégies discursives, la critique légale déploie un appareil d'arguments valides en termes législatifs, et la critique programmatique promeut l'utilisation de de dispositifs d'actions (dont les logiciels font partie) pour remédier aux limitations identifiées de manière pratique. Le type de critique sur lequel je vais me pencher ici est celui de la \emph{critique protocolaire}.

Partant du principe, selon Galloway, qu'un protocole encode des manières de faires qui contraignent ses utilisateurs à la suivre sous peine d'être exclus de la communication se déroulant à travers ce protocole\cite{galloway_protocol_2004}, j'envisage ici la critique protocolaire comme la conception et la distribution d'infrastructures abstraites (devant être implémentées \emph{a posteriori}) qui addressent les limitations identifiées d'une infrastructure existante.

De ce question de critique protocolaire découlent plusieurs questions que nous aborderons à travers la comparaison de deux études de cas: celle du protocole IPFS (Interplanetary Filesystem) et celle du protocole SSB (Secure Scuttlebutt). Il s'agira d'examiner, dans les deux cas, les capacités expressives des protocoles numériques en tant que sous-ensemble des systèmes computationnels, en se basant notamment sur les travaux d'Ian Bogost en rhétorique procédurelle\footnote{Ian Bogost}, d'examiner les possibilités de déterminisme technologique en comparant les usages abstraits imaginés par le protocole et ses implémentations concrètes, et donc de considérer à quel point ces protocoles proposent des nouveaux imaginaires possibles pour l'échange d'information sur des réseaux numériques, notamment quant aux façons d'imaginer, techniquement, l'espace et le temps. Comment se constitue une critique protocolaire? Quels sont les environnements, documents et actions sociales, économiques et techniques qui doivent être déployés pour subvenir à la pérennisation d'un protocole?

Afin d'élucider ces questions, nous procéderons à une analyse du discours des deux écosystèmes d'IPFS et SSB. Ces écosystèmes comportent des éléments discursifs décrivant leurs protocoles respectifs tant au niveau normatif (le protocole en lui-même), que prescriptif (les usages imaginés par les concepteurs), descriptif (la représentation du projet à travers sites webs, entretiens dans la presse et promotion individuelle) ou encore argumentatifs et participatifs (discussions entre concepteurs et utilisateurs autour des intentions et usages des protocoles). Le cadre d'interprétation de ces documents est donc celui d'une analyse critique du discours, partant de l'hypothèse que ces différents facettes du discours d'une même organisation permet alors de mettre à jour une certaine cosmogonie suggérée avec, à sa base, un protocole comme élément similaire.

Cette approche d'analyse critique du discours a lieu au sein d'une analyse comparative, et cela pour deux raisons. Premièrement, il s'agit de mettre en exergue les éléments communs au déploiement d'un protocole: pendant technique, pendant communicationel, et pendant social, et de voir comment le contenu et la forme de ces éléments peuvent varier selons les présupposés des concepteurs. Deuxièmement, il s'agit de considérer de considérer l'implication d'un même but (communication d'un message d'un émetteur à un récepteur), avec un même algorithme (SHA-512) et d'observer à quel point ce but et ce moyens résultent, ou non, en des conséquences drastiquement différentes.

Il s'agira donc d'approcher le sujet en deux temps. Tout d'abord, nous examinerons les \emph{weltanschauung} des deux projets, en commencant par IPFS, suivi de SSB. Ces examinations se feront de manière identique, à travers la description du protocole, puis l'identification du mythe fondateur, et du support technique et discursif de ce dernier, pour enfin conclure sur l'applicabilité et les possibles limites de la confrontation au réel que chaque protocole présente. Ensuite, nous approfondirons notre comparaison en considérant: (1) la composante spatio-temporelle telle qu'elle est impliquée dans chaque protocole, (2) les propriétés d'un tel mode de critique.

\pagebreak

\bibliographystyle{unsrt}
\bibliography{resistic.bib}

\end{document}