\documentclass{article}

\usepackage[hyphens]{url}
\usepackage{fontspec}
\usepackage{listings}
\usepackage{authblk}
\usepackage[bottom]{footmisc}

\defaultfontfeatures{Mapping=tex-text,Scale=1.00}
\setmainfont{Inter Light}
\setmonofont{Liberation Mono}
\linespread{1.50}
\sloppy

\lstset{
    basicstyle=\footnotesize\ttfamily,
    breaklines=true,
    frame=single
}

\begin{document}
\title{Critiques de l'espace-temps d'Internet dans les réseaux Scuttlebutt et IPFS}
\author{Pierre Depaz}
\affil{Paris 3 - Sorbonne Nouvelle - THALIM}
\date{Août 2021}
\maketitle

Mots-clés: Infrastructure, IPFS, Scuttlebutt, Cosmotechnique, Alternative

\section{Abstract}

La critique de l'internet a travers l'infrastructure. L'Internet est avant tout un réseau infrastructurel, en cela qu'il est le résultat de réalités technologiques particulières, composées d'objets et de normes (légales ou informatiques) permettant le développement de contenus ou de pratiques. Cette infrastructure elle-même est née de plusieurs rêves politiques. L'apparition de l'Internet, et de sa version plus accessible, le Web, dépendaient de plusieurs visions du monde (Vanevar Bush, Arpanet, Declaration of Cyberspace). Cependant, une des principales réalités du Web aujourd'hui est celle d'un espace commercialisé, basé sur la marchandisation des relations interpersonnelles (Nick Couldry, The Cost of Connection). Cet article explore les propositions d'infrastructures comme alternatives à l'état contemporain de l'Internet à travers le prisme de deux projets particuliers: IPFS (Interplanetary Files System) et Scuttlebutt.

Ces deux projets seront analaysés comparativement selon les approches à l'espace et au temps qu'ils proposent. En effet, tout média tend à agir sur notre conception de l'espace et du temps, depuis le développement de l'écriture comme ancrage du discours jusqu'aux réseaux électroniques fondant un village global. L'action que l'Internet a eu sur notre conception de l'espace (code/space) et du temps (pettman) n'est pas neutre, et ce sont ces conceptions qu'IPFS et Scuttlebutt cherchent à questionner. D'une part, IPFS\cite{benet_ipfs_2014} considère remédier à l'aspect éphémère du Web en garantissant que tout contenu posté sera conservé \emph{ad vitam aeternam}, accessible à tous à tout moment. À l'opposé Scuttlebutt propose une permanence similaire, mais limitant l'accès aux contenus à une proximité géographique des individus, pouvant échanger des suites de messages uniquement si connectés à un réseau local. L'idée est cependant la même: changer les pratiques des utilisateurs pour le meilleur en proposant de nouvelles cosmotechniques (yuk hui).

Ces deux alternatives à l'actuelle approche sptio-temporelle du Web comportent néanmoins des ressorts politiques implicites ou explicites. À travers cette communication, je propose d'élucider les fondations théoriques et politiques de projets techniques, afin de mettre à jour les implications socio-politiques de projets techno-utopiques. Afin d'établir cela, nous proposons d'analyser les discours des développeurs de ces deux systèmes tels qu'ils se présentent à leur communauté au travers de discussions sur forums, de \emph{white papers}, d'articles de blogs ainsi qu'issus de correspondance personelle. Ce corpus sera examiné sur le mode d'une analyse de discours critique.

Plus que des pratiques alternatives, cette communication cherche donc à mettre en avant les différentes explorations de pistes pouvant constituer des alternatives aux conditions actuelles d'existence spatio-temporelles d'internet et \emph{d'identifier les dimensions d'agencement de ces actes de détournements et de substitution} afin d'atteindre certains buts de résilience, de vie privée et de communauté.

\pagebreak

\bibliographystyle{unsrt}
\bibliography{resistic.bib}

\end{document}