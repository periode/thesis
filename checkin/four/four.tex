\documentclass{article}

\usepackage[hyphens]{url}
\usepackage{fontspec}
\usepackage{graphicx}
\usepackage{listings}

\graphicspath{ {./images/} }

\defaultfontfeatures{Mapping=tex-text,Scale=1.00}
\setmainfont{Inter Light}
\setmonofont{Liberation Mono}
\linespread{1.50}
\sloppy

\lstset{
    basicstyle=\footnotesize\ttfamily,
    breaklines=true,
    frame=single
}

\begin{document}
\title{Thesis check-in - Role of aesthetics in understanding source code}
\author{Pierre Depaz}
\date{March 2021}
\maketitle

\section{Introduction}

After establishing ground work on aesthetic manifestations in source code for software developers during the Spring 2020 semester, I have concluded with both an empirical manifestation of beautiful code, synthesized a typology of such manifestations—how code can be beautiful—and laid out a preliminary investigation as to why code could be beautiful. Aesthetic manifesations ("beauty") seem to occur whenever facilitate the clarity of intent of the writer, and the agency of the reader, are heightened. Beautiful code makes the underlying concepts clear and easily-graspable, and facilitates its modification by the reader by providing an error-free, and cognitively easy way to do so.

Furthermore, I outlined several directions for further research. These included the exploration of aesthetic standards for two additional categories of code writers: \emph{source code poets} and \emph{hackers}. Following discussions around this outcome, I have added three other directions: literary metaphors, architectural parallels and machine understanding. First, the place of literary metaphors is a response to the cognitive stake at play in reading and writing code, since code can be understood as a formal representation of mental models emerging from complex data structures and their processing during execution time. Second, the parallels with architecture were suggested by a similar relationship to structure, planning and construction. These parallels are, as we will see below, claimed by software developers themselves, ranging from job titles to commercial best practices of software patterns; on a more theoretical level, the approaches to beauty in architecture will turn out to be productive lenses when thinking not just about executed code, but about source code as well. Third, when claiming that beautiful code facilitates understanding(s), it is important to clarify \emph{whose} understanding of \emph{what}. While previous work has focused on human understandings of human intentions, and human-made concepts, this document investigates to what extent do computers, as concrete machines, understand anything.

I will start by examining instances of source code poetry, defining it, contextualizing it, and analyzing it through close-readings. This will allow us to highlight specific aesthetic standards emerging from this corpus, namely \emph{semantic layering} and \emph{procedural rhetoric}. Source code poetry, with this clear emphasis on \emph{poetry}, will then allow us to address the traditional relationship between literature and code, on an artistic level as well as on a linguistic one. The two concepts mentioned above will lead to an examination of the metaphor, from a literary and from a cognitive standpoint.

Particularly, the relationship that metaphors maintain to the process of knowing and understanding will be highlighted both in human texts and in program texts\cite{detienne_software_2012}. Connecting it to mental models will allow us to start thinking of these program texts in terms of \emph{structure}, both surface-structure and deep-structure, and address how a theoretical framework of aesthetics might be connecting the two, including the place of imagination in acquiring knowledge and building understanding in these texts.

Mentioning structure will thus lead us into the overlap between architecture and software. After a short overview of how the two are usually related, I examine a particular set of aesthetic standards developed by Christopher Alexander in his work on pattern languages. At the cursory level, these are tied to software patterns, techniques for developing better software that have emerged more out of practice rather than out of theory. At a deeper level, we will see that the standards of beauty—or, rather, of this \emph{Quality Without a Name}—can be applied productively to better understand what qualities are exhibited by a program that is deemed beautiful. In particular, Richard P. Gabriel's work will further provide a connection between software, architecture and poetry.

One particular aspect of architecture—the folly, the pavillion, and to some extent large-scale installation artworks—will allow us to transition into our next corpus: hacking. Hacking, defined further as seemingly-exclusively functional code will further requalify the need for aesthetics in source code. We will see how this practice is focused much less on human understanding than on machine understanding, on producing code that is unreadable for the former, and yet crystal-clear for the latter—with an emphasis on human and machine performance. Despite a current lack of extensive research on hacking-related program texts, we will look into two instances of these: the one-liner and the demo to support our investigation in this domain.

This brings us to the broader question of human understanding and machine understandings. Starting from the distinction between syntax and semantics, I highlight discrepancies between semantics in natural languages and semantics in programming languages to define machine understanding as an autotelic one, completely enclosed within a formal description. Coming back to Goodman, we will see how such a formal system fits as a \emph{language of art}, and yet remains ambiguous: is computation exclusively concerned with itself, or can it be said that it relates to the rest of the (non-computable) world? Additionally, the question of aesthetics within programming languages themselves will be approached in a dual approach: as linguistic constructs presenting affordances for creating program texts which exhibit aesthetic properties, and as objects with aesthetic properties themselves. Whether or not we can agree on machine understanding, the formalism of programming languages, and their aesthetic possibilities, provide an additional perspective on the communication of non-obvious concepts inherent to computing.

In conclusion, we will see that aesthetics in code is not exclusively a literary affair in the strict sense of the term, but is rather at the intersection of literature, architecture and problem-solving, insofar as they manifest through the (re-)presentation of complex concepts and multi-faceted uses, involving their writers and readers in semantics-heavy cognitive processes and mental structures.

Finally, I suggest further directions for research.

\section{Programming and literary arts}

This section focuses on source code poetry, as the closest use of "literary arts" involving code. We will see how this particular way of writing software, to an explicitly aesthetic end, rather than a functional one, summons specific claims to art and beauty. These claims maintain a complex relationship to the nature and purpose of code, in certain ways embracing the former, and moving away from the latter, but nonetheless allow us to more clearly define such a nature and such purposes. After an overview of the field, including delimitation of our corpus, I will highlight and analyze particular source code poems, chosen for their meaning-making affordances, and conclude on the aesthetic standards at play in their reading and writing, expanding on notions of \emph{double-meaning} and \emph{double-coding}.

\subsection{Overview}

Source code poetry is a distinct subset of electronic literature. A broad field encompassing natural language texts taking full advantage of the dynamic feature of computing to redefine the concept of text, authorship and readership, it nonetheless encompasses a variety of approaches, including generative literature, interactive fiction, visual poetry, source code poetry and esoteric programming languages, as well as certain aspects of software art. However, one of the distinctions that can be made in defining the elements of electronic literature which are included in our corpus is, in line with the framework of this research, the shift from output to input, for executable binary to latent source.

A large section of the works which fall within electronic literature focus on the result of an executed program, often effectively obfuscating one of the many chained acts of writing\footnote{See: Béatrice Fraenkel on chains of writing} which allow for the very existence of these works. For instance, the influence of \emph{Colossal Cave Adventure}\cite{crowthers_colossal_1977}, the first work of interactive fiction, has been centered around on the playable output of the software, rather than on its source code. Written in FORTRAN 4 between 1975 1977, it exhibits several features which wouldn't fit within the typology we've previously established, particularly in terms of variable naming (e.g. variables such as `KKKT`, `JSPK`; or `GOTO` statements, whose harm has been considered at the same time this code was written\footnote{retrieved from: https://jerz.setonhill.edu/if/crowther/advf4.77-03-11}). \emph{Colossal Cave Adventure}'s source code was indeed only examined due to the recognition of the cultural influence of the game, decades later, and not for its intrinsic properties.

A more contemporary example would be that of the Twine game engine, lowering the barrier to entry for writing interactive fictions in the age of the hyperlink. The result, while aesthetically satisfying, widely recognized and appreciated by the interactive fiction community, nonetheless consists in a single HTML document, comprising well-formatted and understandable HTML and CSS markups, along with three single lines of "uglified" JavaScript\footnote{For instance, the source code of https://pierredepaz.net/-/who/ consists of three lines of 52980 characters, and only 682 whitespace characters}. The explicit process of uglification\footnote{We could expand on this process of uglification, which consists of compacting humanly-laid out source code into the small possible number of characters, usually for a production-ready build, optimized for loading times and dependency processing.} relies on the assumption that no one would, or should, read the source code.

In the case of visual poetry, one can see how the source code of works such as bpNichol's \emph{First Screening}\footnote{https://www.vispo.com/bp/download/FirstScreeningBybpNichol.txt}, is dictated exclusively by the desired output, with a by-product of visually pleasing artifacts throughout the code as foreshadowing the result to come\footnote{Still, a lovely artefact is the subroutine at line 1600, an "offscreen romance" only visible in the source.}. It is a literal description of static, desired output, more akin to a cinematic timeline editor, in which there is a 1:1 relationship between the clips laid out and the final reel, and no room for unexpected developments. While computer-powered, such an example of visual poetry tend to side-step the \emph{potentiality} of computing, of which source code is one of the descriptive symbol systems: each execution of the code is going to be exactly the same as the previous one, and the same as the next one\footnote{Barring any programmer-independent variables, such as hardware and software platform differences.}. While this might be a drastic example, in which unknowns are reduced to a minimum, visual poetry and interactive do rely heavily on the dynamic aspect of computer procedures to create aesthetic experiences\footnote{For instance, see Text Rain, by Camille Utterbach and Romy Achituv}. The difference I am making here is that such aesthetic experience are claimed to take place in the realization of the computer-aided potentials of the work, rather than in the textual description of these potentials\footnote{Tellingly, the Smithsonian Museum, which acquired Text Rain, makes no mention of the source code of the piece.}. These examples, while far from being exhaustive, nonetheless show how little attention is paid to the source code of these works, since they are clearly—and rightly so—not their most important part.

Computer poetry, an artform based on the playful \emph{détournement} of the computer's constraints, gets closer to our topic insofar as the poems generated represent a more direct application of the rule-based paradigm to the syntactical output of the program. Starting with Christopher Stratchey's love letters, generated (and signed!) by MUC, the Manchester Univac Computer, computer poems are generated by algorithmic processes, and as such rely essentially on this particular feature of programming, laying out rules in order to synthesize syntactically and semantically sound natural language poems. Here, the rules themselves matter as much as the output, a fact highlighted by their ratio: a single rule for a seemingly-infinite amount of outputs.

These works and their authors build on a longer tradition of rule-based composition, from Hebrew to the Oulipo and John Cage's indeterministic composition, amongst many others\cite{cramer_words_2003}, a tradition in which creativity and beauty can emerge from within a strict framework of formal rules. Nonetheless, the source code to these works is rarely released in conjunction with their output, hinting again at their lesser importance in terms of their overall artistic values. If computer poetry is composed of two texts, a natural-language output and a computer-language source, only the former is actually considered to be poetry, often leaving the latter in its shadow (as well as, sometimes, it's programmer, an individual sometimes different from the poet). The poem exists through the code, but isn't exclusively limited to the humanly-readable version of the code, as it only comes to life and can be fully appreciated, under the poet's terms, once interpreted or compiled. While much has been written on computer poetry, few of those commentaries focus on the soundness and the beauty of the source as an essential component of the work, and only in recent times have we seen the emergence of close-readings of the source of some of these works for their own sake\footnote{See the publications in the field of Critical Code studies, Software studies and Platform studies.}. These do constitute a body of work centered around the concept of generative aesthetics\cite{goriunova_read_2005}, in which beauty comes from the unpredictable and somewhat complex interplay of rule-based systems, and whose manifestations encompass not only written works, but games, visual and musical works as well; still, this attention to the result make these works fall on the periphery of our current research.

The aspects of electronic literature examined so far still require computer execution in order to be fully realized as aesthetic experiences. We now turn to these works which still function as works of explicit aesthetic value primarily through the reading of their source. We will examine obfuscated code and code poetry (both at the surface level and at the deep level), to finally delimitate our corpus around the last one.

One of the earliest instances of computer source written exclusively to elicit a human emotional reaction, rather than fulfill any immediate, practical function, is perhaps the Apollo 11 Guidance Computer (AGC) code, written in 1969\footnote{Hamilton et. al., 1969, retrieved from https://github.com/chrislgarry/Apollo-11} in Assembler. Cultural references and jokes are peppered throughout the text as comments, asserting computer code as a means of expression beyond exclusively technical tasks\footnote{See also: "Crank that wheel", "Burn Baby Burn"}, and independent from a single writer's preferences, since they passed multiple checks and review processes to end up in the final, submitted and executed document.

\begin{lstlisting}{caption="AGC source code, LUNAR_LANDING_GUIDANCE_EQUATIONS, 1969"}
663  STODL  CG
664	 TTF/8
665  DMP*  VXSC
666	       GAINBRAK,1  # NUMERO MYSTERIOSO
667        ANGTERM
668	 VAD
669		    LAND
670     VSU	RTB
\end{lstlisting}

Code comments allow a programmer to write in their mother tongue, rather than in the computer's, enabling more syntactic and semantic flexibility, and thus reveal a burgeoning desire for programmers to express themselves within their medium of choice.

At the turn of the 1980s, following the transition to programming from an annex practice to full-fledged discipline and profession, along with the development of more expressive programming languages (e.g. Pascal in 1970, C in 1972), software development has become a larger field, growing exponentially\footnote{Source: \url{https://insights.stackoverflow.com/survey/2019#developer-profile-_-years-since-learning-to-code}}, and fostering practices, communities and development styles and patterns\footnote{From Djikstra's Notes on Structured Programming to Knuth's Literate Programming and Martin's Clean Code}. Source code becomes recognized as a text in its own, which can hold qualities and defects of its own, and to which engineering and artistic attention must be paid. No longer a transitional state from formula to binary, it becomes a semantic material, whose layout, organization and syntax are important to the eyes of its writers and readers. Pushing further into the direction of the visual layout of the code, such an endeavour becomes pursued for its own sake, equally important to the need for a program to be functional.

The Obfuscated C Code Contest\footnote{\url{https://ioccc.org}} is the most popular and oldest organized production of such code, in which programmers submit code that is functional and visually meaningful beyond the exclusive standards of well-formatted code. If the source code's meaning was previously entirely subsumed into the output in computer poetry, and if such a meaning existed in parallel in the comments of the AGC routines, pointing at the overlay of computer-related semantics (e.g. line numbers) and human-related semantics (e.g. number of the beast), obfuscated code is a first foray into closely intertwining these separate meanings in the source code itself, making completely transparent, or completely opaque what the code does just by glancing at it.

\pagebreak
\lstinputlisting{r.c}[language=C, Caption="r.c, 1988"]

The above submission to the 1988 IOCCC\footnote{Source: \url{https://web.archive.org/web/20131022114748/http://www0.us.ioccc.org/1988/westley.c}} is a procedure which does exactly what it shows: it deals with a circle. More precisely, it estimates the value of PI by computing its own circumference. While the process is far from being straightforward, relying mainly on bitwise arithmetic operations and a convoluted preprocessor definition, the result is nonetheless very intuitive—the same way that PI is intuitively related to PI. The layout of the code, carefully crafted by introducing whitespace at the necessary locations, doesn't follow any programming practice of indentation, and would probably be useless in any other context, but nonetheless represents another aspect of the \emph{concept} behind the procedure described, not relying on traditional programming syntax\footnote{For such a program, see for instance: \url{https://crypto.stanford.edu/pbc/notes/pi/code.html}}, but rather on an intuitive, human-specific understanding\footnote{Concrete poetry also makes such a use of visual cues in traditional literary works.}.

Obfuscating practices, beyond their technical necessities (for security and efficiency), are traditionally tied to hacking practices, prominently with one-liners. As such, they rely on the brain-teasing process of deciphering, and on the pleasurable, aesthetic experience of resolving and uniting two parallel meanings: what we see in the code, and what it does\footnote{Also known informally as the "Aha!" moment, crucial in puzzle design.}. What we focus on here is the aspect of obfuscation which plays with the different layers of meaning: meaning to the computer, meaning to the human, and different ways of representing and communicating this meaning (from uglifying, to consistent formatting, to depicting a circle with dashes and underscores). While the aesthetics at play in hacking will be further explored below, we focus on the fact that obfuscating code practices, beyond hiding the meaning and the intent of the program, also manifest an attempt to represent such a meaning in different ways, leaving aside traditional code-writing practices and suggesting the meaning of the program by challenging the abilities of human interpretation at play in the process of deciphering programs.

\subsection{Source code poetry}

It is this overlap of meaning which appears as a specific feature of source code poetry. In a broad sense, code poetry conflates classical poetry (as strict syntactical and phonetical form, along with poetic expressivity) with computer code, but it is primarily defined by the fact that it does not require the code to be executed, but only to be read by a human. Following the threads laid out by computer poetry and obfuscated code, code poetry starts from this essential feature of computers to work with strictly defined formal rules, but departs from it in terms of utility. Source code poems are only functional insofar as they are accepted by the intepreter or compiler of the language in which they are written. To the computer, they are indeed functional, in that they are legal and can be parsed; but they do not do anything of \emph{use}. Such formal compliance is only a pre-requisite, a creative constraint, for their human writers.

Within this reliance on creative constraints provided by a computing environment, the emphasis here is on the act of reading, rather than on the act of deciphering, as we've seen with obfuscated code (and in functional code in general). Source code poems are often easy to read\footnote{See perl haikus in particular}, and have an expressive power which operates beyond the common use of programming. Starting from Flusser's approach, I consider poetry as a means to bring concepts into the thinkable, and to crystallize thoughts which are not immediately available to us\cite{flusser_doubt_2014}; through various linguistic techniques, poetry allows us to formulate new concepts and ideas, and to shift perspectives.

In their different manifestations, code poems make the boundary between computer meaning and human meaning thinner and thinner, a feature often afforded by the existence and use of higher-level programming languages. With the development of FLOWMATIC in 1955 by Grace Hopper, it was shown that an English-like syntactical system could be used to communicate concepts for the computer to process. From there, programming languages could be described along a gradient, with binary at the lowest end, and natural language (in an overwheling majority, English) at the highest end. This implies that they could be written and read similarly to English, including word order, pronouncation and interpretation, similar to the error-tolerance of human laguages, which doesn't make the whole communication process fail whenever a specific word, or a word order isn't understood.

The community of programmers writing in Perl\footnote{See: perlmonks, with the spiritual, devoted and communal undertones that such a name implies.} has been one of the most vibrant and productive communities when it comes to code poetry. Such a use of Perl started in 1990, when the language creator Larry Wall shared some of the poems written in the language, and it gained further exposition through the work of Shannon Hopkins\cite{hopkins_camels_1992}. The poem \emph{Black Perl}, submitted anonymously, is a representative example of the productions of this community:

\linespread{1.00}\selectfont
\begin{lstlisting}{language=Perl, Caption="Black Perl, Anonymous, 1990, updated by kck, 2017"}
    #!/usr/bin perl
    no warnings;
    
    BEFOREHAND: close door, each window & exit; wait until time.
      open spellbook, study, read (scan, $elect, tell us);
    write it, print the hex while each watches,
      reverse its, length, write, again;
    kill spiders, pop them, chop, split, kill them.
      unlink arms, shift, wait & listen (listening, wait),
      sort the flock (then, warn "the goats" & kill "the sheep");
    kill them, dump qualms, shift moralities,
      values aside, each one;
    die sheep? die to : reverse { the => system
      ( you accept (reject, respect) ) };
    next step,
      kill `the next sacrifice`, each sacrifice,
      wait, redo ritual until "all the spirits are pleased";
    do { it => "as they say" }.
      do { it => (*everyone***must***participate***in***forbidden**s*e*x*)
    + }.
      return last victim; package body;
    exit crypt (time, times & "half a time") & close it,
      select (quickly) & warn your (next victim);
    AFTERWARDS: tell nobody.
      wait, wait until time;
    wait until next year, next decade;
    sleep, sleep, die yourself,
      die @last 
\end{lstlisting}
\linespread{1.50}\selectfont
\normalfont

The most obvious feature of this code poem is that it can be read by anyone, including by readers without previous programming experience: each word is valid both as English and as Perl. A second feature is the abundant use of verbs. Perl belongs to a family of programming languages grouped under the \emph{imperative} paradigm, which matches a grammatical mood of natural languages, the \emph{imperative mood}. Such mood emphasizes actions to be take rather than, for instance, descriptions of situations, and thus sets a clear tone for the poem. The fact that Perl is based on stating procedures to be executed and states to be changed creates this feeling of relentless urgency when reading through the poem, a constant need to be taking actions, for things to be changed. Here, the native constraints of the programming language interacts directly with the poetic suggestion of the work in a first way: the nature of Perl is that of giving orders, resulting in a poem which addresses \emph{someone} to execute \emph{something}. Still, Perl's flexibility leaves us wondering as to who and what are concerned by these orders. Is the poem directing its words to itself? To the reader? Is Perl just ever talking exclusively to the computer? This ambiguity of the adressee adds to the ominousness of each verse.

The object of each of these predicates presents a different kind of ambiguity: earlier versions of Perl function in such a way that they ignore unknown tokens\footnote{e.g. undefined variables do not cause a core dump.}\footnote{Which results in the poem having to be updated/ported, in this case by someone else than the original writer}. Each of the non-reserved keywords in the poem are therefore, to the Perl interpreter, potentially inexistant, allowing for a large latitude of creative freedom from the writer's part. Such a feature allows for a tension between the strict, untoucheable meaning of Perl's reserved keywords, and the almost infinite combination of variable and procedure names and regular expressions. This tension nonetheless happens within a certain rhythm, resulting from the programming syntax: \lstinline{kill them, dump qualms, shift moralities}, here alternating the computer's lexicon and the poet's, both distinct and nonetheless intertwined to create a \emph{Gestalt}, a whole which is more than the sum of its parts.

A clever use of Perl's handling of undefined variables and execution order allows the writer to use keywords for their human semantics, while subverting their actual computer function. For instance, the \lstinline{die} function should raise an exception, but wrapped within the \lstinline{exit ()} and \lstinline{close} keywords, the command is not interpred and therefore never reaches the execution point, bypassing the abrupt interruption. The subversion here isn't purely semiotic, in the sense of what each individual word means, but rather in how the control flow of the program operates—technical skill is in this case required for artistic skill to be displayed.

Finally, the use of the \lstinline{BEFOREHAND:} and \lstinline{AFTERWARDS:} words mimick computing concepts which do not actually exist in Perl's implementation: the pre-processor and post-processor directives. Present in languages such a C, these specify code which is to be executed respectively before and after the main routine. In this poem, though, these patterns are co-opted to reminisce the reader of the prologue and epilogue sometimes present in literary texts. Again, these seem to be both valid in computer and human terms, and yet seem to come from different realms.

This instance of Perl poetry highlights a couple of concepts that are particularly present in code poetry. While it has technical knowledge of the language in common with obfuscation, it departs from obfuscated works, which operate through syntax compression, by highlighting the expressive power of semiotic ambiguity, giving new meaning to reserved keywords. Such an ambiguity is furthermore bi-directional: the computing keywords become imbued with natural language significance, bringing the lexicon of the machine into the realm of the poetic, while the human-defined variable and procedure names, and of the regular expressions, are chosen as to appear in line with the rhythm and structure of the language. Such a work highlights the co-existence of human and machine meaning inherent to any program text\footnote{Except perhaps those which deal exclusively with scientific and mathematical concepts}.

\pagebreak

\linespread{1.00}\selectfont
\begin{lstlisting}{language=Ruby}
  class Proc
                              def in_discomfort?; :me; end
                                                                  end
  you_are = you = 
     
     ->(you) do
         self.inspect until true
           until nil
                 break you
                                                    end
             puts you.in_discomfort?
               you_are[you]
                                                  end
  
  you[
            you_are
  ]
  
\end{lstlisting}
\linespread{1.50}\selectfont
\normalfont

The poem above, written in Ruby by maca\footnote{\url{https://github.com/maca}} in 2011 and titled \lstinline{self_inspect.rb}, opens up an additional perspective on the relationship between aesthetics and expressivity in source code. Immediately, the layout of the poem is reminiscent both of obfuscated works and of free-verse poetry, such as E.E. Cummings' and Stéphane Mallarmé's works\footnote{Particularly \emph{Un coup de dés jamais n'abolira le hasard.}}. This particular layout highlights the ultimately arbitrary nature of whitespace use in source code formatting: \lstinline{self_inspect.rb} breaks away from the implicit rhythm embraced in \emph{Black Perl}, and links to the topics of the poem (introspection and \emph{unheimlichkeit}) by abandoning what are, ultimately, social conventions, and reorganizing the layout to emphasize both keyword and topic, exemplified in the \lstinline{end} keyword, pushed away at the end of their line.

The poem presents additional features which operate on another level, halfway between the surface and deep structures of the program text. First, the writer makes expressive use of the syntax of Ruby by involving data types. While \emph{Black Perl} remained evasive about the computer semantics of the variables, such semantics take here an integral part. Two data types, the array and the symbol are used not just exclusively as syntactical necessities (since they don't immediately fulfill any essential purpose), but rather as semantic ones. The use of \lstinline{:me} on line 2 is the only occurence of the first-person pronoun, standing out in a poem littered with references to \lstinline{you}. Symbols, unlike variable names, stand for variable or method names. While \lstinline{you} refers to a (hypothetically-)defined value\footnote{A variable name can represent a value and/or a memory address}, a symbol refers to a variable name, a variable name which is here undefined. Such a reference to a first-person pronoun implies at the same time its ever elusiveness. It is here expressed through this specific syntactic use of this particular data type, while the second-person is referred to through regular variable names, possibly closer to an actual definition. It is a subtlety which doesn't have an immediate equivalent in natural language, and by relying on the concept of reference, hints at an essential \emph{différance} between you and me.

Reinforcing this theme of the elusiveness of the self, maca plays with the ambiguity of the value and type of \lstinline{you} and \lstinline{you_are}, until they are revealed to be arrays. Arrays are basic data structures consisting of sequential values, and representing \lstinline{you} as such suggests the concept of the multiplicity of the self, adding another dimension to the theme of elusiveness. The discomfort of the poem's voice comes from, finally, from this lack of clear definition of who \lstinline{you} is. Using \lstinline{you_are} as an index to select an element of an array, subverts the role suggested by the declarative syntax of \emph{you are}. The index, here, doesn't define anything, and yet always refers to something, because of the assigment of its value to what the lambda expression \lstinline{->} returns. This further complicates the poem's attempt at defining the self, returning the reverse expression \lstinline{you_are[you]}. While such an expression might have clear, even simple, semantics when read out loud from a natural language perspective, knowledge of the programing language reveals that such a way to assign value contributes significantly to the poem's expressive abilities.

A final feature exhibited by the poem is the execution of the procedure. When running the code, the result is an endless output of print statements of "me", since Ruby interprets the last statement of a program as a return value to be printed:

\linespread{1.00}\selectfont
\begin{lstlisting}
...
me
me
me
me
me
me
me
me
me
me
me
me
me
me
me
me
Traceback (most recent call last):                 
        11913: from poem.rb:16:in `<main>'
        11912: from poem.rb:13:in `block in <main>'
        11911: from poem.rb:13:in `block in <main>'
        11910: from poem.rb:13:in `block in <main>'
        11909: from poem.rb:13:in `block in <main>'
        11908: from poem.rb:13:in `block in <main>'          
        11907: from poem.rb:13:in `block in <main>'
        11906: from poem.rb:13:in `block in <main>'                             
         ... 11901 levels...                                                    
            4: from poem.rb:13:in `block in <main>'                             
            3: from poem.rb:13:in `block in <main>'                             
            2: from poem.rb:12:in `block in <main>'                             
            1: from poem.rb:12:in `puts'                                        
self_inspect.rb:12:in `puts': stack level too deep (SystemStackError)  
\end{lstlisting}
\linespread{1.50}\selectfont

The computer execution of the poem provides an additional layer of meaning to our human interpretation. Through the assignment of \lstinline{you_are} in an \lstinline{until} loop, the result is an endless succession of the literal interpretation of the symbol \lstinline{:me}, the actual result of being in discomfort. While we've seen that a symbol only refers to something \emph{else}, the concrete\footnote{Both in terms of actual, and in terms of concrete poetry} output of the poem evokes an insistence of the literal self, exhibiting a different tone than a source in which the presence of the pronoun \emph{you} is clearly dominant. Such a duality of concepts is thus represented in the duality of a concise source and of an extensive output, and is punctuated by the ultimate impossibility of the machine to process the accumulation of these intertwined references to \emph{me} and \emph{you}, resulting in a stack overflow error.

The added depth of meaning from this code poem goes beyond the syntactic and semantic interplay immediately visible when reading the source, as the execution provides a result whose meaning depends on the co-existence of both source and output. Beyond keywords, variable names and data structures, it is also the procedure itself which gains expressive power: a poem initially about \emph{you} results in a humanly infinite, but hardware-bounded, series of \emph{me}\footnote{Another productive comparison could be found in Stein's work, \emph{Rose is a rose is a rose...}}.

\subsection{Aesthetics of source code poetry}

These analyses of program texts have highlighted some of the aesthetic features of source code which can elicit a poetic experience during both reading and execution. These can be further qualified through several concepts, which I introduce and extend here.

The first is \textbf{double-meaning}, taken from Camille Paloque-Bergès's work on networked texts, and her analysis of code poetics\cite{paloque-berges_poetique_2009}. She defines it as the affordance provided by the English-like syntax of keywords reserved for programming to act as natural-language signifiers. As we've seen in \emph{Black Perl}, the Perl functions can indeed be interpreted as regular words when the source is read as a human text. Starting from her analysis of \emph{codeworks}, a body of literature centered around a créole language halfway between humanspeak and computerspeak\footnote{See in particular the work of Alan Sondheim and mezangelle}, it can be extended into the aesthetically productive overlap of syntactic realms.

Previous research by Philippe Bootz has also highlighted the concept of the \emph{double-text} in the context of computer poetry, a text which exists both in its prototypal, virtual, imagined form, under its source manifestation, and which exists as an instantiated, realized one\cite{bootz_problem_2005}. However, he asserts that, in its virtual form, "a work has no reality", specifically because  it is not realized. Here again, we encounter the dependence of the source on its realized output, indeed a defining feature of the generative aesthetics of computer poetry. As we've seen in the \lstinline{self_inspect.rb} poem, a work of code poetry can very much exist in its prototypal form, with its output providing only additional meaning, further qualifying the themes laid out in source beforehand. Indeed, the output of that poem would have a drastically diminished semantic richness if the source isn't also read and understood. For this double-meaning to take place, we can say that the sitation is inverted: the output becomes the virtual, imagined text, while the source is the concrete instantiation of the poem.

Second, we draw on Geoff Cox and Alex McLean's concept of \textbf{double-coding}\cite{cox_speaking_2013}. According to them, double-coding \emph{"exemplifies the material aspects of code both on a functional and an expressive level"} (p.9). Cox and McLean's work, in a thorough exploration of source code as an expressive medium, focus on the political features of speaking through code, as a subversive praxis. They work on the broad social implications of written and spoken\footnote{Which they conflate with the practice of live-coding} code, rather than exclusively on the specific features of what makes source code expressive in the first place. Double-coding nonetheless helps us identify the unique structural features of programming languages which support this expressivity. As we've briefly investigated, notably through the use of data types such as symbols and arrays in source code poetry, programming languages and their syntax hold within them a specific kind of semantics which hold, for those who are familiar with them and understand them, expressive power, once the data type is understood both in its literal sense, and in its metaphorical one. The succint and relevant use of these linguistic features can thicken the meaning of a poem, bringing into the realm of the thinkable ways to approach metaphysical topics.

Finally, the tight coupling of the source code and the executed result brings up Ian Bogost's concept of \textbf{procedural rhetoric}\cite{bogost_rhetoric_2007}. Bogost presents procedures as a novel means of persuasion, along verbal and visual rhetorics. Working within the realm of videogames, he outlines that the design and execution of processes afford particular stances which, in turn, influence a specific worldview, and therefore arguing for the validity of its existence. Further work has shown that source code examination can already represent these procedures, and hence construct a potential dynamic world from the source\cite{tirrell_dumb_2012}. If procedures are expressive, if they can map to particular versions of a world which the player/reader experiences\footnote{Versions of worlds can be explored further through Goodman's \emph{Ways of Worldmaking}}, then it can be said that their textual description can also already persuasive, and elicit both rational and emotional reactions due to their depiction of higher-order concepts (e.g. consumption, urbanism, identity, morality). As its prototypal version, source code acts as the pre-requisite for such a rhetoric, and part of its expressive power lies in the procedures it deploys (whether from value assignment, execution jumps or from its overall paradigms\footnote{e.g. declarative, imperative, functional}). Manifested at the surface level through code, these procedures however run deeper into the conceptual structure of the program text, and such conceptual structures can nonetheless be echoed in the lived experiences of the reader.

We've seen through this section that the poetic expressivity of source code poems rely on several aesthetic mechanisms, which can be combined for further expressive effect. From layout and syntactic obfuscation, to double-meaning through variables and procedure names, double-coding and the integration of data types and functional code into a program text and a rhetoric of procedures in their written form, all of these activate the connection between programming concepts and human concepts to bring the unthinkable within the reach of the thinkable. The next section will explore this connection further, in terms of mental models, literary metaphors and cognitive structures.

\section{Metaphors and Mental Models}

This section focuses on metaphors and mental models in both programming and literature, and how they're related to understanding and cognition, both in broad terms and in specific texts. The evokative power of the excerpts seen above make ample use of the multiple facets of understanding, switching from one frame to the other. This switch relates to the most commonly used definition of metaphor: that of labeling one thing in terms of another, thereby granting additional meaning to the subject at hand. Our approach  here will bypass some of the more minute distinctions made between metonymy (in which the two things mentioned are already conceptually closely related), comparison (explicitly assessing differences and similarities between two things, often from a value-based perspective) and synechdoche (representing a whole by a subset), as they all relate to a larger, more contemporary definition of the concept.

\subsection{Lakoff and Ricoeur}

This part of the thesis relies especially on the works of George Lakoff and Mark Johnson, and of Paul Ricoeur, due to their requalification of the nature and role of metaphor in the 20th century. While Lakoff and Johnson's approach to the conceptual metaphor will serve a basis to explore metaphors in the broad sense across software and narrative, I also argue that Ricoeur's focus on the tension of the \emph{statement} rather than primarily on the \emph{word} will help us better understand some of the aesthetic manifestations of software metaphors, without being limited to tokens. Following a brief overview of their contributions, I examine the various uses of metaphor in software and in literature, touch upon the cognitive turn in literary studies, and conclude the section by the ambiguity of a cognitive account of programming.

Lakoff and Johsnon's seminal work develops a theory of conceptual metaphors by highlighting their essential dependence on pre-existing cognitive structures, which we associate with already-understood concepts. The metaphor maps a source domain (made up of cognitive structure(s)) to a target domain. In the process, they extend the field of applicability of metaphors from the strictly literary to the broadly cultural: metaphors work because each of us has some conception of those domains involved in the metaphorical process. Metaphors rely in part on a static understanding, resulting in a fixed meaning from the application of a given source to a given target. Some of these sources are called \emph{schemas}, and are defined enough to not be mistaken for something else, but broad enough to allow for multiple variants of itself to be applied to various targets\cite{lakoff_metaphors_1980}, providing both diversity and reliability. As we will see below, their approach allows us to focus not just on textual objects, but on the vast range of metaphors used specifically in computing-related environments. Given that the source of the metaphor should be grounded, with as little invariablity as possible, in order to qualify a potentially ill-defined target domain, this provides us with a first foray into the inherent elusiveness and instability of computing when presented to a broader audience.

Going beyond the role of metaphors manifested in expressions such as \emph{the desktop}, \emph{the mouse}, or \emph{the cloud}, we will explore Lakoff's understanding of the specifically poetic metaphor further below as preliminary work to assess the linguistic component of computing—source code. For now, to complement his broadening of the metaphorical field, we turn to Paul Ricoeur's assessment of it.

Writing in \emph{The Rule of Metaphor}, Ricoeur operates two shifts which will help us better assess not just the inherent complexity of program texts, but the ambivalence of programming languages as well. His first shift regards the locus of the metaphor, which he saw as being limited to the single word—a semiotic element—to the whole sentence—a semantic element\cite{ricoeur_rule_2003}. This operates in parallel with his attention to the \emph{lived} feature of the metaphor, insofar it exists in a broader, vital, experienced context. Approaching the metaphor while limiting it to words is counterproductive because words refer back to "contextually missing parts"—they are eminently overdetermined, polysemic, and belong to a wider network meaning than a single, one-to-one relationship\footnote{As he sees it in the traditional, Aristotelician sense of the term.}. Looking at it from the perspective of the sentence brings this rich network of potential meanings and broadens the scope for interpretation. As we've briefly touched upon in the previous section when reading \lstinline{self_inspect.rb}, all of the evocative meaning of the poem isn't contained exclusively in each token, and the power of the whole is greater than the sum of its parts.

Secondly, Ricoeur inspects a defining aspect of a metaphor by the \emph{tensions} it creates. His analysis builds from the polarities he identifies in discourse between event (time-bound) and meaning (timeless), between individual (subjective, located) and universal (applicable to all) and between sense (definite) and reference (indefinite)\footnote{For the extent to which source code can be considered discourse has been discussed, see: Cox and McLean, \emph{Speaking Code}.}. The creative power of the metaphor is its ability to both create and resolve these tensions, to maintain a balance between a literal interpretation, and a metaphorical one—between the immediate and the potential, so to speak. Tying it to the need for language to be fully realized in the lived experience, he poses metaphor as a means to creatively redescribe reality. As we will be approaching the topic of syntax and semantics in programming languages, we will see that these tensions can be a fertile ground for poetic creation through aesthetic manifestations.

\subsection{In literature}

If the conceptual turn initiated by Lakoff and Johnson's analysis of the metaphor broadens the horizon of their applicability beyond the strict domain of literature, it is nonetheless clear that metaphors appear and operate in particular ways in literary works, from fiction to poetry. We look at such specificity here in anticipation of identifying which features of poetic metaphors could be mapped to the program texts of our corpus—whether explicitly poetic, as in source code poetry, or not, as in regular source code.

So while Lakoff bases poetic metaphors on the broader metaphors of the everyday life, he also operates the distinction that, contrary to conventional metaphors which are so widely accepted that they go unnoticed, the poetic metaphor is \emph{non-obvious}. Which is not to say that it is convoluted, but rather that it is new, unexpected, that it brings something previously not thought of into the company of broad, conventional metaphors—concepts we can all relate to because of the conceptual structures we are already carry with us, or are able to easily integrate. This echoes our mention of Flusser's analysis of poetry as that which brings ideas into the realm of the thinkable.

It does so along four different axes, in terms of how the source domain affects the target domain that is connected to. First, a source domain can \emph{extend} its target counterpart: it pushes it in an already expected direction, but does so even further, sometimes creating a dramatic effect by this movement from conventional to poetic. For instance, a conventional metaphor would be saying that \emph{"Juliet is radiant"}, while a poetic one might extend the attribution of positivity associated with brightness by saying \emph{"Juliet is the sun}\footnote{From \emph{Romeo and Juliet}, Act 2, Scene 2}.

Poetic metaphors can also \emph{elaborate}, by adding more dimensions to the target domain, while nonetheless being related to its original dimension. Here, dimensions are themselves categories within which the target domain usually falls (e.g. the sun has an astral dimension, and a sensual dimension). Naming oneself as \emph{The Sun-King} brings forth the additional dimension of hierarchy, along with a specific role within that hierarchy—the sun being at the center of the then-known universe.

Metaphors gain poetic value when they \emph{put into question} the conventional approaches of reasoning about, and with, a certain target domain. Here is perhaps the most obvious manifestation of the \emph{non-obvious} requirement, since it quite literally proposes something that is unexpected from a conventional standpoint. When Camus describes Tipasa's countryside as being \emph{blackened from the sun}\footnote{"\emph{A certaines heures, la campagne est noire de soleil}", from \emph{Noces à Tipasa}}, it subverts our pre-conceptions about what the countryside is, what the sun does, and hints at a semantic depth which would go on to support a whole philosophical thought (\emph{la pensée de midi}). Interestingly, the re-edition of L'Étranger for its 70th anniversary can itself be seen as a form of poetic metaphor, since it was published under Gallimard's \emph{Futuropolis} collection. While the actual \emph{Futuropolis} doesn't claim to focus on any sort of science-fiction publications, and rather on illustrations, the very name of the collection applies onto the work of Camus, and of the others published alongside him, can elicit in the reader a sense of a kind of avant-gardism that is still present today.

Finally, poetic metaphors \emph{compose} multiple metaphors into one, drawing from different source domains in order to extend, elaborate, or question the original understanding of the target domain. Such a technique of superimposition creates semantic depth by layering these different approaches. It is particularly at this point that literary criticism and hermeneutics appear to be necessary to expose some of the threads pointed out by this process. As an example, the metaphor of Charles Bovary's cap, a drawn-out metaphorin Flaubert's work which ends up depicting something which clearly isn't a cap, operates by extending the literal understanding of how a cap is constructed, elaborating on the different components of a hat in such a rich and lush manner that it leads the reader to question whether we are still talking about a hat. This metaphorical composition can be interpreted as standing for the orientalist stance which Flaubert takes vis-à-vis his protagonists, or for the absurdity of material pursuit and ornament\footnote{Which ultimately leads Emma to her demise.}, or for the novel itself, whose structure itself is composed of complex layers, under the guise of banal appearances. Composed metaphors highlight how they exist along \emph{degrees of meanings}, from the conventional to the poetic, and further to the non-sensical. This difference of degree, rather than of kind, is one I ascribe to when it comes to delimiting corpus of the present research in different ways of writing and reading code—writing code as poetry, as tool, as a hack or as research aren't absolutely siloed off from each other.

Through these, Lakoff and Johnson highlight how metaphors \emph{function}, and how they can be identified. Another issue they address is that of the \emph{role} they fulfill in our everyday experiences as well as in our aesthetic experiences. Granted a propensity to structure, to adapt, to reason and to induce value judgment, metaphors are ultimately seen as a means to comprehend the world. By importing structure from the source, the metaphor in turn creates structure in our lives, in our understandings (and thus have power over us). Our understanding grasps these structures through their features and attributes (one might even call them affordances, following Gibson\cite{gibson_ecological_1986}), and integrates them as a given—in what Ricoeur would call a \emph{dead} metaphor. This is one of their key contribution, that metaphors have a function which goes beyond an exclusive, disinterested, self-referential, artistic role. If metaphors are ornament, it is far from being a crime, because these are ornaments which, in combining imagination and truth, expand our conceptions of the world by making things \emph{fit} in new ways.

This approach of beauty as means to understand however predates Lakoff and Johnson. Through his contribution to aesthetic philosophy, Monroe Beardsley's started touching upon metaphor from a semantic perspective. Published alongside his inquiries into the aesthetic character of an experience, and taken later on by Ricoeur as a basis for his study, \emph{The Metaphorical Twist} implies that semantics and aesthetics might be connected through the structuring operation of the metaphor—that which elicits an aesthetic experience can do so through the creation of unexpected, or previously unattainable meaning. Ricoeur's theory of the metaphor indeed builds on Beardsley's conception that metaphor can have a designative role (the primary subject) which adds a \emph{"local texture of irrelevance"}, a \emph{"foreign component"}, whose semantic richness might over-reach and obfuscate the intended meaning, as well as a connotative one (the secondary subject), in which meaning is peripheral. For Ricoeur, it is indeed literary criticism, beyond logical grammar and linguistics, which hold the key to understanding metaphors. Through an analysis of Beardsley's work, he highlights the metaphor-induced tension, between central and periphery, between illuminating and obfuscating, between evidence and irrelevance.

As Beardsley inquiries into the features necessary for an aesthetic experience, of which the metaphor is part, he lists five criteria to distinguish the character of such an experience. Besides object-directedness, felt-freedom, detached-affect and wholeness, is the criteria of \emph{active discovery}, which is

\begin{quote}
  "a sense of actively exercising the constructive powers of the mind, of being challenged by a variety of potentially conflicting stimuli to try and make them cohere; exhilaration in seeing connections between percepts and meanings; a sense of intelligibility"\footnote{The Aesthetic Experience, in The Aesthetic Point of View\cite{beardsley_aesthetic_1970}.}
\end{quote}

As such, Beardsley highlights the possibility of an aesthetic experience to make understandable, to unlock new knowledge in the beholder, and he considers metaphors as a way to do so. The stages he lists go from (1) the word exhibiting properties, to (2) those properties being made into meaning, and finally into (3) a staple of the object, consolidating into (or dying from becoming) a commonplace. This interplay of a metaphor being integrated into our everyday mental structures, of poetry bringing forth into the thinkable, and in metaphor creating a tension for such bringing-forth to happen, makes the case for at least one of the consequences of an aesthetic experience, and therefore one of its functions: making sense of the complex concepts of world.

More recent work in aesthetics and literary research have continued in this direction. Building on the focus on conceptual structures, the attention has shifted to the relationship between literature (as part of aesthetic work and eliciting aesthetic experiences) and cognition. This move starts from the limitation of explaing "art for art's sake", and inscribing it into the real, lived experiences of everyday life mentioned above, perhaps best illustrated by the question posed in Jean-Marie Schaeffer's eponymous work—\emph{Why fiction?}. Indeed, if literary and aesthetic criticism are to be rooted in the everyday, and in the conventional conceptual metaphors which structure our lives, our brains seem to be the lowest common denominator, and thus a good starting point for a new contribution to understanding the arts. A similar approach, related to scientific knowledge, can be seen in Michael Polanyi's work on tacit knowledge, in which that which the scientist knows isn't entirely and absolutely formal and abstracted, but rather embodied, implicit, experiential. This limitation of codified, rigorous language when it comes to communicating knowledge, opens up the door for an investigation of how literature and art can help with this communication, while keeping in mind the essential role of the senses and lived experience in knowledge acquisition (i.e. integration of new conceptual structures)\cite{polanyi_tacit_2009}.

Some of the cognitive benefits of art aren't too dis-similar to those posed by Beardsley, but shift their rationale from strict hermeneutics and criticism to cognitive science. These benefits can be pleasure, emotion, or understanding. Terrence Cave focuses on the latter when he says that literature \emph{"allows us to think things that are difficult to think otherwise}. We now examine such a possibility from two perspectives: in terms of the role of imagination, and in terms of the role of the senses.

Harris posits that literature is an object of knowledge, a creator of knowledge, and that it does so through the interplay between rational thought and imaginative thought, between the "counterfactual imagination" and our daily lives and experiences. Through this tension, this suspension of disbelief is nonetheless accompanied by an epistemic awareness, making fiction reliant on non-fiction, and vice-versa. Working on literary allusions, Ziva Ben-Porat shows that this simultaneous activation of two texts is influenced by several factors. First, the form of the linguistic token itself has a large influence over the understanding of what it alludes to. Its aesthetic manifestation, then, can be said to modulate the conceptual structures which will be acquired by the reader. Second, the context in which the alluding token(s) appears also influences the correct interpretation of such an allusion, and thus the overall understanding of the text. This contextual approach, once again hints at the change of scale that Ricoeur points in his shift from the word to the sentence, and demands that we focus on the whole, rather than single out isolated instances of linguistic beauty. Finally, a third factor is the personal bagage (a personal encyclopedia) brought by the reader. Such a bagage consists of varying experience levels, of quality of the know-how that is to be activated during the reading process, and of the cognitive schemas that readers carry with them. Imagination in literary interpretation, builds on these various aspect, from the very concrete form and choice of the words used, to the unspoken knowledge structures held in the reader's mind, themselves depending on varied experience levels. By allowing the reader to project themselves into potential scenarios, imagination allows us to test out possibilities and crystallize the most useful ones to continue building our conception of the fictional world.

The work of imagination also relies on how the written word can elicit the recall of sensations. This takes place through the re-creation, the evokation of sensory phenomena in linguistic terms, such as the \emph{perceptual modeling}\footnote{Elane Scarry's expression} of literary works, which she defines as (linguistic) simulations relying on the senses to communicate situations, concepts, and potential realities. Depiciting movement, vision, tactility and other embodied sensations allows us to crystallize and verify the work of the imaginative process. As such, literature unleashes our imaginary by recreating sensual experiences—Lakoff even goes as far as saying that we can only imagine abstract concepts if we can represent them in space\footnote{Geoff Hinton, pioneer of modern deep-learning, has reportedly said that, to visualize 100-dimensional spaces, one should first visualize a 3-dimensional, and then "shout 100 really really loud, over and over again", source: \url{https://medium.com/artists-and-machine-intelligence/a-journey-through-multiple-dimensions-and-transformations-in-space-the-final-frontier-d8435d81ca51}}. It seems that the imaginative process depends in part on visual and spatial projections, and suggests the fitness of the conceptual structures depicted. By describing situations which, while fictional, nonetheless are possible in a reality often very similar to the one we live in, it is easy for the reader to connect and understand the point being made by the author. So if literature is an object of knowledge, both sensual and conceptual, offering an interplay between rational and imaginative thought, it still relies on the depiction of mostly familiar situations (the protagonists physiologies, the rules of gravity, the fundamental social norms are rarely challenged). A first issue that we encounter here, in trying to connect source code and computing to this line of thought, is that code has close to no sensual existence, beyond its textual form. In trying to communicate concepts, states and processes related to code and computing, and in being unable to depict them by their own material and sensual properties, we once again resort to linguistic abstraction processes, including metaphor.

\subsection{In Software}

It is interesting to consider that the first metaphor in computing might be concommitant with the first instance of modern computing—the Turing \emph{machine}. While Turing machines are widely understood as being manifested into what we call computers (laptops, tablets, smartphones, etc.), and thus definitely within the realm of machines, the Turing machine isn't strictly a machine \emph{per se}. Rather, it is more accurately defined as a mathematical model which in turn defines an abstract machine. Humans can be considered Turing machines (and, in fact, one of the implicit requirements of the Turing machine is that, given enough time and resources, a human should be able to compute anything that the Turing machine can compute), and non-humans can also be considered Turing machines\footnote{See research in biological computing, using DNA and protein to perform computational tasks}. Debates in computer science related to the nature of computing\cite{rapaport_philosophy_2005} have shown that computation is far from being easily reduced to a simple mechnical concern, and the complexity of the concept is perhaps why we ultimately revert to metaphors in order to better grasp them.

Jumping ahead to the 1980s, these uses of metaphors became more widespread and entered public discourse once personal computing became available to ever larger audiences. With the release of the XEROX Star, features of the computer which were until then described as data processing were given a new life in entering the public discourse. The Star was seminal since it introduced technological innovations such as a bitmapped display, a two-button mouse, a window-based display including icons and folders. For instance, the desktop metaphor relies on previous understanding of what a desktop is, and what it is used for in the context of physical office-work; since early personal computers were marketed for business applications (such as the Star), these metaphors built on the broad cognitive structures of the user-base in order to help them make sense of this new tool. Paul DuGay, in his cultural study of the Sony Walkman, makes a similar statement when he describes the Sony Walkman, a never-before-seen compound of technological innovations, in terms of pre-existing, and well-established technologies\cite{gay_doing_2013}. The icon of a floppy disk for writing data to disk, the sound of wrinkled paper for removing data from disk, the designation of a broad network of satellite, underground and undersea communications as a cloud, these are all metaphors which help us make sense of the broad possibilities brought forth by the computing revolution.

The work of metaphors takes on an additional dimension when we introduce the concept of interfaces. As permeable membranes which enable (inter)actions between the human and the machine, they are essential insofar as they allow for various kinds of agency, based on different degrees of understanding. Departing from the physically passive posture of the reader towards an active engagement with a dynamic system, interfaces highlight even further the cognitive role of the metaphor. These depictions of things-as-other-things influence the mental model which we build of the computer system we engage in. For instance, the prevalent windows metaphor of our contemporary desktop and laptop environments obfuscates the very concrete action of the CPU (or CPUs, in the case of multi-core architecture) of executing one thing at a time, except at speeds which cannot be intuitively grasped by human perception. Alexander Galloway 's work on interfaces as metaphorical representations suggests a similar concern when he bases it on Jameson's theory of cognitive mapping. While Jameson uses it in a political and historical context, the heuristic is nonetheless useful here: cognitive mapping is the process by which the individual subject situates himself within a vaster, unrepresentable totality, a process that corresponds to the workings of ideology. Substituting ideology with the computer\footnote{The relation between which has been explored by Galloway, Chun, Holmes and others, and is particularly apparent in how an operating system is designated in French: \emph{système d'exploitation}.}, we can see how such a process helps make sense of the unthinkable, of that which is too complex to grasp and therefore must be put into symbols (words, icons, sounds, etc.).

Moving away from userland, in which most of these metaphors exist, we now turn to examine the kinds of metaphors that are used by programmers and computer scientists themselves. Since the sensual reality of the computer is that it is a high-frequency vibration of electricity, one of the first steps taken to productively engage with computers is that of abstraction. The word computer itself can be considered as an abstraction: originally used to designate the women manually inputting the algorithms in room-scale mainframes, the distinction between the machine and its operator was considered to be unnecessary. The relation between metaphor and abstraction is a complex one, but we can say that metaphorical thought requires abstraction, and that the process of abstraction ultimately implies designating one thing by the name of another (a woman by a machine's, or a machine by a woman's), being able to use it interchangeably, and therefore lowering the cognitive friction inherent to the process of specification, freeing up mental resources to focus on the problem at hand.

This need to get away from the specificities of the machines has been one of the essential drives in the development of programming languages. Since we cannot easily and intuitively deal with binary notation to represent complex concepts, programming helps us deal with this hurdle by presenting things in terms of other things. Most fundamentally, we represent binary signs in terms of English language (e.g. from binary to Assembler). This is, again, by no means a metaphorical process, but rather an encoding process, in which tokens are being separated and parsed into specific values, which are then processed by the CPU as binary signs. Still, this abstraction layer offered by programming languages allowed us to focus on \emph{what} we want to do, rather than on \emph{how} to do it. The metaphorical aspect comes in when the issue of interpretation arises, as the possibility to deal with more complex concepts required us to grasp them in a non-rigorous way, one which would have a one-to-one mapping between concepts. Allen Newell and Herbert A. Simon, in their 1975 Turing Award lecture, offer a good example of symbolic (i.e. conceptual) manipulation relates inherently to understanding and interpretation:

\begin{quote}
  In none of [Turing and Church's] systems is there, on the surface, a concept of the symbol as something that \emph{designates}.
\end{quote}

The complement to what he calls the work of Turing and Church as automatic formal symbol manipulation is to be completed by this process of \emph{interpretation}, which they define simply as the ability of a system to designate an expression and to execute it. We encounter here one of the essential qualities of programming languages: the ambivalence of the term \emph{interpretation}. A machine interpretation is clearly different from a human interpretation: in fact, most people understand binary as the system comprised of two numbers, 0 and 1, when really it is intepreted by the computer as a system of two distinct signs (red and blue, Alex and Max, hot and cold, etc.). To assist in the process of human interpretation, I argue that metaphors have played a part in helping programmers construct useful mental representations related to computing. These metaphors can go both ways: helping humans understand computing concepts, and to a certain extent, helping computers understand human concepts.

Perhaps one of the first metaphors a programmer encounters when learning about the discipline is that which states that the function is like a kitchen recipe. You specify a series of instructions which, given some input ingredients (arguments), result in an output result (return value). The difficulty in explaining, in that context, the need for a \emph{void} keyword to individuals with limited experience and knowledge of how programming works is a good example of the non-straightforwardness of computing concepts. Similarly, the use of the term \emph{server} is conventionally associated and represented as a machine sending back data when asked for it, when really it is nothing but an executed script or process running on said machine. Incidentally, a server is also a style of software architecture, to which we will return later.

Another instance of synbolic use relying on metaphorical interpretation can be found in the word \emph{stream}. Originally designating a flow of water within its bed, it has been gradually accepted as designating a continuous flow of contingent binary signs. \emph{Memory}, in turn, stands for record, and is stripped down of its essentially partial, subjective and fantasized aspects usually highlighted in literary works (perhaps \emph{volatile memory} gets closer to that point). Finally, \emph{objects}, which came to prominence with the rise of object-oriented programming, have only little to do with the physical properties of objects, with no affordance for being traded, for acting as social symbols, for gaining intrinsic value, but rather the word is used as such for highlighting its boundedness, and ability to be manipulated without interfering with other objects.

Most of these designations, stating a thing in terms of another aren't metaphors in the full-blown, poetic sense, but they do hint at the need to represent complex concepts into humanly-graspable terms. The need for these is only semantic insofar as it allows for an intended interaction with the computer to be carried out successfully—e.g. one has an intuitive understanding that interrupting a stream is an action which might result in incompleteness of the whole. This process of linguistic abstraction doesn't actually require clear definitions for the concepts involved. The example of the terminology in modern so-called cloud computing uses a variety of terms stacked up to each other in what might seem to have no clear \emph{denotative} meaning (e.g. Google Cloud Platform offers \emph{Virtual machine compute instances}), but nonetheless have a clear \emph{operative} meaning (e.g. the thing on which my code runs). This further qualifies the complexity of the sense-making process in dealing with computers: we don't actually need to truly understand what is precisely meant by a particular word, as long as we use it in a way which results in the expected outcome\footnote{See the famous comment in the UNIX source: \emph{You are not expected to understand this}.}.

The reverse process also brings forth issues of conceptual representation through formal symbolic means. The work of early artifical intelligence researchers consists not just in making machines perform intelligent tasks, but also implies that intelligence itself should be clearly and inambiguously represented. The work of Terry Winograd, for instance, was concerned with language processing (intepretation and generation)\cite{winograd_language_1982}. Through his inquiry, he touches on the different ways to represent the concept of language in machine-operational terms, and highlights two possible represenations which would allow a computer to interact meaningfully with language. He considers a \emph{procedural} representation of language, one which is based on algorithms and rules to follow in order generate an accurate linguistic model, and a \emph{declarative} representation of language, which relies on data structures which are then populated in order to create valid sentences. At the beginning of his exposé, he introduces the historically successive metaphors which we have used to build an accurate mental representation of language (language as law, language as biology, language as chemistry, language as mathematics). As such, we also try to present language in other terms than itself in order to make it actionable within a computing environment.

As we've seen, metaphors are implicitly known not to be true in their most literal sense. Max Black in \emph{Models and Metaphors} argues that metaphors are too loose to be useful in analytic philosophy, and therefore too loose for programming languages, heavily based on the analytic tradition. Yet, they still rely heavily on models in order to make human concepts graspable and operation to the computer. These tools deployed during the representational process differ from conventional or poetic metaphors insofar as they can be logically operated upon and therefore empirically verifiable or falsifiable. These models are means through which we aim at taking the conceptual structures on which metaphors also operate, and explicit them in formal symbol systems\footnote{For a further inquiry of models and theories, see Weizenbaum in \emph{Computer Power and Human Reason}}.

Abstraction, metaphors and synmbolic representations are thus useful tools when it comes to computing, in terms of trying to represent to ourselves what it is that a computer can and effectively does, and in terms of explaining to the computer what it is we're trying to operate on (from an integer, to a non-ASCII word, to a renewable phone subscription or to human language).

\subsection{Programming and psychology}

So metaphors work in software because they do not exist just within literature, and yet remain too vague for a strict computer interpretation. Such a computer interpretation, in turn, is too complex and fine-grained for most individuals interacting with them (from end-users to most programmers and computer scientists) to be useful. The conclusion we establish here, is about how connections between mental models relate to the process of understanding, at the overlap between human understanding and computer understanding, and how aesthetic experience can affect this encounter.

The mental model offers a good starting point for exploring this overlap. A mental model, as a kind of internal symbolic representation of external reality, is a more rigorous and formal conceptual structure than a metaphor—which only offers a broad direction through evokative power, rather than an actionable basis. They are related to knowledge, since the construction of accurate and useful mental models through the process of understanding underpins knowledge acquisition. However, mental models need not be correlated with empirical truth, but extensive enough to be described by logical means. Mental models can be informed, constructed or further qualified by the use of metaphors, but they are nonetheless more precise than the cognitive structures on which metaphors rely—a mental model can be seen as a more specific instance of a conceptual structure. The term \emph{schema}, used in cognition-influenced literary studies, again following Lakoff, is here a point of entry into the psychology of computer programming.

Francoise Détienne, in her study of how computer programmers design and understand programs\cite{detienne_software_2012}, defines the activity of designing programs in activating schemas, mental representations that are abstract enough to encompass a wide use (web servers all share a common schema in terms of dealing with requests and responses), but nonetheless specific enough to be useful (requests and responses are qualitatively different subsets of the broader concept of inputs and outputs). This flexibility is useful when one needs to deal with two aspects of working within a programming environment. An added complexity to the task of programming comes with the dual nature of the mental models needing to be activated: the computer's actions and responses are comprised of the prescriptive (what the computer should do) to the effective (what the computer actually does), one of the tensions at the heart of computer programming. In order to be appropriately dealt with, then, programmers must activate and refine mental models of a program which resolves this tension.

In programming, within a given context—which include goals and heuristics—, elements are being perceived, processed through existing knowledge schemas in order to extract meaning. Starting from Kintsch and Van Dijk's approach of understanding text\cite{kintsch_toward_1978}, she nonetheless highlights some differences. In program texts, there is an entanglement of the plan, of the arc, of the tension, which does not happen so often in most of the traditional narrative text. A programmer can jump between lines and files in a non-linear, explorative manner. Program texts are also dynamic, procedural texts, which exhibit complex causal relations between states and events, which need to be kept track of in order to resolve the presctiptive/effective discrepancies. Finally, the understanding of program text is first a general one, which only subsequently applies to a particular situation (a fix or an extension needing to be written), while narrative texts tend to focus on specific instances of protagonists, scenes and descriptions.

A similarity in understanding program texts and narrative texts is that the sources of information for understanding either are: the text itself, the individual experience and the broader environment in which the text is located (e.g. technical, social). Building on Chomsky's concepts, the activity of understanding in programming can be seen as understanding the \emph{deep structure} of a text throuh its \emph{surface structure}\cite{chomsky_aspects_1965}. One of the heuristics deployed to achieve such a goal is looking out for what she calls \emph{beacons}, as thematic organizers which structure the reading and understanding process. However, one of the questions that isn't answered specifically, and which is the aim of this thesis, is to highlight how does the specific surface structure in programming result in the understanding of the deep structure.

Additional recent research in the cognitive responses to programming tasks, conducted by Ivanova et. al., do not appear to settle the question of whether programming is rather dependent on language processing brain functions, or on functions related to mathematics (which do not rely on the language part of the brain)\cite{ivanova_comprehension_2020}. They conclude that, while language processing might not be one of the essential ways that we process code, it also does not rely on exclusively mathematical functions. Stimulating in particular the multi-demand system, it seems that programming is a polymorphous activity involving multiple exchanges between different brain functions. What this implies, though, is that neither literature nor linguistics should be the only lens through which we look at code.

Going back to research in contemporary literary studies can start laying out threads of an answer. Jérôme Pelletier uses Carl Plantinga to define emotional responses in the face of aesthetic objects as dual: either one has an emotional response to the artefact itself (surface), of an emotional response to what it represents (deep). In the context of reading fiction, the reader is helped in their understanding by looking out for \emph{guides} or \emph{props}\footnote{Currie, 1990}, which are similar to the \emph{beacons} emphasized by Détienne. A notable difference is that the guides are suggested, implied, left as traces for the reader to subtly construct (as in the case of the cap metaphor in \emph{Madame Bovary}), rather than explicitly stated throughout the program text (usually most obviously in the form of comments). However, we've seen previously that the use of comments is, by most programmers, not considered to be an aesthetic feature of an inspected source code, hinting at the fact that subtlety (useful subtlety), might be a desired attribute of beautiful code.

Programming is then fiction, in that the pinpointing of its source of existence is difficult, and in that it affords the experience of imagining contents of which one is not the source, and of which the certainty of isn't defined. Furthermore, both programming and fiction suggest surface-level guiding points helping the process of constructing mental models and conceptual representations. It is also non-fiction, in that it deals with concrete issues and problems (more often than not, a pestering bug), and that it provides a pragmatic frame for processing representations, in which assumptions stemming from burgeoning mental models can be easily verified or falsified. It might then be appropriate to treat it as such, simultaneously fiction and non-fiction.

To conclude this section, then, we can turn to Jerome Bruner, who considers that art allows us to \emph{"reading in others' minds"}, to anticipate what a writer has been intending for us to understand through their text, eithe program or narrative. This intent component relates to the interpretation issue mentioned above: the interpretation of the machine is different from the interpretation of the human, and therefore what also needs to be intepreted is the intent of the author. Reading is then akin to constructing a \emph{cognitive cartography}, allowing for an experience to be made intelligible, sensible. The repeated implication of spatial and visual components of metaphors and mental models allows us to consider metaphors as an architecture of thought\cite{forsythe_cathedrals_1986}. The next section is therefore dedicated to examining more closely the parallels between software architecture and physical architecture, and how the aesthetic standards of the latter could apply to the aesthetic standards of the former.

\section{Architecture}

Beyond its physical manifestation, the term architecture has also been applied to software development. While software does not deal with anything of concrete, immediate existence beyond lines of code (in contrast with hardware), it nonetheless holds similarities with physical architecture insofar as it is about the design, planning and construction of complex structures for human use, and possibly enjoyment. This section takes a closer look at this relationship, and particularly to the place of \emph{patterns}. Patterns, as we will see, can elicit \emph{goodness} in a construction and, through the concrete manifestation of \emph{habitability}, itself of form of beauty.

\subsection{General software architecture}

Software architecture emerged as a consequence of the structured revolution\cite{dijkstra_chapter_1972}, which was concerned more with the higher-level organization of code in order to ensure the quality of the software produced. Such an assurance was suggested by Djikstra in two ways: by ensuring the provability of programs in a rigorously mathematic approach, and by ensuring that programs remained as readable as possible for the programmers. Structure has therefore been an essential component of the intelligibility of the software since the 1970s. It's only in the late 1990s that software architecture as a discipline has been recognized as such, stemming form a bottom-up approach of recognizing that some ways in which code is organized is better than others.

Today, 
\begin{quote}
  software architectural models are intended to describe the structure and behavior of a system in terms of computational entities, their interactions and its composition patterns, so to reason about systems at more abstract  level, disregarding implementation details.\cite{garland_software_2000}
\end{quote}

At its most common denominator, architecture is concerned with the gross structure of a system. At its best, architecture can support the understanding of a system by addressing the same problem as cognitive mapping does: simplifying our ability to grasp large system. Jameson indeed borrows the phrase from Kevin Lynch, whose work on \emph{The Image of the City} highlighted that our understanding of an urban environment relies on combinations of patterns (node, edge, area, limit, landmark) to which personal, imagined identities are ascribed. The process is once again that of abstraction, but goes beyond that. Garland notes that, in its most effective cases, software architecture can expose the high-level constraints on the design of a system, as well as \emph{the rationale for making specific architectural choices}. Again, the intent of the architect (or the programmer) matters along with a purely descriptive depiction of the system. Intent (along with reuse, construction, evolution, analysis and management) is one of the crucial aspects of the computational paradigm on which the software is built.

As an example, the Linux Kernel's architecture can be considered, amongst others reasons, one of the reasons why the project became so popular once integrated into the GNU ecosystem. Along with its distribution license, two of its defining features are speed and portability. While speed can be attributed to its use of C code, also responsible to some extent for its portability, the architecture of the kernel is separated in multiple components which make its extension \emph{relatively} simple. On one side is the monolithic architecture of the kernel, in which process and memory management, virtual file systems, input/output schedulers, device drivers and network interfaces are all lumped together in kernel space. However, this architecture also allows ofr dynamically loadable kernel modules, pieces of the operating system which can be added and removed to the operating system without interference with the core features. This provides a quality of extendability which further contributes to the success of the ecosystem of the Linux ecosystem (see the numerous Linux-based distributions, from Ubuntu to Red Hat and Android).

An architecture, such as that of the Linux kernel, thus provides significant \emph{semantic}  content  about  the  kinds  of properties that developers should be concerned about and the expected paths of evolution of the overall system, as well as its subparts. Other architectures include, for instance, the client-server architecture (with the peer-to-peer architecture as an alternative), the model-view-controller architecture (and its presentation-abstraction-control counterpart), and one can even find their source in chip design, with Friedrich Kittler famously claiming that the last people who ever truly wrote anything where the Intel engineers laying out the plan of the 8086 chip (which would engender the whole family of x86-based devices)\cite{kittler_there_1997}. In this case, this instance is one of the few which relates software architecture to its physical counterpart, albeit in a very technical sense of plans and diagrams.

In the literature consulted for this research, there are only few explicit references to beauty in software architecture design. instead, desirable properties are those of performance, security, availability, functionality, usability, modifiability, portability, reusability, integrability and testability. Perhaps this is due to the fact that the understanding of beauty in terms of external manifestation—decoration—isn't here the main point of the endeavour. Following Adolf Loos, and inspecting software architecture diagram and specifications, it seems that in this specific case, ornament seems to be a crime: highlighting the rise of the useful object in modern civilization\footnote{Along with its white-supremacist undertones rampant in Europe at the time}, he argues that there should be no decoration needed to contribute to the functionality of the object, for fear that it must go out of style.

Style is nonetheless present in software architecture. In this context, an architectural style typically specifies a design vocabulary, constraints on how that vocabulary is used, and semantic assumptions about that vocabulary. For example, a pipe-and-filter style might specify a vocabulary in which the processing components are data transformers (filters), and the interactions are via order-preserving streams (pipes). When it comes down to programming such an architectural style, pipes and filters do have a very real existence in the lines of source code. These concepts are inscribed as the \lstinline{|} character for pipes, or the \lstinline{.filter()} method on the JavaScript array type, which itself has different ways of being written (e.g. with an anonymous callback function, or an externally defined function). By virtue of there being different ways being written, one can always argue for whether or not one is better than the other, ultimately resulting in better, clearer—and perhaps therefore beautiful—source code.

More specifically, the aesthetic manifestations in the form of source code enter in a dialogue with software architecture. If a good system architecture should first and foremost exhbit conceptual integrity\cite{spinellis_beautiful_2009}, one can extend this integrity to its source code manifestation. A message-passing architecture with a series of global variables at the top of each file, or an HTTP server which also subscribes to event channels, would look ugly to most, since they betray their original concept. These concrete manifestations of a \emph{local texture of incoherence}, to paraphrase Beardsley, might be more akin to a \emph{code smell}, a hint that something in the program might be deeply wrong.

Different kinds of architecture all deal with structures, and these structure can only fit together in a satisfactory way if their componnents relate to each other in a satisfactory way. One could also make the parallel between musical orchestration, and computing orchestration as the automated configuration, coordination, deployment and maintenance of (mostly distributed) computer systems and software. If an orchestra has an architecture, it nonetheless also features virtuosos and aesthetically-pleasing phrases (dealing, once again, with tension and its resolution); if a novel has an architecture, it also has specific aesthetic manifestations in its sentences; if computer systems have an architecture, they could possibly have a beautiful manifestation in its individual components which, perhaps not originating from its architecture, nonetheless relate closely to it.

Speaking of individual components, we can recall that before the architect came the craftsman. Architecture as a field and the architect as a role have been solidified during the Renaissance, consecrating a separation of abstract design and concrete work, in which the craftsman is relegated to the role of executioner, until the arrival of civil engineering and blueprints overwhelmingly formalized the discipline. The classical architect, here, serves as the counterpart to the computer scientist, except in an inverse relation: the architect emerged from centuries of hands-on work, while the computer scientist (formerly known as mathematician) was first to a whole field of practicioners as programmers, followed by a need to regulate and structure those practices. Different sequences of events, perhaps, but nonetheless mirroring each other. On one side, construction work without an explicit architect, under the supervision of bishops and clerks, did indeed result in significant results (e.g. Notre Dame de Paris, Basilica of Sienna). On the other side, letting go of structured and restricted modes of working characterizing computer programming up to the 1980s resulted in a comparison described in the aptly-named \textit{The Cathedral and the Bazaar}. This essay described the Linux project, the open-source philosophy it propelled into the limelight, and how the quantity of self-motivated workers without rigid working structures (which is not to say without clear designs) can result in better work than if made by a few, select, highly-skilled individuals\cite{raymond_cathedral_2001}.

What we see, then, is a similar result: individuals can cooperate on a long-term basis out of intrinsic motivation, and without clear, individual ownership of the result; a parallel seen in the similar concepts of \textit{collective craftsmanship} in the Middle-Ages and the \emph{egoless programming} of today\cite{brooks_mythical_1975}, both putting pride in the quality and beauty of their work.

The relation of craftsmanship to architecture is as important as the relationship of craftsmanship to programming. A traditional perspective on the matter is that of the motor skills, with dexterity, care and experience as essential features of a craftsman's ability to realize something beautiful\cite{osborne_aesthetic_1977}, along with self-assigned standards of quality\cite{sennett_craftsman_2009}. These qualitative standards which, when pushed to their extreme, result in a craftsperson's \emph{style}, are to be gained through practice and experience, rather than by explicit measurements\cite{pye_nature_2008} \footnote{See Pye's account of craftsmanship, and his intent to make explicit the question of quality craftsmanship and \emph{"answer factually rather than with a series of emotive noises such as protagonists of craftsmanship have too often made instead of answering it."}}. Two things are concerned here: tools and materials\cite{pye_nature_2008}. A craftsperson should have a deep, implicit knowledge of both, what they use to manipulate (chisels, hammers, ovens, etc.) as well as what they manipulate (stone, wood, steel, etc).

This relationship to tools and materials is expected to have a relationship to \emph{the hand}, and at first seems to exclude the keyboard-based practice of programming. But even within a world in which automated machines have replaced hand-held tools, Osborne writes:

\begin{quote}
    \textit{In modern machine production judgement, experience, ingenuity, dexterity, artistry, skill are all concentrated in the programming before actual production starts.\cite{osborne_aesthetic_1977}}
\end{quote}

He opens here up a solution to the paradox of the hand-made and the computer-automated, as programming emerges from the latter as a new skill. If machines, more and more driven by computing systems, have replaced traditional craftsmanship's skills and dexterity, this replacement can nonetheless suggest programming as a distinctly 21st-century craftsmanship, as well as other forms of cratsmanship-based work in an information economy.
Beautiful code, code well-written, is indeed an integral part of software craftsmanship\cite{oram_beautiful_2007}. More than just function for itself, code among programmers can, and should be held to beauty standards\cite{pineiro_aesthetics_2003}. Such standards are another relationship with traditional craftsmanship—in this case, form is indeed mostly following function.

A craftsman's material consciousness is recognized by the anthropomorphic qualities ascribed by the craftsman to the material. In the case of code, adjectives such as "clean", "elegant", "smelly" occur over and over in online discussions of programmers. Clean code, elegant code, are indicators not just of the awareness of code as a raw material that should be worked on, but also of the necessities for code to exist in a social world. As software craftsmen assemble in loose hierarchies to construct software, the aesthetic standard is \emph{the respect of others}\cite{abelson_structure_1979}.

Another unique feature of software craftsmanship is its blending between tools and material: code, indeed, is both. This is, for instance, represented at its extreme by languages like LISP, in which functions and data are treated in the same way\cite{mccarthy_lisp_1965}. In that sense, code is a material which can be almost seamlessly converted from information to information-\emph{processing}, and vice-versa. Disregarding for now the very real impact of computing on the environment, code as a material is perhaps the only non-finite material that craftspeople can work with—along with words.

Code, then, is not just an overarching, theoretical concept which can only be reckoned with in the abstract, but also the very material foundation from which the reality of software craftsmanship evolves. An analysis of computing phenomena, from software studies to platform studies, should therefore take into account the close relationship to their material that software developers can have. As Fred Brooks put it,

\begin{quote}
    \textit{The programmer, like the poet, works only slightly removed from pure thought-stuff. He builds his castles in the air, from air, creating by exertion of the imagination. Few media of creation are so flexible, so easy to polish and rework, so readily capable of realizing grand conceptual structures.\cite{brooks_mythical_1975}}
\end{quote}

This bottom-up approach to programming, of the craftsman carefully writing and assembling their code into a satsfying artefact, can be examplified in the emergence of \emph{patterns} in software design. Before we examine these more closely, we first turn to a the particular aspect of architectural theory which influenced it, that of Christopher Alexander's.

\subsection{Beauty in architecture}

If it is still unclear whether or not beauty is part of software architecture, beauty is definitely one of the essential components, and aims, of the architect, dating back to Vitruvius's maxim that a building should exhibit \emph{firmitas, utilitas, venustas} (solidity, usefulness, beauty). While structure is meant to, by definition, stand the test of time\footnote{See the still standing structures of Roman and Greek antiquities, based on a particular mixture of cement.}, utility can be assessed by the extent to which a building fulfills its intended function, and beauty remains elusive, more often igniting debates amongst architects, rather than creating consensus over particuarly beautiful constructions\footnote{With the exception, perhaps of Frank Lloyd Wright's Fallingwater, and Ludwig Mies van der Rohe's Neue Nationalgallerie.}. Different schools of beauty in architecture exist, which would could very roughly separate between \emph{top-down} or \emph{bottom-up} approaches.

In terms of top-down approaches, we've seen that, since beautiful software is, first and foremost, software which \emph{runs as intended}, perhaps the first architectural standard which we can apply here is that of Louis Sullivan's \emph{form follows function}. Due to its physical manifestation, Sullivan's statement is therefore inevitably translated into concrete, visible, and sensual consequences.

\begin{quote}
  All things in nature have a shape, that is to say, a form, an outward semblance, that tells us what they are, that distinguishes them from ourselves and from each other.

  [...]

  It is the pervading law of all things organic and inorganic, of all things physical and metaphysical, of all things human and all things superhuman, of all true manifestations of the head, of the heart, of the soul, that the life is recognizable in its expression, that form ever follows function. This is the law.\cite{sullivan_tall_1896}
\end{quote}

We must however keep in mind that Sullivan, within the Chicago School of Architecture, was one of the pioneers of the modern skyscraper, and therefore focused on a very particular kind of architecture, in which productivity became the only, explicit function and the context for his statement. Jacques Rancière, in his study of the Werkbund and the Bauhaus-inspired architecture, offers an alternative approach, away from the strcit functionality laid out by Sullivan and by Loos before him. The simplification of forms and processes, he writes of the AEG Turbinenhalle in Berlin, which is normally associated with the reign of the machine, finds itself, on the contrary, related to art, the only y thing able to spiritualize industrial work and common life\cite{ranciere_aisthesis_2013}.

Rancière offers us an additional perspective, departing form the strict function of an object or of a building, to its \emph{use}. Such a shift moves from a structure-centric perspective (such as Le Corbusier's ideal dimensions), to a human-centric perspective (such as Lacaton \& Vassal's practical extension of space and light). Peter Downton reiterates this point, when he states that "\emph{buildings and design are often judged from artistic perspectives that bear no relation to how the building’s occupants perceive or occupy the building.}"\cite{downton_knowledge_1998}. The bottom-up approach is therefore one which might relate less to software architecture as a top-down, abstraction process, and more to an immediate, usable one of the craftsman as a creator who is highly conscious of the user. The work of Lynch, which we've mentioned above, stands in the tradition of various urban planners and architects such as Jane Jacobs and William H. Whythe. Whythe's work, for instance, focused on empirical observation in order to determine what makes a good space, deducing its aesthetic properties (such as flowing water, movable chairs, the presence of street food vendors, multiple-layered sittings, etc.).

Christopher Alexander also belongs to this empirical tradition of examining what makes a space \emph{good} or not, by exmaining its uses and the feelings it elicits in the people who tread its grounds. First developing his theory when working on the design of the Bay Area Rapid Transit System, he elaborated an approach of architecture which does not exclusively rely on abstract design, but rather takes into account the multiple layers and factors that go into making 

\begin{quote}
  [...] beautiful places, places where you feel yourself, places where you feel alive\cite{alexander_timeless_1979} [...]
\end{quote}

In this work, he focuses on how beauty is involved in moving from \emph{disorganized complexity} to \emph{organized complexity}, an organizing process which is not, in itself, the essence of beauty, but rather the condition for such beauty to arise. Alexander's conception of beauty, while very present throughout his work, is however not immediately concerned with the specifics of aesthetics, understood as the sensual, formal properties of an object, but rather with the existence of such objects.

In this process of achieving organized complexity, he highlights the paradoxical interplay between symmetry and asymmetry, and pinpoints beauty as the "\emph{deep interlock and ambiguity}" of the two, a beauty he also finds the the relationship between static structures of the built environment, and the flow of living individuals in their midst. Architecture as a whole does clearly take into account the role of tension, of which it is yet another manifestation, akin to those we've seen in, amongst others, Ricoeur's analysis of the metaphor, and the resolution of the riddles presented in works of obfuscated source code.

His approach to this quality is successively named as \emph{appropriateness}, \emph{rightness to fit}, \emph{not-simplicity} and \emph{wholeness}. All of these have in common the subsequent need for a purpose, a purpose which he calls the \emph{Quality Without a Name}. This quality, he says, is complicated to name, but nonetheless exists: it is, ultimately, the quality which sustains life, a conclusion which he reached after extensive empirical research: no one can name it precisely, but everyone knows what it refers to. It is the quality which makes one feel at home, which makes one feel like things make sense in a deep, unexplicable way.

Among the adjectives he uses to circle around this quality are \emph{whole, comfortable, free, exact, egoless, eternal}. Since his work applies more broadly to any design-connected discipline, it also applies to software development. Using the word program as an umbrella for \emph{code block}, we can briefly sketch out how such properties could apply to software.

A \emph{whole} program is a program which isn't missing any features, whose encounter (or lack thereof) might cause a crash.

I see two interpretations to a program being \emph{comfortable}. First, it is a program which does not work against its material, a piece of code which is not trying to "re-invent the wheel", when the wheel might already be built-in, or which does indeed re-invent it, if the existing wheels aren't satisfying. Second, a comfortable program is one which might be modified without fear of some unintended side-effects, without inivisible dependencies which might then compromise the whole.

\emph{Free} programs are programs which, anthropomorphizing set aside for an instant, lead their own lives, while being mindful of the lives of others in the shared environment (the design philosophy of the UNIX operating system of "\emph{doing one thing well}", with its consequence of being able to compose these programs into elaborate chains of data processing, can be a good example of such freedom).

\emph{Exact} programs are, then, programs which do not exhibit any verbosity, in which every line is necessary and required, without being so obscure that it hinders comfort.

A program that is too exact might be exhibiting too much of its writer's \emph{ego}, too specific and requiring a unique kind of background knowledge which other readers might not have\footnote{See \emph{egoless programming}, mentionned above.}.

Finally, an \emph{eternal} program relates to the timelessness mentioned in the title of its work—it touches upon the idea of the sublime, a deep, ambivalent feeling of something that stands beyond past, present and future. Programming might be too young of a discipline to be able to single out a precise example, but the Lisp interpreter might be a good candidate, since it is a concise, succint way of writing an interpreter of Lisp in Lisp, embodying the essence of programming language research and some of the main principles of computation (recursion, symbols, interchangeability between data and procedures).

Alexander did conduct empirical research to find examples of such qualities, in a study led at the University of Berkley which resulted in his most popular book, \emph{A Pattern Language}\cite{alexander_pattern_1977}. In it, he lists out 253 patterns which, he claims, form a language, akin to a Chomskian generative grammar, re-usable and extendable in a very concrete way. This study has had a significant impact on the computer science community, to which we turn to next.

\vspace*{4\baselineskip}
\centerline{\rule{0.3334\linewidth}{.4pt}}

\pagebreak

\subsection{Applying architectural beauty in software}

\begin{itemize}
  \item Focus on Gabriel's work, his connection to poetry, and also looking at how it applies well (softdev), and not so well (code poetry, since the functionality of code poems isn't quite so obvious as the functional use of a building, or of a program)
  \item Software patters as a design cannon, including the ambivalence of \emph{style}
  \item Design patterns: one subset of these is the aesthetic, and the one I focus on. Not all patterns are beautiful, but according to Alexander, they help make something "good" (intrinsically linked to the beautiful, in the sense that aesthetics might be a condition for making something good).
\end{itemize}

Transition with the case of the case of hacking: code that is not meant to be read.

\section{Hacking}

What beauty is there in machine-enforced inscrutability? Particularly, this redirects to the understanding of the machine (e.g. trying to reduce character counts for one-liners).

\subsection{History of hacking}

Starting with a disambiguation of the term, with a focus on the specific subversion of the machine. Connection to obfuscated code, but with the distinction made that this obfuscation isn't on purpose, but rather due to machine constraints.

\subsection{Unreadability and aesthetics}

\begin{itemize}
  \item Example of the one-liners (trying to compress as much as possible, aspect of social competition of who can write it in the densest way)
  \item Example of the demoscene (trying to extract as much as possible from the machine, along with dual presentation of the work: output, and code walkthrough)
\end{itemize}

This shows in an extreme way that you need a degree of expert knowledge in order to appreciate it—in this sense, aesthetics are also dependent on pre-existing knowledge.

\subsection{Hacking and/or the lack of beauty}

Starting from the limited number examples I have found for beauty in hacking, I inquire as to why is there no beauty (or not as much) in hacking? Does there need to be beauty anywhere, or does function not need it? Does it just never need to be read, or returned to? Is a hack, by definition, idiosyncratic, and thus only has its own aesthetic standard?

There doesn't seem to be so many instances of individuals saying "look, this part is beautiful" because it's so focused on what the machine understands; the real beauty is how they make the humans \emph{realize} what the machine \emph{really} understands.

\subsection{Computation}

What is a computer? What is computation? There are differring approaches, and this allows me to highlight the one in this thesis (that of the formal, symbolic system). Cantell-Smith is quite relevant here, since he explain that things are complicated to explain, but nonetheless intuitive to any serious practitioner of CS.

\subsection{Programming languages}

\begin{itemize}
  \item What makes a PL (from historical and theoretical perspectives)
  \item Touching on programming paradigms as moods.
  \item What makes a \emph{good} PL, with a sub-section on idiomaticity and how it can resolve the tension between the specific and the universal.
  \item Esoteric languages
\end{itemize}

\subsection{Formal systems in programming and art}

\begin{itemize}
  \item Define a formal system.
  \item Show how it matches Goodman's definition in \emph{Languages of Art} (density, disjointedness, boundedness).
\end{itemize}

\section{Programming Semantics}

This sections aims at answering the question: under what form do semantics exist, and to what extent can we say that computers understand?

\subsection{Semantics in PL}

\begin{itemize}
  \item Approach it first in a very practical way (parse trees, tokens, environments), and then in a more theoretical way (what are the meanings that we are trying to communicate?)
  \item Programming semantics do exist according to the textbooks, but in the end they're just another \emph{décalage}, describing semantics as a meta-syntax layer.
  \item Consequences: computers only have operational semantics, while humans require denotational semantics.
\end{itemize}

\subsection{Concepts of PL}

\begin{itemize}
  \item The issue actually comes from the fact that some concepts might be very foreign and hard to communicate (which is why programming can be hard, and how the whole section has shown relative limitations of doing so). Some of these concepts include \emph{thread concurrency}, \emph{streaming data}, \emph{pointer arithmetic}
\end{itemize}

And so this is why we need aesthetics: to communicate both easy and complicated things, to reduce friction as much as possible. Example: Alan Perlis's Epigrams are poem-like sentences.

\section{Conclusion}

\begin{itemize}
  \item Aesthetics matter, even in such a highly formal, syntactical, autotelic system as a computer, since they connect a surface-structure with a deep-structure, and act as a bridge between subjectivities.
  \item There is a complex relationship between abstraction, metaphor and mental model when it comes to understanding programs. Some patterns can help with this task of understanding, and point at a different direction of aesthetics (as manifestation of "the good").
  \item Concepts to build on going forward: \emph{semantic compression} and \emph{spatio-visual pattern-solving}
\end{itemize}

\begin{quote}
    The emphasis is placed on the symbolic, cognitive, planning aspects of the arts leads us to give value to the role played by problem-solving, seeing there a model in terms of which the moment-to-moment artist’s behavior at work can be described. \emph{An analysis of behavior as a sequence of problem-solving and planning activities seems to be most promising [...]"} (Goodman, Project Zero)
\end{quote}

\subsection{Next steps}

\begin{itemize}
  \item More close-reading of more source code, including parts of large-scale, production software (UNIX? LaTeX?)
  \item Gathering of more hacking program texts and computer science/research program texts.
  \item More rigorously tie in the analysis of source code poems more tightly with aesthetics (denotation, depictions, labelling, etc.)
  \item Actually answer the question of whether or not computers "care" about aesthetics.
\end{itemize}

\bibliographystyle{unsrt}
\bibliography{thesis.bib}

\end{document}