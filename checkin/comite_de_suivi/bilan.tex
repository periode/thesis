\documentclass{article}

\usepackage[hyphens]{url}
\usepackage{fontspec}
\usepackage{graphicx}
\usepackage{listings}
\usepackage{authblk}
\usepackage[bottom]{footmisc}

\graphicspath{ {./images/} }

\defaultfontfeatures{Mapping=tex-text,Scale=1.00}
\setmainfont{Inter Light}
\setmonofont{Liberation Mono}
\linespread{1.50}
\sloppy

\lstset{
    basicstyle=\footnotesize\ttfamily,
    breaklines=true,
    frame=single
}

\begin{document}
\title{Le rôle de l'esthétique dans les compréhensions du code source}
\author{Pierre Depaz\\sous la direction d'Alexandre Gefen (Paris-3)\\et Nick Montfort (MIT)}
\affil{ED120 - THALIM}
\date{Juin 2021}
\maketitle

\section{Introduction}

\subsection{problématique}

\subsection{etat de l'art}

\vspace*{4\baselineskip}
\centerline{\rule{0.3334\linewidth}{.4pt}}

\section{Approche}

à la base, discussion des codes sources par diverses personnes (programmeurs, artistes, académiques)

esthétique comme manifestation physique qui peut être saisie de manière sensuelle \cite{genette_fiction_1993} et de manière formelle \cite{goodman_languages_1976}.

relation subsidiaire avec la fonctionalité et donc connection avec la compréhension

définition de la compréhension (ce que le code veut faire, et ce que le code doit faire)

\subsection{Sources et corpus}

- 4 groupes de pratiquants
- corpus en ligne

\subsection{Méthodologie}

avec éventuels changements initiels

place de la métaphore: quelles métaphores? language, architecture, matériau

\vspace*{4\baselineskip}
\centerline{\rule{0.3334\linewidth}{.4pt}}

\section{Avancement}

\subsection{Corpus analysés}

typologie 

- ingénieurs

- artistes

- hackers


\subsection{Analyses effectuées et résultats}

dépendance sur ce qu'on veut représenter (une action immédiate, une vision du monde, une idée théorique, une capacité)

différentes métaphores qui se rapprochent de l'élégance et du code comme matériau

\vspace*{4\baselineskip}
\centerline{\rule{0.3334\linewidth}{.4pt}}

\section{Prochaines étapes}

\subsection{Restant}

- académiques

\subsection{Calendrier}

\vspace*{4\baselineskip}
\centerline{\rule{0.3334\linewidth}{.4pt}}

\pagebreak

\bibliographystyle{unsrt}
\bibliography{../thesis.bib}

\end{document}