\documentclass{article}

\usepackage[hyphens]{url}
\usepackage{fontspec}
\usepackage{graphicx}
\usepackage{listings}
\usepackage{authblk}
\usepackage[bottom]{footmisc}

\graphicspath{ {./images/} }

\defaultfontfeatures{Mapping=tex-text,Scale=1.00}
\setmainfont{Inter Light}
\setmonofont{Liberation Mono}
\linespread{1.50}
\sloppy

\lstset{
    basicstyle=\footnotesize\ttfamily,
    breaklines=true,
    frame=single
}

\begin{document}
\title{Le rôle de l'esthétique dans les compréhensions du code source}
\author{Pierre Depaz\\sous la direction d'Alexandre Gefen (Paris-3)\\et Nick Montfort (MIT)}
\affil{ED120 - THALIM}
\date{Juin 2021}
\maketitle

\section{Introduction}

Les discusssions autour du l'impact de la computation sur l'esthétique, la créativité et la littérature se sont largement développées lors des dernières décennies, avec l'avènement des machines de Turing, puis de leur miniaturization et de leur personalization (i.e. l'avènement des \emph{personal computers}). En effet, l'implication des algorithmes fait apparaître de nouvelles sortes de textes, au sein du champ de la littérature numérique. Ce projet de recherche se penche sur un autre des ces "textes", ceux constitués par le code source, condition nécéssaire à l'existence de ces derniers.

En tant que texte dont le but même est de disparaître (transformé en changements de courant électrique), le code source est un object hybride, description et action, communication entre humain et humain, entre humain et machine, ou entre machine et machine. Cependant, en tant que création humaine, il est possible de s'interroger sur les modalités de ses manifestations, notamment en termes d'expressivité. Si l'expressivité fonctionnelle d'un code source implémentant un algorithme est indéniable, l'expressivité artistique d'un texte de code source demeure élusive. Quelle est donc la place de l'esthétique, d'une manifestation formelle sensuellement plaisante, dans l'écriture ou la lecture du code source? Le code source, en tant que créations basées sur un système syntaxique similaire au language naturel, se révèle être un système de communication aussi bien d'humain à humain que d'humain à machine, si ce n'est principalement pour les humains, et secondairement pour les machines\cite{abelson_structure_1979}. Une fois que la condition principale d'existence du code source (sa validité d'exécution), la seconde condition semble être non pas sa beauté mais sa compréhensibilité.

Se pose donc la question d'un texte dont les manifestations formelles sont vouées à disparaitre, et dont la lecture n'est qu'un processus collatéral de son exécution. Donald Knuth, dans son oeuvre d'informatique \emph{The Art of Computer Programming}\cite{knuth_art_1997}, commence par établir que la programmation (l'écriture et la lecture du code source) est bel et bien un art. Cette déclaration, dans les premières du premier volume, n'est pourtant pas réitérée ni élaborée dans les autres volumes du monographe. Si écrire du code peut être un art, certains codes sources peuvent donc exhiber des propriétés esthétiques, mais ces dernières sont rarement explicitées par la littérature sur le sujet.

La problématique principale de ce projet de recherche est donc celle du \emph{rôle de l'esthétique dans les compréhensions du code sources}. Il s'agit de mettre à jour et d'interroger les standards esthétiques appliqués au code, et d'identifier si ces standards sont similaires à d'autres activités créatives humaines, notamment celle de la littérature.

Une approache rapide de l'état de l'art sur ce sujet révèle deux tendances séparées: d'une part, la littérature en informatique et ingénierie prend en compte l'évidence de l'existence d'une esthétique du code source, du point de vue la productivité de ceux et celles qui écrivent du code, et d'un point de vue cognitif de la compréhension des bases de code\cite{oram_beautiful_2007,cox_programming_2009,gabriel_mob_2001,martin_clean_2008,detienne_software_2012,weinberg_psychology_1998}. De l'autre, les recherches en sciences humaines traitent de l'interaction entre esthétique et code, tout en restant sur une conception abstraite et désincarnée du "code", élaborant une esthétique du digital sans pour autant rentrer dans les détails des codes sources eux-mêmes (qu'ils soient écrits en Perl, Python, Ruby, etc.)\cite{cramer_executupable_2019,hayles_my_2010,mackenzie_cutting_2006,levy_programmation_1992}. Il existe cependant un certain nombres d'ouvrages de sciences humaines approchant la question matérielle du code de manière directe, que ce soit au niveau culturel\cite{montfort_10_2014}, politique\cite{cox_speaking_2013} ou sociologique\cite{paloque-berges_poetique_2009}. C'est donc au sein de cette recherche sur les manifestations du code source que cette recherche s'inscrit.

\vspace*{2\baselineskip}
\centerline{\rule{0.3334\linewidth}{.4pt}}

\section{Méthodologie}

L'approche principale de ce projet est donc empirique. Il s'agit avant tout d'examiner les codes sources eux-mêmes, ainsi que les discours de ceux et celles qui les écrivent et les lisent.

Cette examination se fait également par le biais d'un cadre théorique partant de la philosophie esthétique et la littérature afin de définir mon utilisation du terme \emph{esthétique}. D'un point de vue littéraire, je m'appuie sur les travaux de Gérard Genette et sa distinction entre fiction et diction\cite{genette_fiction_1993}, considérant l'esthétique du code comme sa diction, tandis que la poétique serait sa diction. À cheval entre littérature et philosophie se trouvent les oeuvres de Paul Ricoeur\cite{ricoeur_rule_2003} et de George Lakoff\cite{lakoff_metaphors_1980}, reliant composition verbale et évocation d'images mentales. Enfin, cette manifestation en surface est reprise par Nelson Goodman dans son analyse des languages de l'art\cite{goodman_languages_1976}, et notamment dans son exploration entre systèmes syntactiques et communcation, ainsi que son approache scientifique des phénomènes artistiques.

C'est donc une approche analytique et cognitive du phénomène artistique que je prends, et cela pose donc la question de la compréhension. Pour celà, je m'appuie sur les travaux de Kintsch et van Dijk et leurs études des stratégies de compréhension du discours\cite{kintsch_toward_1978}, l'approche métaphorique étant considérée comme une stratégie de compréhension. Le code source étant un texte, il produit donc un discours.

\begin{itemize}
    \item définition de la compréhension (ce que le code veut faire, et ce que le code doit faire)
\end{itemize}

\subsection{Sources et corpus}

\begin{itemize}
    \item 4 groupes de pratiquants: à la base, discussion des codes sources par diverses personnes (programmeurs, artistes, académiques)
    \item corpus en ligne
\end{itemize}

\subsection{Méthodologie}

avec éventuels changements initiels

place de la métaphore: quelles métaphores? language, architecture, matériau

\vspace*{2\baselineskip}
\centerline{\rule{0.3334\linewidth}{.4pt}}

\section{Avancement}

\subsection{Corpus analysés}

typologie 

- ingénieurs

- artistes

- hackers


\subsection{Analyses effectuées et résultats}

dépendance sur ce qu'on veut représenter (une action immédiate, une vision du monde, une idée théorique, une capacité)

différentes métaphores qui se rapprochent de l'élégance et du code comme matériau

\vspace*{4\baselineskip}
\centerline{\rule{0.3334\linewidth}{.4pt}}

\section{Prochaines étapes}

\subsection{Restant}

- académiques

\subsection{Calendrier}

\vspace*{2\baselineskip}
\centerline{\rule{0.3334\linewidth}{.4pt}}

\pagebreak

\bibliographystyle{unsrt}
\bibliography{../thesis.bib}

\end{document}