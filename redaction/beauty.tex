\chapter{Beauty and understanding}
\label{chap:beauty}

% 80k chars

This chapter focuses on what beauty has to do with understanding, first from a theoretical perspective, and then diving specifically into how specific domains approach this relation. Our theoretical approach will be based on the aesthetic theory of Nelson Goodman, and a lineage which links aesthetics to cognition, most recently aided by the contribution of neurosciences.

After argumenting for a conception of aesthetics which tends to intellectual, rather than emotional, engagement, we will pay attention to how surface structure and conceptual assemblages relate. That is, we will highlight how each of the domains contigent to source code— literature, mathematics and architecture—communicate certain concepts through their respective and specific means of symbolic representation.

\section{Aesthetics and cognition}
\label{sec:aesthetic-cognition}

%20k characters, doing an overview of how aesthetic philosophy relates to cognition

The way that things are presented formally—which we've defined as their aesthetics—have been empirically shown to affect the comprehension of content. Without engaging too directly in the media-determination thesis, which states that what one can say is determined by the medium through which they say it, be it language or technical media, we nonetheless start from the point that form influences content \citep{postman_amusing_1985}.

Jack Goody and Walter Ong have shown in their anthropological studies that the primary means of communication of the surveyed communities does affect the engagement of said communities with concepts such as ownership, history and governance \citep{ong_orality_2012,goody_logic_1986}. More recently, Edward Tufte and his work on data visualization have furthered this line of research by focusing on the translation of similar data from textual medium to graphic medium \citep{tufte_visual_2001}. A case has indeed been made for the impact of appearance towards structure, both in source code and elsewhere. To complement this comparative approach between several mediums, we now look at how source code performs expressively as a specific system, starting from Nelson Goodman's theorization of such a system..

\subsection{Source code as a language of art}
\label{subsec:source-code-language-art}

From the question of the nature of the aesthetic experience from the perspective of the audience, whether as an aesthetic emotion being felt or as an aesthetic judgment being given, we shift our attention to the object of aesthetic experience, and to the questions of \emph{how does a work represent?} and \emph{what does a work represent?}. To answer these, we rely on answers provided by by Nelson Goodman in the \emph{Languages of Art: An Approach to a Theory of Symbols} \citep{goodman_languages_1976} and \emph{The Structure of Appearance} \citep{goodman_structure_1966}.

The starting point for Goodman's analysis is that production and understanding in the arts involve human activities that, though they differ in specific ways among themselves and from other activities, are nevertheless generically related to perception, scientific inquiry, and other cognitive activity, and such an activity specifically involves symbolic systems. It is those two components that Goodman aims at expliciting: what constitutes an aesthetic symbol system, and how does it express?

Goodman develops a systematic approach to symbols in art, freed from any media-specificity (e.g. from pictorial symbols to musical notations and even time marks on clocks and watches). A symbolic system, in his definition, consists of characters, along with rules to govern their combination with other characters, itself correlated with a field of reference. These symbols and their arrangement within a work of art supports an aesthetic experience\footnote{It should be noted here that Goodman does not limit the aesthetic experience to a positive, pleasurable one. An artistic symbolic system can be seen even if the result is considered bad.}. A work, as a part, particular arrangement of a symbol system according to specific syntactic and semantic rules, can therefore enable an aesthetic experience.

A symbol system is based on five requirements: a system should be composed of signs which are unambiguous, syntactically and semantically disjointed, and differentiated \citep{goodman_languages_1976}. This classification makes it possible to compare the various symbolization systems used in art, science, and life in general: from clocks to counters, from diagrams to maps models, from musical scores to painters' sketches and scripts (intended in a broad sense as the characters of natural languages). In our case, this provides us for a framework to investigate the extent to which source code qualifies as a language of art.

Source code is written in a formal linguistic system called a programming language. Such a linguistic system is, obviously, digital in nature, and therefore satisfies at least the syntactic requirements of disjointedness, differentiation (a mark only ever corresponds to that symbol, such as a variable or function name), as well as syntactic repleteness (relatively fewer factors need to be taken into account during the interpretative process)\footnote{This does not mean that any program text written in this symbolic system will tend to be syntactically replete. On the contrary, the tendency of program text to veer towards verbosity implies the desirable state of repleteness.}. This does qualify source code as a potential language for art, through which aesthetic expressiveness can emerge.

As Goodman notes, the distinct signs that compose a symbols system do not have intrinsic properties, but a thing serves as a sign only in relation to a symbol system, and a field of reference. Some requirements need to be fulfilled for such a symbol to be what is called a symptom of the aesthetic. Amongst exemplification, syntactic density, semantic density and syntactic repleteness, source code fulfills the last two criteria: with a limited set of symbols (at one of the lowest levels, only two symbols, traditionally marked as 0 and 1), programs can refer to and enact complex states and behaviours.

The field of reference is understood her as being the set of concepts which are being referred to by a symbolic system. For instance, a symbolic system such as western classical music can refer to concepts such as lament, piety, heroism or grace, while a chine \emph{shanshui} painting has a landscape composed of mountains and rivers as its field of reference. The combination of both the problem domain, as evoked in \ref{chap:understanding}, and of the technological environment on which the source code is to be executed, is posited here as an equivalent to the Goodman's field of reference.

Now that we have highlighted what a symbolic system is, we turn to how such a system can signify and reference a particular field of reference.

Goodman highlights the ways in which symbols systems communicate, through the notion of \emph{reference}. To refer to, in this sense, is the action by which a symbol stands in for an item. Reference, he sketches out, takes place through the different dyads of denotation and exemplification, description and representation, possession and expression \citep{goodman_languages_1976}. We will see how these various means of referring can be instantiated in the symbolic system of source code.

First, then, denotation; it is the core of representation, a reference from a symbol to one or many objects it applies to and is independent of resemblance. Rather, it uses a particular relationship via the use of labels; that is, a symbol stands in for an item in the field of reference. For instance, a name denotes its bearer and a predicate each object in its extension. Names such as variable names or function names thus denote a particular item in the field of reference, and act as their label. For instance, \lstinline{var auth_level} denotes an ability to access and modify resources.

The labelling process therefore serves as the symbolic expression for a particular field. In source, this can happen through variable naming (as seen above), but also through type definition\footnote{For instance, a particular choice of a numeric value, such as \lstinline{int} or \lstinline{float} denote a particular level of preciseness}, as well as additional affordances which we look at in \ref{sec:programming-aesthetic-framework}.

Source code also make extensive use of description. If we consider a program text as a series of steps, a series of states, or a series of instructions, then it follows that source code is leaning heavily on the side of description, when it comes to its power of reference. Indeed, a program text is a description of how to solve a problem from the computer's perspective, written extensively in machine language\footnote{Pseudo-code is therefore a representation of a potential source code written in a specific language.}. All source code can therefore be said to be a description of a combination of action and states.

States are also a particular case in source code: they are both a description and, because they are not the thing itself, they are also a representation. As one can see in \ref{code:representation}, an individual can be represented within source code with a particular construct (here called a \lstinline{struct}).

\begin{listing}
    \inputminted{rust}{./corpus/representation.rs}
    \caption{An example of how source code can be a representation an individual, and can exemplify encapsulation, written in Rust.}
    \label{code:representation}
\end{listing}

This representation, in the specific instance of object-oriented programming in \ref{code:representation}, also manifests Goodman's aesthetic symptom of possession. Here, the source code posseses similar properties as the thing referenced (since our prototypal image of a person has an age, a name and interests). Through this possession of a property, it acts as an example of a prototypal person.

Exemplification is another aspect of Goodman's theory, which has nonetheless remained somewhat limited \citep{elgin_making_2011}. A symbol exemplifying, also called an examplar, is considered as a stand-in for an item in the field of reference. Specifically for source, code, this is a case that we have seen in \ref{subsec:scientists}, where a particular source code is written in order to act as an example of a broader concept. For instance, a program text can, at a lower level, exemplify a particular kind of procedure, such as encapsulation or nestedness. The program text therefore exemplifies the constitutive element of the linked list\footnote{A linked list is a basic data structure in computer science, which consists in a succession of connected objects.}. However, a similar program text can also be an example of cleanliness, of clarity, or elegance (see \ref{sec:ideals-beauty}): a program text written by a software developer can be seen as possessing the property of cleanliness, by virtue of its implementation of syntactic and semantic rules, while another program text written by a hacker can be seen as highlighting detailed hardware knowledge.

Additionally, the features a symbol exemplifies depends on its function (or, more precisely, its functional context) \citep{elgin_understanding_1993}. A symbol can perform a variety of function: a piece of code in a textbook might exemplify an algorithm, while the same piece of code in production software might be seen as a liability, or as a boring section in a code poem.

Source code maintains specific features on its relation to the field of reference. On the one hand, a particular class of characters employed as symbols (also called \emph{tokens} in the context of programming languages), do not maintain a clear relationship with the items in the field of reference. That is, in program texts, two distinct symbols can be referring to the same concept, value, or place in memory (see \ref{chap:programming} for a further explanation of these differences and \ref{code:multiple_references} for an example), something Goodman nonetheless assigns as another symptom of the aesthetic: multiple and complex references.

\begin{listing}
    \inputminted{rust}{./corpus/multiple_references.rs}
    \caption{The system of value, references and pointers make source code into a highly complex symbolic system.}
    \label{code:multiple_references}
\end{listing}

On the other hand, the representation of a field of reference is done through a disjointed and differentiated system: the boundaries of each items in the field of reference are clearly defined, in virtue of the specific symbol system that programming languages are. These programming languages do dictate the rules of engagement of the symbolic system with the field of reference.

We have shown here that source code qualifies as a symbolic system susceptible of affording symptoms of the aesthetic. We have also highlighted its specificities, particularly in terms of descriptions and representations, and of complex and multiple references. Source code being a dual language, between human and machine, makes it have such complex and multiple references. A final aspect to investigate is the expressiveness of source code, with a particular attention to how source code can manifest of metaphorical exemplification and representation.

The particular expressive power of an aesthetic experience surfaces when the examplification involves a foreign element, an event that Goodman refers to as metaphorical exemplification. While this approach has been broadened by Lakoff et. al., and mentioned in \ref{subsec:metaphor-computation}, philosophers of art have pinpointed the metaphorical event as a reliable symptom of the aesthetic.

Max Black initiates a view of metaphors which go beyond a simple comparison; dubbed the \emph{interaction view}, he considers the metaphorical device as containing positive cognitive content. Simply paraphrasing a metaphor, even if one captures precisely the same connotations/associations as the metaphor, does not convey the same meaning as the metaphor itself\footnote{For instance, saying \emph{Je chavire dans l'écume des phénomènes} does not have the similar expressive power as listing all the properties of \emph{phénomènes}. The original sentence is from Beckett.}.

Through his contribution to aesthetic philosophy, Monroe Beardsley's started touching upon metaphor from a semantic perspective. Published alongside his inquiries into the aesthetic character of an experience, and taken later on by Ricoeur as a basis for his study, \emph{The Metaphorical Twist} implies that semantics and aesthetics might be connected through the structuring operation of the metaphor—that which elicits an aesthetic experience can do so through the creation of unexpected, or previously unattainable meaning. Ricoeur's theory of the metaphor indeed builds on Beardsley's conception that metaphor can have a designative role (the primary subject) which adds a \emph{"local texture of irrelevance"}, a \emph{"foreign component"}, whose semantic richness might over-reach and obfuscate the intended meaning, as well as a connotative one (the secondary subject), in which meaning is peripheral. The cognitive stimulation and enlightment takes place through a metaphor-induced tension, between central and periphery, between illuminating and obfuscating, between evidence and irrelevance.

As Beardsley inquiries into the features necessary for an aesthetic experience, of which the metaphor is part, he lists five criteria to distinguish the character of such an experience. Besides object-directedness, felt-freedom, detached-affect and wholeness, is the criteria of \emph{active discovery}, which is

\begin{quote}
    "a sense of actively exercising the constructive powers of the mind, of being challenged by a variety of potentially conflicting stimuli to try and make them cohere; exhilaration in seeing connections between percepts and meanings; a sense of intelligibility"\footnote{The Aesthetic Experience, in The Aesthetic Point of View \citep{beardsley_aesthetic_1970}.}
\end{quote}

As such, Beardsley highlights the possibility of an aesthetic experience to make understandable, to unlock new knowledge in the beholder, and he considers metaphors as a way to do so. The stages he lists go from (1) the word exhibiting properties, to (2) those properties being made into meaning, and finally into (3) a staple of the object, consolidating into (or dying from becoming) a commonplace. This interplay of a metaphor being integrated into our everyday mental structures, of poetry bringing forth into the thinkable, and in metaphor creating a tension for such bringing-forth to happen, makes the case for at least one of the consequences of an aesthetic experience, and therefore one of its functions: making sense of the complex concepts of world.

Finally, Catherine Elgin has pursued the work of Goodman by furthering the inquiry into arts as a branch of aesthetics. Drawing on the work mentioned above, she investigates the relationship between art and understanding, stating that aesthetics then is the branch of epistemology that explains how interpretively indeterminate symbols advance understanding \citep{elgin_understanding_2020}, and that it does so in the context of interpretive indeterminacy. As syntactically and semantically dense symbol systems are used in artworks,  it is this multiplicity in interpretations which requires sustained cognitive attention with the artwork. To explain these multiple interpretations, the metaphor is again presented the key device in explaining the epistemic potency of aesthetics, based on an interpretative feedback loop from the viewer. And yet, in the context of source code, this interpretation is always shadowed by its machine counterpart of how the computer interprets the program.

\subsection{Contemporary approaches to art and cognition}
\label{subsec:art-cognition-contemporary}

We have this far drawn from existing work in philosophy of art, in order to map out the expressive power of a given formal representation, as a traditional pre-requisite to the gaining of art status of an object, and highlighted the crucial role of metaphors in engaged cognition in an aesthetic experience. Contemporary literature, and the emergence of neuroscientific studies of such aesthetic experience seem to confirm empirically this approach, and highlight as well two related additional components: sequential experience and skill levels.

The aesthetic experience—that is, the positively received perception of a natural or crafted object—has traditionally been laid out across multiple axes, with more or less overlap. Whether this positive perception is due to an emotional response, to a harmonious assessment, to an axiomatic adherence or to disinterested pleasure has indeed been the topic of debates amongst philosophers for centuries \citep{peacocke_aesthetic_2023}.

Noël Carroll sums up these different directions under the broad areas of affect, axiom and content \citep{carroll_aesthetic_2002}. He underlines how an aesthetic experience dictated by affect removes the object from one's assessment of purpose, value and effect, and limiting it to form, following Kant's principle of disinterested pleasure via passive contemplation. As such, a flower, a sunset or a musical melody can evoke affective aesthetic experiences. Yet, the supposed tendency of this kind of experience to release us from worldly concerns fails, for Carroll, to encompass aesthetic experiences that are rooted in so-called worldly concerns—such as a documentary photography, skillful physical performance, or delicatedly crafted glassware—and is therefore unsatisfying as a root explanation for the aesthetic experience.

An axiomatic aesthetic experience is, in turn, based on the sort of value that the object is being associated with—such as depiction of religious topics or a manifestation of a particular style. While Carroll does acknowledge a certain virtue of this aesthetic experience in terms of contribution to group cohesion through shared values and imaginaries, its limitations are found in a pre-existing answer to the value judgment that is being bestowed upon the object—the material and sensual properties of the object at hand are irrelevant since their quality is already decided \emph{a priori}.

It is in the content approach that Carroll finds the most satisfying answer to what the aesthetic experience is. Content, here, is defined as the forms being apprehended, along with its combinations, juxtapositions and comparisons with other forms. When we engage with the sensual aspects or an object, our attention is indeed directed first and foremost at what the object looks like. More specifically, Carroll notes, if attention is directed with understanding to the form of the art work or to its expressive and aesthetic properties or to the interaction between those features, then the experience is said to be aesthetic \citep{carroll_aesthetic_2002}.

Form, and the attention to form, will thus be taken as our starting point.  This content approach to form, i.e. the set of appearing choices intended to realize the purpose of the artwork, involves questions of function, implied by the presence of purpose pertaining to an artwork. Particulary, how does the object of aesthetic experience manifest this purpose, in such a way that it can be correctly judged, insofar as its perceived form and perceived purpose are aligned, distinct from any emotional or axiomatic charge?

This analysis is complemented by the study conducted by Anjan Chatterjee and Oshin Vartanian on the evaluation of the aesthetic experience from a neuroscientific point of view. Like Carroll, they highlight three different perspectives: a sensory-motor perspective, loosely mapped to an affective experience, an emotion-valuation perspective, similar to an axiological experience, and a meaning-knowledge experience, which we equate to the content approach to the aesthetic experience \citep{chatterjee_neuroscience_2016}.

Additionally, they make the distinction between an aesthetic judgment, which emanates from the process of understanding the work, and an aesthetic emotion, which follows from the ease of acquisition of such an understanding. Without being mutually exclusive, these two pendants are related to the amount of engagement provided by the person who aesthetically experiences the object. One can have an aesthetic emotion without being able to provide an aesthetic judgment, a case in which one does not hold enough expertise to apprehend or appreciate a particular realisation. In this sense, the aesthetic judgment, unlike the aesthetic emotion, requires something additional. This conditioning of the aesthetic experience to a certain kind of pre-existing knowledge or skill is supported by the authors' mention of the theory of fluency-based aesthetics \citep{chatterjee_neuroscience_2016}, and their view builds on models that frame aesthetic experiences as the products of sequential and distinct information-processing stages, each of which isolates and analyzes a specific component of a stimulus (e.g., artwork).

These stages, based on Leder et. al's model, are based on empirical observation in scientific studies which segment an aesthetic experience in sequential steps \citep{leder_model_2004}. These evolve form perception, to implicit classification, explicit classification, cognitive mastering and evaluation—that is, fully-qualified aesthetic judgment. This conception is concomittant to Rebert et. al.'s proposal for an aesthetic framework based on processing fluency, which they define as a function of the perceiver's processing dynamics: the more fluently the perceiver can process an object, the more positive is her aesthetic response \citep{reber_processing_2004}. While they focus their study on perceptual fluency, tending to traditional aesthetic features such as symmetry, contrast and balance; they also consider conceptual fluency as an influence on the aesthetic experience, through the attention given to the meaning of a stimulus and the relation to semantic knowledge structures. Such a conceptualizing thus hints at a similar skill-based, contextual framework which we have seen for the aesthetic judgment of source code, and yet an additional establishment of a relation between truth and beauty\footnote{"these findings suggest that judgments of beauty and intuitive judgments of truth may share a common underlying mechanism. Although human reason conceptually separates beauty and truth, the very same experience of processing fluency may serve as a nonanalytic basis for both judgments." \citep{reber_processing_2004}}.

This approach of cognitive ease, which we've already identified in \ref{chap:ideals}, is finally echoed in the view that Gregory Chaitin, a computer scientist and mathematician, offers of comprehension as compression. By considering that the understanding of a topic is correlated with the lower cognitive burden experienced when reasoning about such topic, Chaitin forms a view in which an individual understands better through a properly tuned model—a model that can explain more with less \citep{zenil_compression_2021}.

\spacer

These studies thus show a particular empirical attention to the cognitive engagement with respect to the apprehension an object from an aesthetic perspective, as opposed to passive contemplation or value-driven aggreement. While these other types of experiences remain valid when apprehending such an object, we do focus here on this specific kind of experience: the cognitive approach to the aesthetic experience. Goodman describes such an experience as involving:

\begin{quote}
    making delicated discriminations and discerning subtle relationships, identifying symbol systems and what these characters denote and exemplify, interpreting works and reorganizing the world in terms of works of art and works in termins of the world. \citep{goodman_languages_1976}
\end{quote}

\spacer

In this section we've glanced at an overview of research on how cognitive engagement is involved in an aesthetic experience, both from the point of view of the philosophy of art and psychology. However, highlighting this involvment does not immediately explicit the nature and details of such cognitive engagement. Speaking in terms of form and object are higher-level concepts which tend to erase the specificities of the various systems of aesthetic properties, and how their arrangement expresses various concepts. Now that we have sketched out an understanding of source code as a symbolic system supporting an aesthetic experience, we must provide a more detailed account of the specificities of source code. To do so, we turn to a comparative approach, looking at the set of aesthetic domains located contingently to source code through programmer discourse, and we analyse how each  of these domains involve cognition in their formal presentations.


\section{Literature and understanding}
\label{sec:aesthetic-literature}

Literature as a cognitive device relies, as we've seen in \ref{sec:ideals-beauty}, on the use of metaphors to provide a new perspective on a familiar concept, and hence complement the understanding that one has of it. While Lakoff and Johnson's approach to the conceptual metaphor will serve a basis to explore metaphors in the broad sense across software and narrative, I also argue that Ricoeur's focus on the tension of the \emph{statement} rather than primarily on the \emph{word} will help us better understand some of the aesthetic manifestations of software metaphors, without being limited to tokens. Following a brief overview of his contribution, I examine the various uses of metaphor in software and in literature, touch upon the cognitive turn in literary studies, and conclude the section by the ambiguity of a cognitive account of programming.

% 7k with a subsection on metaphor
\subsection{Literary metaphors}
\label{subsec:literary-metaphors}

Writing in \emph{The Rule of Metaphor}, Ricoeur operates two shifts which will help us better assess not just the inherent complexity of program texts, but the ambivalence of programming languages as well. His first shift regards the locus of the metaphor, which he saw as being limited to the single word—a semiotic element—to the whole sentence—a semantic element \citep{ricoeur_rule_2003}. This operates in parallel with his attention to the \emph{lived} feature of the metaphor, insofar it exists in a broader, vital, experienced context. Approaching the metaphor while limiting it to words is counterproductive because words refer back to "contextually missing parts"—they are eminently overdetermined, polysemic, and belong to a wider network meaning than a single, one-to-one relationship\footnote{As he sees it in the traditional, Aristotelician sense of the term.}. Looking at it from the perspective of the sentence brings this rich network of potential meanings and broadens the scope for interpretation. As we've briefly touched upon in the previous section when reading \lstinline{self_inspect.rb}, all of the evocative meaning of the poem isn't contained exclusively in each token, and the power of the whole is greater than the sum of its parts.

Secondly, Ricoeur inspects a defining aspect of a metaphor by the \emph{tensions} it creates. His analysis builds from the polarities he identifies in discourse between event (time-bound) and meaning (timeless), between individual (subjective, located) and universal (applicable to all) and between sense (definite) and reference (indefinite)\footnote{For the extent to which source code can be considered discourse has been discussed, see \citep{cox_speaking_2013}.}. The creative power of the metaphor is its ability to both create and resolve these tensions, to maintain a balance between a literal interpretation, and a metaphorical one—between the immediate and the potential, so to speak. Tying it to the need for language to be fully realized in the lived experience, he poses metaphor as a means to creatively redescribe reality. In the context of syntax and semantics in programming languages, we will see that these tensions can be a fertile ground for poetic creation through aesthetic manifestations. For instance, we can see in \ref{code:cynical-preamble} a poetic metaphor hinging on the concept of the attribute. In programming as in reality, an attribute is a specificity possessed by an entity; in this specific code poem, the tension is established between the computer interpretation and the human interpretation of an attribute. Starting from a political target domain (the constitution of the United States of America), the twist happens in the source domain of the attribute. Loosely attributed by the people in writing, the execution of the declaration (that is, the living together of the United States citizens) implies and relies on the fact that power resides in the people, as is being stated in a literal way. However, from the computer perspective, the definition is not rigorous enough and the execution of the code will throw an error that is shown on the last line—the people have no power.

\begin{listing}
    \inputminted[]{python}{./corpus/cynical_american_preamble.py}
    \caption{Cynical American Preamble, by Michael Carlisle, published in code::art \#0 \citep{brand_code_2019}}
    \label{code:cynical-preamble}
\end{listing}

If the conceptual turn initiated by Lakoff and Johnson's analysis of the metaphor broadens the horizon of their applicability beyond the strict domain of literature, it is nonetheless clear that metaphors appear and operate in particular ways in literary works, from fiction to poetry. We look at such specificity here in anticipation of identifying which features of poetic metaphors could be mapped to the program texts of our corpus—whether explicitly poetic, as in source code poetry, or not, as in regular source code.

So while Lakoff bases poetic metaphors on the broader metaphors of the everyday life, he also operates the distinction that, contrary to conventional metaphors which are so widely accepted that they go unnoticed, the poetic metaphor is \emph{non-obvious}. Which is not to say that it is convoluted, but rather that it is new, unexpected, that it brings something previously not thought of into the company of broad, conventional metaphors—concepts we can all relate to because of the conceptual structures we are already carry with us, or are able to easily integrate. This echoes our mention of Flusser's analysis of poetry as that which brings ideas into the realm of the thinkable (see \ref{subsec:poets}).

It does so along four different axes, in terms of how the source domain affects the target domain that is connected to. First, a source domain can \emph{extend} its target counterpart: it pushes it in an already expected direction, but does so even further, sometimes creating a dramatic effect by this movement from conventional to poetic. For instance, a conventional metaphor would be saying that \emph{"Juliet is radiant"}, while a poetic one might extend the attribution of positivity associated with brightness by saying \emph{"Juliet is the sun}\footnote{From \emph{Romeo and Juliet}, Act 2, Scene 2}.

Poetic metaphors can also \emph{elaborate}, by adding more dimensions to the target domain, while nonetheless being related to its original dimension. Here, dimensions are themselves categories within which the target domain usually falls (e.g. the sun has an astral dimension, and a sensual dimension). Naming oneself as \emph{The Sun-King} brings forth the additional dimension of hierarchy, along with a specific role within that hierarchy—the sun being at the center of the then-known universe.

Metaphors gain poetic value when they \emph{put into question} the conventional approaches of reasoning about, and with, a certain target domain. Here is perhaps the most obvious manifestation of the \emph{non-obvious} requirement, since it quite literally proposes something that is unexpected from a conventional standpoint. When Camus describes Tipasa's countryside as being \emph{blackened from the sun}\footnote{"\emph{A certaines heures, la campagne est noire de soleil}", from \emph{Noces à Tipasa}}, it subverts our pre-conceptions about what the countryside is, what the sun does, and hints at a semantic depth which would go on to support a whole philosophical thought (\emph{la pensée de midi}). Interestingly, the re-edition of L'Étranger for its 70th anniversary can itself be seen as a form of poetic metaphor, since it was published under Gallimard's \emph{Futuropolis} collection. While the actual \emph{Futuropolis} doesn't claim to focus on any sort of science-fiction publications, and rather on illustrations, the very name of the collection applies onto the work of Camus, and of the others published alongside him, can elicit in the reader a sense of a kind of avant-gardism that is still present today.

Finally, poetic metaphors \emph{compose} multiple metaphors into one, drawing from different source domains in order to extend, elaborate, or question the original understanding of the target domain. Such a technique of superimposition creates semantic depth by layering these different approaches. It is particularly at this point that literary criticism and hermeneutics appear to be necessary to expose some of the threads pointed out by this process. As an example, the metaphor of Charles Bovary's cap, a drawn-out metaphorin Flaubert's work which ends up depicting something which clearly isn't a cap, operates by extending the literal understanding of how a cap is constructed, elaborating on the different components of a hat in such a rich and lush manner that it leads the reader to question whether we are still talking about a hat. This metaphorical composition can be interpreted as standing for the orientalist stance which Flaubert takes vis-à-vis his protagonists, or for the absurdity of material pursuit and ornament\footnote{Which ultimately leads Emma to her demise.}, or for the novel itself, whose structure is composed of complex layers, under the guise of banal appearances. Composed metaphors highlight how they exist along \emph{degrees of meanings}, from the conventional to the poetic, and further to the non-sensical.

Through these, we highlight how metaphors \emph{function}, and how they can be identified. Another issue they address is that of the \emph{role} they fulfill in our everyday experiences as well as in our aesthetic experiences. Granted a propensity to structure, to adapt, to reason and to induce value judgment, metaphors are ultimately seen as a means to comprehend the world. By importing structure from the source, the metaphor in turn creates structure in our lives, in our understandings (and thus have power over us). Our understanding grasps these structures through their features and attributes (one might even call them affordances, following Gibson \citep{gibson_ecological_1986}), and integrates them as a given—in what Ricoeur would call a \emph{dead} metaphor. This is one of their key contribution, that metaphors have a function which goes beyond an exclusive, disinterested, self-referential, artistic role. If metaphors are ornament, it is far from being a crime, because these are ornaments which, in combining imagination and truth, expand our conceptions of the world by making things \emph{fit} in new ways.

\subsection{Literature and cognitive structures}
\label{subsec:literature-cognition}

More recent work in aesthetics and literary research have continued in this direction. Building on the focus on conceptual structures, the attention has shifted to the relationship between literature (as part of aesthetic work and eliciting aesthetic experiences) and cognition. This move starts from the limitation of explaing "art for art's sake", and inscribing it into the real, lived experiences of everyday life mentioned above, perhaps best illustrated by the question posed in Jean-Marie Schaeffer's eponymous work—\emph{Why fiction?}. Indeed, if literary and aesthetic criticism are to be rooted in the everyday, and in the conventional conceptual metaphors which structure our lives, our brains seem to be the lowest common denominator, and thus a good starting point for a new contribution to understanding the arts. A similar approach, related to scientific knowledge, can be seen in Polanyi's work on tacit knowledge, in which that which the scientist knows isn't entirely and absolutely formal and abstracted, but rather embodied, implicit, experiential. This limitation of codified, rigorous language when it comes to communicating knowledge, opens up the door for an investigation of how literature and art can help with this communication, while keeping in mind the essential role of the senses and lived experience in knowledge acquisition (i.e. integration of new conceptual structures) \citep{polanyi_tacit_2009}.

Some of the cognitive benefits of art aren't too dis-similar to those posed by Beardsley, but shift their rationale from strict hermeneutics and criticism to cognitive science. These benefits can be pleasure, emotion, or understanding. Terence Cave focuses on the latter when he says that literature \emph{"allows us to think things that are difficult to think otherwise}. We now examine such a possibility from two perspectives: in terms of the role of imagination, and in terms of the role of the senses \citep{lavocat_interpretation_2015}.

Cave posits that literature is an object of knowledge, a creator of knowledge, and that it does so through the interplay between rational thought and imaginative thought, between the "counterfactual imagination" and our daily lives and experiences. Through this tension, this suspension of disbelief is nonetheless accompanied by an epistemic awareness, making fiction reliant on non-fiction, and vice-versa. Working on literary allusions, Ziva Ben-Porat shows that this simultaneous activation of two texts is influenced by several factors. First, the form of the linguistic token itself has a large influence over the understanding of what it alludes to. Its aesthetic manifestation, then, can be said to modulate the conceptual structures which will be acquired by the reader. Second, the context in which the alluding token(s) appears also influences the correct interpretation of such an allusion, and thus the overall understanding of the text. This contextual approach, once again hints at the change of scale that Ricoeur points in his shift from the word to the sentence, and demands that we focus on the whole, rather than single out isolated instances of linguistic beauty. Finally, a third factor is the personal bagage (a personal encyclopedia) brought by the reader. Such a bagage consists of varying experience levels, of quality of the know-how that is to be activated during the reading process, and of the cognitive schemas that readers carry with them. Imagination in literary interpretation, builds on these various aspect, from the very concrete form and choice of the words used, to the unspoken knowledge structures held in the reader's mind, themselves depending on varied experience levels. By allowing the reader to project themselves into potential scenarios, imagination allows us to test out possibilities and crystallize the most useful ones to continue building our conception of the fictional world.

The work of imagination also relies on how the written word can elicit the recall of sensations. This takes place through the re-creation, the evokation of sensory phenomena in linguistic terms, such as the \emph{perceptual modeling} of literary works, which can be defined as (linguistic) simulations relying on the senses to communicate situations, concepts, and potential realities.

This attention to the sense calls for an approach of literary criticism as seen through embodied cognition, which starts from the postulate human cognition is grounded in sensorimotricity, i.e., the ability to feel, perceive, and move. The pervading cognitive process called perceptual simulation, which is activated when we cognitively process a gesture in a real-life situation, is also recruited when we read about actions, movements, and gestures in texts.

Depiciting movement, vision, tactility and other embodied sensations allows us to crystallize and verify the work of the imaginative process. As such, literature unleashes our imaginary by recreating sensual experiences—Lakoff even goes as far as saying that we can only imagine abstract concepts if we can represent them in space\footnote{Geoff Hinton, pioneer of modern deep-learning, has reportedly said that, to visualize 100-dimensional spaces, one should first visualize a 3-dimensional, and then "shout 100 really really loud, over and over again", cited in \citep{akten_journey_2016}}. It seems that the imaginative process depends in part on visual and spatial projections, and suggests the fitness of the conceptual structures depicted. By describing situations which, while fictional, nonetheless are possible in a reality often very similar to the one we live in, it is easy for the reader to connect and understand the point being made by the author. So if literature is an object of knowledge, both sensual and conceptual, offering an interplay between rational and imaginative thought, it still relies on the depiction of mostly familiar situations (the protagonists physiologies, the rules of gravity, the fundamental social norms are rarely challenged). A first issue that we encounter here, in trying to connect source code and computing to this line of thought, is that code has close to no sensual existence, beyond its textual form. In trying to communicate concepts, states and processes related to code and computing, and in being unable to depict them by their own material and sensual properties, we once again resort to linguistic processes which enable the bringing-into-thinking of the program text.

\begin{listing}
    \inputminted[]{java}{./corpus/unhandled_love.java}
    \caption{Unhandled Love, by Daniel Bezera, published in \{code poems\} \citep{bertram_code_2012}}
    \label{code:unhandled-love}
\end{listing}

The code poem listed in \ref{code:unhandled-love} suggests a similar phenomenon when it comes to perceiving motions and sensations through words. The key part of the poem here is the use of the keyword \lstinline{throw}: as a reserved keyword in some of the most popular programming languages, it is known and has been encountered by multiple programmers, as opposed to a word defined in a specific program (such as a variable name). This previous encounters build up a feeling of familiarity and of dread—indeed, the act of the throwing in programming is as dynamic and as violent as in human prose. To throw an object in programming, is to interrupt the smooth execution flow of the program, because something unexpected has happened, oftentimes an exception. Additionally, the title of the poem hints at a supplemental implication of the poems motion; any exception that is thrown should be caught, or handled, by another part of the program, in order to gracefully recover from the mishap and proceed as expected. If it's not handled—as is the case in the poem—the program terminates and the source code itself aborts all function.

Love is therefore depicted here as an exception that must be handled (with care) , and the use of a particularly dynamic keyword elicits such a feeling in a reader who previously had to throw and handle exceptions.

% todo add a bit more of this cognitive engagement with bootz and jérôme pelletier
Philippe Bootz has described this mode of reading (e-lit), not sourec code, in programmed digital poetry as cognitive reading, in which the reader assumes a position of metareader

\subsection{Words in space}
\label{subsec:spatial-literature}
% 5k space

Beyond the use of metaphor, literature allows the reader to engage cognitively with the world of the work, and the interrelated web of concepts that can then be conversed once they are put into words. This process of putting down intention, through language and into written words, is also the process of transforming a time-based continuum (speech) into a space-based discreete sequence \footnote{This process is called grammatization, and is explored further in \citep{bouchardon_valeur_2014}}. This is valid both for human prose and machine languages: the unfathomably fast execution of sequential instructions is manifested as static space in source code.

Literary theory  also engages with the concept of space. We have seen in the subsection above that there is a particular attention being given to movement in space, through embodied cognition; in that case, the use of a specific syntax elicits a rkinetic reaction in the incarnated reader. Second, attention is also being paid to fictional space—that is, the web of relationships, connotations and suggestions that hint at a broader world than the one immediately at hand in a work of literature.

% add this ref to the section above
Guillemette Bolens, Kinesic Humor: Literature, Embodied Cognition, and the Dynamics of Gesture / Le Style des gestes: corporéité et kinésie dans le récit littéraire, Guillemette Bolens

% connect with code as reading direction, and "naturalness" of reading, and refer to the graphical orientation of IDEs and also the people on stack overflow who vertically align their code

%2k
Marie-Laure Ryan - Complexity \url{http://marilaur.info/complexity.pdf} / Mapping and Geography \url{http://marilaur.info/2018-mapping.pdf} / \url{https://ohiostatepress.org/books/BookPages/ryanetal_narrating.html}

Marie Laure Ryan, touch upon space but also narration:
- The storyworld (fiction) and the world of reference (non-fiction)

% find some examples of software development that hints at a world of reference (the hardware one that separates changeable variables for hardware clock)

%1k contribution de gefen
Pour Joseph Carroll, il s'agit de construire une "cartographie cognitive", permettant de rendre l'expérience intelligible (parce que visualo-spatiale?)

Thomas Pavel: le roman est une boite a outil pour comprendre le monde, et cela se passe par *l'habitation* du monde dans lequel le roman voit le jour.

to be complemented with Elaine Scarry (notamment le modelage perceptuel inspiré de la théorie de la perception de James J. Gibson)

%1k on the commentary of this and maybe another one
\begin{listing}
    \inputminted[]{html}{./corpus/nested.html}
    \caption{Nested, by Dan Brown and published in \{code poems\} \citep{bertram_code_2012}}
    \label{code:nested}
\end{listing}

Space is also a core feature of the digital medium. As N. Katherine Hayles states in her eponymous essay, \emph{"print is flat, code is deep"} \citep{hayles_print_2004}, and thus the spatial dimension must be taken into account when the digital object is being engaged with. Janet Murray also puts spatiality as one of the core distinguishing features of digital media, at the forefront of which are digital games\footnote{\emph{"The computer's spatial quality is created by the interactive process of navigation. We know that we are in a particular location because when we enter a keyboard or mouse command the (text or graphic) screen display changes appropriately.} \citep{murray_hamlet_1998}}.

An example of this intertwining of flat textual screen and spatial depth is the overall genre of interactive fiction, which displays prompts for textual interaction on a screen, accompanied with the description of where the reader is currently standing in the fictional world. Exploration can only be done in a linear fashion, entering one space at a time; and yet the system reveals itself to contain spaces in multiple dimensions, connected by complex pathways and relationships. The figure shown in \ref{graphic:howling-dogs} shows the map of porpentine's \emph{howling dogs}, a work of interactive fiction made in the Twine engine \citep{porpentine_howling_2012}, illustrates the distinction between the current position of the reader in the reading process and the broader narrative space.

\begin{figure}
    \includegraphics[width=\textwidth,height=\textheight,keepaspectratio]{howling_dogs_node_map.png}
    \caption{The node map of the game howling dogs highlights the spatiality inherent to interactive digital systems.}
    \label{graphic:howling-dogs}
\end{figure}

% and cd and go to.
As Murray mentions, these features are not limited to those playful interactive systems presented as works to be explored (be it e-literature or digital games), but rather a core component of digitality. Beyond the realm of fiction, one can see instances of this in the syntax used in both programming languages and programming environments (see \ref{subsec:tools-cognition} for an overview of IDEs). For instance, the use of the \lstinline{GOTO} statement in BASIC, of the \lstinline{JMP} and \lstinline{MOV} instructions in x86 Assembly, or the use of the \lstinline{return} in the C family of programming languages all hint at movement, at going places and coming back, representing the non-linear perception of program execution. From the machine perspective, program execution can be considered to be linear, since instructions are executed one after the other. The use of multi-core architecture and parallel processing does complicate this picture, but programmers rarely engage directly with the specification of which CPU core executes which instruction. What they do engage with, is parallel programming, in which things happen simultaneously, thus complicating the picture insofar as two processes being run in parallel imply some sort of distinct semantic spaces to be reflected in the mental model of the programmer.

\spacer

In conclusion, blah blah

%mention specifically cognition as ambiguous interpretation, metaphor, kinetic and mapping worlds of reference

\section{Architecture and understanding}
\label{sec:arch-understanding}

%15k

Start with \url{https://plato.stanford.edu/entries/architecture/} to give an overview, and show which aspect of the research we focus on in this section.

Then check the references \url{https://books.google.de/books/about/Philosophy_for_Architects.html?id=qUJ7KzsXpTsC&redir_esc=y}

%5k  form and function

louis sullivan

robert venturi

%5k patterns, objects and compressions

alexander and oop

patterns and beacons

%5k cognition, landscape cognition

specifically the cognition paper. there isn't much but hey

%5k craftsmen

The world of architecture has accrued its own set of design values over the years. One of those values is the principle of material honesty. One material should not be used as a substitute for another. Otherwise the end result is deceptive.\citep{keith_resilient_2016}

Cette influence des spécificités techniques sur le choix d'une scène ne signifie pas que l'appareil photographique produirait un résultat prévisible, qui ne dirait rien de l'auteur, mais que certaines orientations techniques seraient plus justes. Ce qui donne de l'effet à l'image, c'est cette correspondance entre l'objet et la technique Walter Benjamin in http://www.softphd.com/these/walter-benjamin-authenticites/pensee-declin -> question du matériel

Recherche sur \url{https://beautiful.software/}


\section{Forms of scientific activity}
\label{sec:aesthetic-scientific}

This section looks at the forms of sciences and activity

% total 15k

\subsection{Mathematics and elegance}
\label{subsec:aesthetic-mathematics}
% 10k on elegance, relation between proof and theorem

The point here is to show the stretch between gorgeous abstract theorems and pretty solid mechanisms.

poincaré

Epiphany, enlightenment

all the reading resources from alberto

% 5k on constructivism, papert & cie.

\subsection{Making and understanding}
\label{subsec:aesthetic-engineering}

%10k

Learning by doing, craft, extract from chap 2

\spacer

In conclusion, we have seen that there is a clear connection between aesthetics and cognition, and that it exists across domains. For literature, it is about accessing three-dimensional space through two-dimensional surface and one-dimensional sentences. For architecture, it is about cognition as ability to modify and act within, as well as the ability to derive the meaning of things from their appearances. In mathematics, it is about compressing the maximum amount of insight (which is different from just knowledge) in the minimum amount of explanation/tokens. For engineering it's not quite sure yet, but it's related to architecture: how functional (in the social and technical sense) it is.