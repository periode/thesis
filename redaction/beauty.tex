\chapter{Beauty and understanding}
\label{chap:beauty}

% 80k chars

This chapter focuses on what beauty has to do with understanding, and specifically how the reference domains manifest this relationship between the surface structure and the deep structure. We start with the development of how aesthetics involve "something beyond".

This will have us look at cognition in aesthetics; that is, in much more contemporary terms, starting in the 1960s and 1970s with Goodman. Then, we will go domain by domain and see how cognition and aesthetics are conjugated in each of the domains that are being referred to when we talk about source code beauty: literature, engineering, mathematics and architecture.

% todo: show that there is always some sort of relation established between the surface structure and the deep structure

\section{Aesthetics and cognition}
\label{sec:aesthetic-cognition}

%20k characters, doing an overview of how aesthetic philosophy relates to cognition

- beardsley [beardsley_aesthetic_experience#15 - the metaphorical twist]

nelson goodman

[contini_goodman_art_cognition_education.md], summary of goodman and project zero

[carroll_aesthetic_experience_revisited.md]  theory (affect, axiom, content)

[chatterjee_vartanian_neuroscience_of_aesthetics.md] psychology/cognition, fluency theory

[fauconnier_turner_conceptual_blending.md]

[goodman_the_status_of_style.md], along with[lopes_goodman_symbol_theory.md]

[goody_logic_of_writing.md], goody comes here because he is dealing at the same higher level of the symbol system (orality vs. literacy), and then combine it with [ong_orality_literacy.md]

davis how to make analogies in a digital age

\section{Literature and understanding}
\label{sec:aesthetic-literature}

% 7k with a subsection on metaphor

Ricoeur - Métaphore vive

Lakoff (take away from chap 2.)

barthes 

voleshov and social aesthetics

[mace_styles_critiques_de_nos_formes_de_vie]

rousset: forme et signification

[portela_scripting_reading_motions]

% 4k specificity of the digital?

Bouchardon - Valeur heuristique de la littérature numérique

N. Katherine Hayles - Speech, Writing Code, My Mother Was A Computer

% 7k and a subsection on space

% make it clear what are the particular techniques used to spatialize text
Françoise Lavocat - interprétation et sciences cognitives, dimension spatiale de la représentation textuelle.

Jérôme Pelletier -  L'attrait esthétique de la fiction : un point de vue de philosophie cognitive /  La Fiction comme Culture de la Simulation 

Marie-Laure Ryan - Complexity \url{http://marilaur.info/complexity.pdf} / Mapping and Geography \url{http://marilaur.info/2018-mapping.pdf} / \url{https://ohiostatepress.org/books/BookPages/ryanetal_narrating.html}

\section{Architecture and understanding}
\label{sec:arch-understanding}

%15k

\url{https://beautiful.software/}

habitability, navigation, landscape cognition, relationship between form and function

patterns and beacons

compression

\section{Forms of scientific activity}
\label{sec:aesthetic-mathematics}

This section looks at the forms of sciences and activity

% total 15k

% 10k on elegance, relation between proof and theorem

poincaré

Epiphany, enlightenment

all the reading resources from alberto

% 5k on constructivism, papert & cie.

\section{Engineering and understanding}
\label{sec:aesthetic-engineering}

%10k

Learning by doing, craft, extract from chap 2

\spacer

In conclusion, we have seen that there is a clear connection between aesthetics and cognition, and that it exists across domains. For literature, it is about accessing three-dimensional space through two-dimensional surface and one-dimensional sentences. For architecture, it is about cognition as ability to modify and act within, as well as the ability to derive the meaning of things from their appearances. In mathematics, it is about compressing the maximum amount of insight (which is different from just knowledge) in the minimum amount of explanation/tokens. For engineering it's not quite sure yet, but it's related to architecture: how functional (in the social and technical sense) it is.