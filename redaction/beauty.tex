\chapter{Beauty and understanding}
\label{chap:beauty}

This chapter focuses on what beauty has to do with understanding, and specifically how the reference domains manifest this relationship between the surface structure and the deep structure. We start with the development of how aesthetics involve "something beyond".

This will have us look at cognition in aesthetics; that is, in much more contemporary terms, starting in the 1960s and 1970s with Goodman. Then, we will go domain by domain and see how cognition and aesthetics are conjugated in each of the domains that are being referred to when we talk about source code beauty: literature, engineering, mathematics and architecture.

% todo: show that there is always some sort of relation established between the surface structure and the deep structure

\section{Aesthetics and cognition}
\label{sec:aesthetic_cognition}

\section{Literature and understanding}
\label{sec:aesthetic-literature}

Ricoeur - Métaphore vive

Françoise Lavocat - interprétation et sciences cognitives, dimension spatiale de la représentation textuelle.

Bouchardon - Valeur heuristique de la littérature numérique

N. Katherine Hayles - Speech, Writing Code, My Mother Was A Computer

Jérôme Pelletier -  L'attrait esthétique de la fiction : un point de vue de philosophie cognitive /  La Fiction comme Culture de la Simulation 

Marie-Laure Ryan - Complexity \url{http://marilaur.info/complexity.pdf} / Mapping and Geography \url{http://marilaur.info/2018-mapping.pdf} / \url{https://ohiostatepress.org/books/BookPages/ryanetal_narrating.html}

\section{Architecture and understanding}
\label{sec:arch_understanding}

\url{https://beautiful.software/}

habitability, navigation, landscape cognition, relationship between form and function

patterns and beacons

compression

\section{Mathematics and understanding}
\label{sec:aesthetic-mathematics}

Epiphany, enlightenment

\section{Engineering and understanding}
\label{sec:aesthetic-engineering}

Learning by doing, craft.

\vspace{1\baselineskip}
\centerline{\rule{0.13334\linewidth}{.4pt}}
\vspace{1\baselineskip}

In conclusion, we have seen that there is a clear connection between aesthetics and cognition, and that it exists across domains. For literature, it is about accessing three-dimensional space through two-dimensional surface and one-dimensional sentences. For architecture, it is about cognition as ability to modify and act within, as well as the ability to derive the meaning of things from their appearances. In mathematics, it is about compressing the maximum amount of insight (which is different from just knowledge) in the minimum amount of explanation/tokens. For engineering it's not quite sure yet, but it's related to architecture: how functional (in the social and technical sense) it is.