\chapter{Beauty and understanding}
\label{chap:beauty}

This chapter provides background argumentation for what beauty has to do with understanding. First from a theoretical perspective, and then diving specifically into how specific domains approach this relation. Our theoretical approach will be start from the aesthetic theory of Nelson Goodman, and a lineage which links aesthetics to cognition, most recently aided by the contribution of neurosciences. We will see how source codes does qualify as a language of art—that is, a symbol system which allows for aesthetic experiences.

After argumenting for a conception of aesthetics which tends to intellectual engagement, we will pay attention to how surface structure and conceptual assemblages relate. That is, we will highlight how each of the domains contigent to source code— literature, mathematics and architecture—communicate certain concepts through their respective and specific means of symbolic representation. The identification of how specific aesthetic properties enable cognitive engagement in each of these domains will in turn support the identification of how equivalent properties can manifest in source code.

This thesis argues that aesthetics have a useful component, insofar as formal arrangments at the surface-level can facilitate the understanding of the underlying deep structure of concepts denotated. In the specific context of source code, we show that aesthetic standards are contextual, as they vary along two axes. First, they depend on whether the attention of the writer (and thus the reader) is directed at the hardware, or at the software (which can, in turn, address real-world ideas, or computational ideas). Second, they depend on the socio-technical context in which source code is written, a context constituted of whether the program text is read-only or read-write, and of whether the intent is for the program text to be primarily functional, educational or entertaining.

\section{Aesthetics and cognition}
\label{sec:aesthetic-cognition}

The way that things are presented formally has been empirically shown to affect the comprehension of content. Without engaging too directly in the media-determination thesis, which states that what one can say is determined by the medium through which they say it, be it language or technical media \citep{postman_amusing_1985}, we nonetheless do start from the point that form influences the perception of content.

Jack Goody and Walter Ong have shown in their anthropological studies that the primary means of communication of the surveyed communities does affect the engagement of said communities with concepts such as ownership, history and governance \citep{ong_orality_2012,goody_logic_1986}. More recently, Edward Tufte and his work on data visualization have furthered this line of research by focusing on the translation of similar data from textual medium to graphic medium \citep{tufte_visual_2001}. Several cases have thus been made for the impact of appearance towards structure, both in source code and elsewhere. Here, we intend to generalize this comparative approach between several mediums, by looking at how source code performs expressively as a language of art, stemming from Nelson Goodman's theorization of such a languages.

\subsection{Source code as a language of art}
\label{subsec:source-code-language-art}

Moving away from the question of the nature of the aesthetic experience from the perspective of the audience, whether as an aesthetic emotion being felt or as an aesthetic judgment being given, we shift our attention to the object of aesthetic experience, and to the questions of \emph{how does a program text represent?} and \emph{what does a program text represent?}. To answer these, we rely on the approaches provided by Nelson Goodman in the \emph{Languages of Art: An Approach to a Theory of Symbols} \citep{goodman_languages_1976}.

The starting point for Goodman's analysis is that production and understanding in the arts involve human activities that, though they differ in specific ways among themselves and from other activities, are nevertheless generically related to perception, scientific inquiry, and other cognitive activities, since both artistic and scientific activities involve symbolic systems. It is those two components that Goodman aims at expliciting: what constitutes an aesthetic symbol system, and how does such a system express?

Goodman develops a systematic approach to symbols in art, freed from any media-specificity (e.g. f from clocks to counters, from diagrams to maps models, from musical scores to painters' sketches and linguistic scripts). A symbolic system, in his definition, consists of characters, along with rules to govern their combination with other characters, itself correlated with a field of reference. These symbols and their arrangement within a work of art supports an aesthetic experience\footnote{It should be noted here that Goodman does not limit the aesthetic experience to a positive, pleasurable one. An artistic symbolic system can be seen even if the result is considered bad.} and, since they are syntactic system which operate at the semantic level, they can be rigorous communicative systems.

A symbol system is based on requirements which might indicate that the work created in such a system would be able to elicit an aesthetic experience\footnote{Goodman approaches it as such: "\emph{Perhaps we should being by examining the aesthetic relevance of the major characteristics of the several symbol processes involved in experience, and look for aspects or symptoms, rather than for crisp criterion of the aesthetic. A symptom is neither a necessary nor a sufficient condition for, but merely tends in conjunction with other such symptoms to be present in, aesthetic experience}" \citep{goodman_languages_1976}}. Such a system should be composed of signs which are syntactically and semantically disjointed, syntactically replete and semantically dense \citep{goodman_languages_1976}. This classification makes it possible to compare the way various symbolization systems used in art and sience express concepts. In our case, this provides us for a framework to investigate the extent to which source code qualifies as a language of art.

Source code is written in a formal linguistic system called a programming language. Such a linguistic system is digital in nature, and therefore satisfies at least the two requirements of syntactic disjointedness (no mark can be mistaken for another) and differentiation (a mark only ever corresponds to that symbol). Indeed, this is due to the fact that these requirements are fulfilled by any numerical or alphabetical system, as programming languages are systems in which alphabetical characters are ultimately translated into numbers. While not as syntactically dense as music or paint, it is nonetheless unambiguous.

Third, the requirement of syntactic repleteness demands that relatively fewer factors need to be taken into account during the interpretative process\footnote{Goodman mentions the symptom that such a system might engender: "\emph{[\dots] relative syntactic repleteness in a syntactically dense system demands such effort at discrimination along, so to speak, more dimensions}" \citep{goodman_languages_1976}}. On one hand, we can consider that any additional aspects of the source code (such as the display font or the syntax highlighting discussed in \ref{subsec:tools-cognition}) are ultimate irrelevant to the computer, thus making it a poorly replete symbol sytem. On the other hand, the importance of such factors, along with abilities to write a program with the same function but with different syntax, pleads for a relatively replete syntactical system. The tendency of program text to veer towards verbosity indeed implies this desirable state of repleteness: more subtleties and intermediate syntax can be added within any proposition, always implying the possibility of clarifying, or obfuscating—both being, as we have seen, different kinds of aesthetic experiences.

Finally, semantic density refers to whether or not there is a limit to the amount of concepts that the symbol system can refer to. As we have shown in \ref{code:representation}, the affordances that programming languages provide to represent phenomena and concepts from the problem domain fulfill this requirement. While we have been previously concerned with syntax, this ability of programming languages to refer to a problem domain which has not yet shown its limitations at the semantic level is one which gives it representational power beyond strict computational concepts.

As Goodman notes, the distinct signs that compose a symbols system do not have intrinsic properties, but a mark serves as a sign only in relation to a symbol system, and to a field of reference. The field of reference is understood here as being the set of concepts which are being referred to by a symbolic system. For instance, a symbolic system such as western classical music can refer to concepts such as lament, piety, heroism or grace, while a chinese \emph{shanshui} painting has a landscape composed of mountains and rivers, as well as concepts of harmony, complementarity, presence and absence, as its field of reference. The combination of both the problem domain, as evoked in \ref{subsubsec:modelling-complexity}, and of the technological environment on which the source code is to be executed, developed in \ref{subsec:style-idioms-programming}, are posited here as an equivalent to the Goodman's field of reference.

It thus seems like source code satisfies to a large extent the critieria to be a language of art, meaning that it exhibits some of the properties which tend to elicit an aesthetic feeling. Most notably, it does not possess a very dense syntax, nor can it be considered replete both from the perspective of the computer and of the human\footnote{See \ref{subsec:programming-languages} for a discussion of syntactic limitation in programming languages, also known as orthogonality.}, but it nonetheless refers possesses a certain amount of semantic density. Its ability to connect to a particular field of reference, such as hardware, mathematics, or the world at large is another aspect of being a language of art, and is an important part of how programming languages can communicate concepts.

Goodman highlights the ways in which symbols systems communicate, through the notion of \emph{reference}. To refer to, in this sense, is the action by which a symbol stands in for an item or an idea. Reference, he sketches out, takes place through the different dyads of denotation and exemplification, description and representation, possession and expression \citep{goodman_languages_1976}. We will see how these various means of referring can be instantiated in the symbolic system of source code.

Denotation is the core of representation, a reference from a symbol to one or many objects it applies to and is independent of resemblance. To refer, it uses a particular relationship via the use of labels, in which a symbol stands in for an item in the field of reference. For instance, a name denotes its bearer and a predicate each object in its extension. Names such as variable names or function names thus denote a particular item in the field of reference, and act as their label. For instance, \lstinline{var auth_level} denotes an ability to access and modify resources; the first token \lstinline{var} is chosen by the language designer, while the second token \lstinline{auth_level} is chosen by the programmer.

The labelling process therefore serves as the symbolic expression for a particular field. In source code, this can happen through variable naming, but also through type definition\footnote{For instance, a particular choice of a numeric value, such as \lstinline{int} or \lstinline{float} denote a particular level of preciseness}, as well as additional affordances which we look at in \ref{sec:cognitive-aesthetics}, such as the layering of semantic references and the establishment of habitable cognitive structures.

Source code also make extensive use of description. If we consider a program text as a series of steps, a series of states, or a series of instructions, then it follows that source code is explicitly describing the algorithm used—the how of the program, rather than the why. Indeed, a program text is a description of how to solve a problem from the computer's perspective, written extensively in machine language\footnote{Pseudo-code is therefore a representation of a potential source code written in a specific language.}. All source code can therefore be said to be a description of a combination of states (data) and actions (functionality).

States are also a particular case in source code: they are both a description and, because they are not the thing itself, they are also a representation. As one can see in \ref{code:representation}, an individual can be represented within source code with a particular construct in which states and actions are encapsulated. Interestingly, this representation of a concept as an object in soure code does not imply that it reveals the intrinsic properties of the object; rather, these properties appear as they are given by the modelling process of source code syntax. As a symbol system, source code thus proposes a model of the world in which objects have properties; a slightly different representation is therefore always possible.

\begin{listing}
    \inputminted{java}{./corpus/representation.java}
    \caption{An example of how source code can be a representation an individual, and can exemplify encapsulation, written in Java.}
    \label{code:representation}
\end{listing}

This representation, in the specific instance of object-oriented programming in \ref{code:representation}, also manifests Goodman's aesthetic symptom of possession. Here, the source code posseses similar properties as the thing referenced (since our prototypal image of a person has an age, a name and interests). Through this possession of a property, it acts as an example of a prototypal person.

Exemplification is another aspect of Goodman's theory, which has nonetheless remained somewhat limited \citep{elgin_making_2011}. A symbol exemplifying, also called an examplar, is considered as a stand-in for an item in the field of reference. We have seen source code act as an example in \ref{subsec:scientists}, where a particular program text is written in order to stand in for a broader concept. For instance, a program text can, at a lower level, exemplify a particular kind of procedure, such as encapsulation (see \ref{code:representation}) or nestedness. The program text therefore exemplifies the constitutive element of the linked list\footnote{A linked list is a basic data structure in computer science, which consists in a succession of connected objects.}. However, a similar program text can also be an example of cleanliness, of clarity, or elegance. A program text written by a software developer can be seen as possessing the property of cleanliness (see \ref{code:clearer_method_c} in \ref{subsubsec:simple}), by virtue of its implementation of syntactic and semantic rules, while another program text written by a hacker can be seen as highlighting detailed hardware knowledge s(ee \ref{code:smr_c} in \ref{subsubsec:cleverness}).

Different implementations of a concept are necessary but not sufficient for aesthetic judgment, whether these different implementations are virtual or actual. The comparative approach is the one which enables the labelling of \emph{good} or \emph{bad} only insofar as there is a relative \emph{worse} or \emph{better}, respectively. Additionally, the features which a symbol exemplifies always depend on its function (or, more precisely, its functional context) \citep{elgin_understanding_1993}. As we show in \ref{subsec:functions-source-code}, a symbol can perform a variety of functions: a piece of code in a textbook might exemplify an algorithm, while the same piece of code in production software might be seen as a liability, or denote boredom in a code poem. It is then both the possibility of alternative implementations and the reality of the current implementation context which give the exemplification of program texts its aesthetic potential.

Source code maintains a specific kind of relation to the field of reference. The particular class of characters employed as symbols (called \emph{tokens} in the context of programming languages), involves a separation between name, value and address, and as such does not guarantee a direct relationship with the items in the field of reference, we can see in the line \lstinline{unsigned three = 1;} of  \ref{code:multiple_references}, where the reference of the name is not the same reference as the value. That is, in program texts, two distinct symbols can be referring to the same concept, value, or place in memory, something Goodman nonetheless assigns as another symptom of the aesthetic: multiple and complex references.

\begin{listing}
    \inputminted{c}{./corpus/multiple_references.c}
    \caption{An example from the Linux kernel showing that the name and the value of a variable might refer to different things \citep{linux_fs_2023}.}
    \label{code:multiple_references}
\end{listing}

On the other hand, the representation of a field of reference is done through a disjointed and differentiated system: the boundaries of each items in the field of reference are clearly defined, in virtue of the specific symbol system that programming languages are. It is their combination which, in turn, enables complex interplay of references.

We have shown here that source code qualifies as a symbolic system susceptible of affording symptoms of the aesthetic. We have also highlighted its specificities, particularly in terms of descriptions and representations through a restricted syntactic system enabling complex and multiple references, due to it being a language across human and machine understanding. Source code is thus written in a specific kind of symbol system, one which counts as a language of art, but does with restricted syntax and expansive semantics. 

A final aspect to investigate is the expressiveness of source code, with a particular attention to how source code can manifest of metaphorical exemplification and representation. One particular expressive power of an aesthetic experience surfaces when the examplification involves a foreign element, an event that Goodman refers to as metaphorical exemplification. While this approach has been broadened by Lakoff et. al., and mentioned in \ref{subsec:metaphor-computation}, other philosophers of art have also pinpointed the metaphorical event as a reliable symptom of the aesthetic.

Max Black initiates a view of metaphors which go beyond a simple comparison; dubbed the \emph{interaction view}, he considers the metaphorical device as containing positive cognitive content, rather than simply entertaining or limiting \citep{black_metaphor_1955}. Against a traditional view of metaphor being a word which stands in for another, Black reveals a large web of interactions which prove harder to disentangle, beyond usual similarities between two words\footnote{"\emph{Reference to 'associated commonplaces' will fit the commonest cases where the author simply plays upon the stock of common knowledge (and common misinformation) presumably shared by the reader and himself. But in a poem, or a piece of sustained prose, the writer can establish a novel pattern of implications for the literal uses of the key expressions, prior to using them as vehicles for his metaphors. [\dots] Metaphors can be supported by specially constructed systems of implications, as well as by accepted commonplaces; they can be made to measure and need not be reach-me-downs.}" \citep{black_metaphor_1955}.}. Simply paraphrasing a metaphor, even if one captures precisely the same connotations/associations as the metaphor, does not convey the same meaning as the metaphor itself. For instance, saying "\emph{Je chavire dans l'embrun des phénomènes}"\footnote{Literally translated as "\emph{I capsize in the spray of phenomena}"} \citep{beckett_molloy_1982} does not have the similar expressive power as listing all the properties of \emph{phénomènes}. The use of the verb capsize in conjunction with spray relates to the domain of navigation, while capsize alone tends more to a dynamic movement, and spray to uncertainty and bluriness of shape. Phenomenas of the world are all requalified in the light of these new kinetic and perceptual associations.

Through his contribution to aesthetic philosophy, Monroe Beardsley's started touching upon metaphor from a semantic perspective. Published alongside his inquiries into the aesthetic character of an experience, \emph{The Metaphorical Twist} implies that semantics and aesthetics might be connected through the structuring operation of the metaphor—that which elicits an aesthetic experience can do so through the creation of unexpected, or previously unattainable meaning. Beardsley's conception is that metaphor can have a designative role (the primary subject) which adds a "\emph{local texture of irrelevance}", a "\emph{foreign component}", whose semantic richness might over-reach and obfuscate the intended meaning, as well as a connotative one (the secondary subject), in which meaning is peripheral \citep{beardsley_metaphorical_1962}. The cognitive stimulation and enlightment takes place through a metaphor-induced tension, between central and periphery, between illuminating and obfuscating, between evidence and irrelevance.

As Beardsley inquiries into the features necessary for an aesthetic experience, of which the metaphor is part, he lists five criteria to distinguish the character of such an experience. Besides object-directedness, felt-freedom, detached-affect and wholeness, is the criteria of \emph{active discovery}, which is

\begin{quote}
    a sense of actively exercising the constructive powers of the mind, of being challenged by a variety of potentially conflicting stimuli to try and make them cohere; exhilaration in seeing connections between percepts and meanings; a sense of intelligibility \citep{beardsley_aesthetic_1970}.
\end{quote}

As such, Beardsley highlights the possibility of an aesthetic experience to make understandable, to unlock new knowledge in the beholder, and he considers metaphors as a way to do so. The stages he lists go from (1) the word exhibiting properties, to (2) those properties being made into meaning, and finally into (3) a staple of the object, consolidating into (or dying from becoming) a commonplace. This interplay of a metaphor being integrated into our everyday mental structures, of poetry bringing forth into the thinkable, and in the creation of a tension for such bringing-forth to happen, makes the case for at least one of the consequences of an aesthetic experience, and therefore one of its functions: making sense of the complex concepts of world.

Finally, Catherine Elgin has pursued the work of Goodman by furthering the inquiry into arts as a branch of epistemology. Drawing on the work mentioned above, she investigates the relationship between art and understanding, considering how interpretively indeterminate symbols advance understanding \citep{elgin_understanding_2020}, and that it does so in the context of interpretive indeterminacy. As syntactically and semantically dense symbol systems are used in artworks, it is this multiplicity in interpretations which requires sustained cognitive attention with the artwork. To explain these multiple interpretations, the metaphor is again presented the key device in explaining the epistemic potency of aesthetics, based on an interpretative feedback loop from the viewer. And yet, in the context of source code, this interpretation is always shadowed by its machine counterpart—how the computer interprets the program.

\subsection{Contemporary approaches to art and cognition}
\label{subsec:art-cognition-contemporary}

We have drawn from existing work in philosophy of art, in order to map out the expressive power of a given formal representation, as a traditional pre-requisite to the gaining of art status of an object, and highlighted the role of metaphors in engaged cognition during an aesthetic experience. Contemporary literature, and the emergence of neuroscientific studies of such aesthetic experience seem to confirm empirically this approach, and highlight as well two related additional components: sequential experience and skill levels.

The aesthetic experience—that is, the positively received perception of a natural or crafted object—has traditionally been laid out across multiple axes, with more or less overlap. The axes involved in this positive perception include an emotional response, a harmonious assessment, an axiomatic adherence or disinterested pleasure, and have been the topic of debates amongst philosophers for centuries \citep{peacocke_aesthetic_2023}.

Noël Carroll sums up these different directions under the broad areas of affect, axiom and content ultimately considering a content-based approach as the most fruitful \citep{carroll_aesthetic_2002}. First, he underlines how an aesthetic experience dictated by affect removes the object from one's assessment of purpose, value and effect, and limiting it to form, following Kant's principle of disinterested pleasure via passive contemplation. As such, a flower, a sunset or a musical melody can evoke affective aesthetic experiences. Yet, the supposed tendency of this kind of experience to release us from worldly concerns fails, for Carroll, to encompass aesthetic experiences that are rooted in so-called worldly concerns—such as a documentary photography, skillful physical performance, or delicatedly crafted glassware—and is therefore unsatisfying as a root explanation for the aesthetic experience.

An axiomatic aesthetic experience is based on the sort of value that the object is being associated with—such as depiction of religious topics or a manifestation of a particular style. While Carroll does acknowledge a certain virtue of this aesthetic experience in terms of contribution to group cohesion through shared values and imaginaries, its limitations are found in a pre-existing answer to the value judgment that is being bestowed upon the object: the material and sensual properties of the object at hand are irrelevant since their quality is already decided \emph{a priori}.

It is in the content approach that Carroll finds the most satisfying answer to what the aesthetic experience is. Content, here, is defined as the significant forms being apprehended, along with its combinations, juxtapositions and comparisons with other forms\footnote{"\emph{Whereas affect-oriented approaches tend to identify aesthetic experience in terms of certain distinctive experiential qualia or feeling tones, such as being lifted out of the flow of life, content-oriented approaches proceed by distinguishing the specific objects of said experiences.}" \citep{carroll_aesthetic_2002}.}. When we engage with the sensual aspects or an object, our attention is indeed directed first and foremost at what the object looks like, rather than how it makes one feel, or what value system it belongs to. More specifically, Carroll notes, if attention is directed with understanding to the form of the art work or to its expressive and aesthetic properties or to the interaction between those features, then the experience is said to be aesthetic \citep{carroll_aesthetic_2002}.

Form, and the attention paid to it, will thus be taken as our starting point.  This content approach to form, i.e. the set of appearing choices intended to realize the purpose of the artwork, also involves questions of function, implied by the presence of purpose pertaining to an artwork. Particulary, how does the object of aesthetic experience manifest such a purpose, in a way that it can be correctly judged, insofar as its perceived form and perceived purpose are aligned, distinct from any emotional or axiomatic charge?

We can find an answer in the study conducted by Anjan Chatterjee and Oshin Vartanian on the evaluation of the aesthetic experience from a neuroscientific point of view. Like Carroll, they highlight three different perspectives: a sensory-motor perspective, loosely mapped to an affective experience, an emotion-valuation perspective, similar to an axiological experience, and a meaning-knowledge experience, which we equate to the content approach to the aesthetic experience \citep{chatterjee_neuroscience_2016}.

Importantly, they make the distinction between an aesthetic judgment, which emanates from the process of understanding the work, and an aesthetic emotion, which follows from the ease of acquisition of such an understanding. Without being mutually exclusive, these two pendants are related to the amount of engagement provided by the person who aesthetically experiences the object. One can have an aesthetic emotion without being able to provide an aesthetic judgment, a case in which one does not hold enough expertise to apprehend or appreciate a particular realisation. In this sense, the aesthetic judgment, unlike the aesthetic emotion, requires something additional. This conditioning of the aesthetic experience to a certain kind of pre-existing knowledge or skill is supported by the authors' mention of the theory of fluency-based aesthetics \citep{chatterjee_neuroscience_2016}, and their view builds on models that frame aesthetic experiences as the products of sequential and distinct information-processing stages, each of which isolates and analyzes a specific component of a stimulus (e.g., artwork).

These stages, drawn from Leder et. al's model, are based on empirical observation in scientific studies which segment an aesthetic experience in sequential steps \citep{leder_model_2004}. These evolve form perception, to implicit classification, explicit classification, cognitive mastering and fianlly evaluation—that is, fully-qualified aesthetic judgment. This conception is concomittant to Rebert et. al.'s proposal for an aesthetic framework based on processing fluency, which they define as a function of the perceiver's processing dynamics: the more fluently the perceiver can process an object, the more positive is her aesthetic response \citep{reber_processing_2004}. While they focus their study on perceptual fluency, tending to traditional aesthetic features such as symmetry, contrast and balance; they also consider conceptual fluency as an influence on the aesthetic experience, through the attention given to the meaning of a stimulus and the relation of form to semantic knowledge structures. Such a conceptualizing thus hints at a similar skill-based, contextual framework which we have seen emerge in the aesthetic judgment of source code, and yet an additional establishment of a relation between truth and beauty\footnote{"\emph{these findings suggest that judgments of beauty and intuitive judgments of truth may share a common underlying mechanism. Although human reason conceptually separates beauty and truth, the very same experience of processing fluency may serve as a nonanalytic basis for both judgments.}" \citep{reber_processing_2004}}.

This approach of cognitive ease, which we've already identified in the conclusion of \ref{chap:ideals}, is finally echoed in the view that Gregory Chaitin, a computer scientist and mathematician, offers of comprehension as compression. By considering that the understanding of a topic is correlated with the lower cognitive burden experienced when reasoning about such topic, Chaitin forms a view in which an individual understands better through a properly tuned model—a model that can explain more with less \citep{zenil_compression_2021}. In this sense, aesthetics help compress concepts, which in turn allows someone told hold more of these concepts in short-term memory, and grasp a fuller picture, so to speak.

These studies thus show a particular empirical attention to the cognitive engagement with respect to the apprehension an object from an aesthetic perspective, as opposed to passive contemplation or value-driven aggreement. While these other types of experiences remain valid when apprehending such an object, we do focus here on this specific kind of experience: the cognitive approach to the aesthetic experience. Going back to Goodman, he describes such an experience as involving:

\begin{quote}
    making delicated discriminations and discerning subtle relationships, identifying symbol systems and what these characters denote and exemplify, interpreting works and reorganizing the world in terms of works of art and works in terms of the world. \citep{goodman_languages_1976}
\end{quote}

\spacersmall

In this section we've glanced at an overview of research on how cognitive engagement is involved in an aesthetic experience, both from the point of view of the philosophy of art and from cognitive psychology. However, highlighting this involvment does not immediately explicit the nature and details of such cognitive engagement. Speaking in terms of form and object are higher-level concepts tend to erase the specificities of the various systems of aesthetic properties, and how their arrangement expresses various concepts.

Now that we have sketched out an understanding of source code as a symbolic system supporting an aesthetic experience, we must provide a more detailed account of the specificities of source code. To do so, we first turn to a comparative approach, looking at the set of aesthetic domains metaphorically connected to source code through programmer discourse, and we analyse how each of these domains involve cognition in their formal presentations.


\section{Literature and understanding}
\label{sec:aesthetic-literature}

Literature as a cognitive device relies, as we've seen in \ref{sec:ideals-beauty}, on the use of metaphors to provide a new perspective on a familiar concept, and hence complement and enrich the understanding that one has of it. While Lakoff and Johnson's approach to the conceptual metaphor will serve a basis to explore metaphors in the broad sense across software and narrative, we also argue that Ricoeur's focus on the tension of the \emph{statement} rather than primarily on the \emph{word} will help us better understand some of the aesthetic manifestations of software metaphors, without being limited to tokens, but going beyond to statement and structure. Following a brief overview of his contribution, we examine the various uses of metaphor in software and in literature, touch upon the cognitive turn in literary studies, and conclude with an account of how this turn involves further thinking into the spatial and temporal properties of the written word.

\subsection{Literary metaphors}
\label{subsec:literary-metaphors}

Writing in \emph{The Rule of Metaphor}, Ricoeur operates two shifts which will help us better assess not just the inherent complexity of program texts, but the ambivalence of programming languages as well. His first shift regards the locus of the metaphor, which he saw as being limited to the single word—a semiotic element—to the whole sentence—a semantic element \citep{ricoeur_rule_2003}. This operates in parallel with his attention to the \emph{lived} feature of the metaphor, insofar it exists in a broader, vital, experienced context. Approaching the metaphor while limiting it to words is counterproductive because words refer back to "contextually missing parts"—they are eminently overdetermined, polysemic, and belong to a wider network meaning than a single, Aristotelician, one-to-one relationship. Looking at it from the perspective of the sentence brings this rich network of potential meanings and broadens the scope for and the depth of interpretation. As we develop in \ref{subsec:semantic-layers} in our reading of \ref{code:self_inspect}, not all of the evocative meaning of the poem are contained exclusively in each token, and the power of the whole is greater than the sum of its parts.

Secondly, Ricoeur inspects a defining aspect of a metaphor by the \emph{tensions} it creates. His analysis builds from the polarities he identifies in discourse between event (time-bound) and meaning (timeless), between individual (subjective, located) and universal (applicable to all) and between sense (definite) and reference (indefinite). The creative power of the metaphor is its ability to both create and resolve these tensions, to maintain a balance between a literal interpretation, and a metaphorical one—between the immediate and the potential, so to speak. Tying it to the need for language to be fully realized in the lived experience, he poses metaphor as a means to creatively redescribe reality. In the context of syntax and semantics in programming languages, we will see that these tensions can be a fertile ground for poetic creation through aesthetic manifestations. For instance, we can see in \ref{code:cynical-preamble} a poetic metaphor hinging on the concept of the attribute. In programming as in reality, an attribute is a specificity possessed by an entity; in this code poem, the tension is established between the computer interpretation and the human interpretation of an attribute. Starting from a political target domain (the constitution of the United States of America), the twist happens in the source domain of the attribute. Loosely attributed by the people in writing, the execution of the declaration (that is, the living together of the United States citizens) implies and relies on the fact that power resides in the people, as is being stated in a literal way. However, from the computer perspective, the definition is not rigorous enough and the execution of the code will throw an error that is shown on the last line—the people have no power.

\begin{listing}
    \inputminted[]{python}{./corpus/cynical_american_preamble.py}
    \caption{Cynical American Preamble, by Michael Carlisle, published in code::art \#0 \citep{brand_code_2019}}
    \label{code:cynical-preamble}
\end{listing}

In such case, the expressiveness of the program text can be said to derive from the continuous threading of metaphorical references, weaving the properties of computational objects and the properties of conceptual objects in order to deep the mapping from one unto the other.

So while Lakoff bases poetic metaphors on the broader metaphors of the everyday life, he also operates the distinction that, contrary to conventional metaphors which are so widely accepted that they go unnoticed, the poetic metaphor is non-obvious. Which is not to say that it is convoluted, but rather that it is new, unexpected, that it brings something previously not thought of into the company of broad, conventional metaphors—concepts we can all relate to because of the conceptual structures we are already carry with us, or are able to easily integrate.

Poetic metaphors deploy their expressive powers along four different axes, in terms of how the source domain affects the target domain that is connected to. First, a source domain can \emph{extend} its target counterpart: it pushes it in an already expected direction, but does so even further, sometimes creating a dramatic effect by this movement from conventional to poetic. For instance, a conventional metaphor would be saying that \emph{"Juliet is radiant"}, while a poetic one might extend the attribution of positivity and dramatic important associated with brightness and daylight by saying \emph{"Juliet is the sun}\footnote{From \emph{Romeo and Juliet}, Act 2, Scene 2}.

Poetic metaphors can also \emph{elaborate}, by adding more dimensions to the target domain, while nonetheless being related to its original dimension. Here, dimensions are themselves categories within which the target domain usually falls (e.g. the sun has an astral dimension, and a sensual dimension). Naming oneself as \emph{The Sun-King} brings forth the additional dimension of hierarchy, along with a specific role within that hierarchy—the sun being at the center of the then-known universe.

Metaphors gain poetic value when they \emph{put into question} the conventional approaches of reasoning about, and with, a certain target domain. Here is perhaps the most obvious manifestation of the \emph{non-obvious} requirement, since it quite literally proposes something that is unexpected from a conventional standpoint. When Albert Camus describes Tipasa's countryside as being \emph{blackened from the sun}\footnote{"\emph{A certaines heures, la campagne est noire de soleil.}" \citep{camus_noces_1972}}, it subverts our pre-conceptions about what the countryside is, what the sun does, and hints at a semantic depth which would go on to support a whole philosophical thought, knowns as \emph{la pensée de midi}, or \emph{the noon-thought}\footnote{Interestingly, the re-edition of L'Étranger for its 70th anniversary can itself be seen as a form of poetic metaphor, since it was published under Gallimard's \emph{Futuropolis} collection. While the actual \emph{Futuropolis} doesn't claim to focus on any sort of science-fiction publications, and rather on illustrations, the very name of the collection applies onto the work of Camus, and of the others published alongside him, can elicit in the reader a sense of a kind of avant-gardism that is still present today.}.

Finally, poetic metaphors \emph{compose} multiple metaphors into one, drawing from different source domains in order to extend, elaborate, or question the original understanding of the target domain. Such a technique of superimposition creates semantic depth by layering these different approaches. It is particularly at this point that literary criticism and hermeneutics appear to be necessary to expose some of the threads pointed out by this process. As an example, the symbol of Charles Bovary's cap, a drawn-out metaphor in Flaubert's \emph{Madame Bovary} ends up depicting something which clearly is less of a garment and more of an absurd structure, operates by extending the literal understanding of how a cap is constructed, elaborating on the different components of a hat in such a rich and lush manner that it leads the reader to question whether we are still talking about a hat \citep{nabokov_lectures_1980}. This metaphorical composition can be interpreted as standing for the orientalist stance which Flaubert takes vis-à-vis his protagonists, or for the absurdity of material pursuit and ornament, one which ultimately leads the novel's main character, Emma, to her demise, or for the novel itself, whose structure is composed of complex layers, under the guise of banal appearances. Composed metaphors highlight how they exist along \emph{degrees of meanings}, from the conventional and expected to the poetic and enlightening.

We have therefore highlighted how metaphors \emph{function}, and how they can be identified. Another issue they address is that of the \emph{role} they fulfill in our everyday experiences as well as in our aesthetic experiences. Granted a propensity to structure, to adapt, to reason and to induce value judgment, metaphors can ultimately be seen as a means to comprehend the world. By importing structure from the source domain, the metaphor in turn creates cognitive structure in the target domains which compose our lives. Our understanding grasps these structures through their features and attributes, and integrates them as a given, a reified convention—in what Ricoeur would call a \emph{dead} metaphor. This is one of their key contribution: metaphors have a function which goes beyond an exclusive, disinterested, self-referential, artistic role. If metaphors are ornament, it is far from being a crime, because these are ornaments which, in combining imagination and truth, expand our conceptions of the world by making things \emph{fit} in new ways.

\subsection{Literature and cognitive structures}
\label{subsec:literature-cognition}

Building on the focus on conceptual structures hinted at by metaphors, the attention of more recent work has shifted to the relationship between literature (as part of aesthetic work and eliciting aesthetic experiences) and cognition. This move starts from the limitation of explaining "art for art's sake", and inscribing it into the real, lived experiences of everyday life mentioned above, perhaps best illustrated by the question posed in Jean-Marie Schaeffer's eponymous work—\emph{Why fiction?} \citep{schaeffer_pourquoi_1999}. Indeed, if literary and aesthetic criticism are to be rooted in the everyday, and in the conventional conceptual metaphors which structure our lives, our brains seem to be the lowest common denominator in our comprehension of both real facts and literary works \citep{lavocat_interpretation_2015}.

This echoes our discussion in \ref{subsec:knowing-what-how} of Polanyi's work on tacit knowledge, in which the scientist's knowledge is not wholly and absolutely formal and abstracted, but rather embodied, implicit, experiential. This limitation of codified, rigorous language when it comes to communicating knowledge, opens up the door for an investigation of how literature and art can help with this communication, while keeping in mind the essential role of the senses and lived experience in knowledge acquisition (i.e. integration of new conceptual structures) \citep{polanyi_tacit_2009}.

Some of the cognitive benefits of art (pleasure, emotion, or understanding) are not too dis-similar to those posed by Beardsley above, but shift their rationale from strict hermeneutics and criticism to cognitive science. Terence Cave focuses on the latter when he says that literature \emph{"allows us to think things that are difficult to think otherwise}. We now examine such a possibility from two perspectives: in terms of the role of imagination, and in terms of the role of the senses.

Cave posits that literature is an object of knowledge, a creator of knowledge, and that it does so through the interplay between rational thought and imaginative thought, between the "counterfactual imagination" and our daily lives and experiences. Through this tension, this suspension of disbelief is nonetheless accompanied by an epistemic awareness, making fiction reliant on non-fiction, and vice-versa. Working on literary allusions, Ziva Ben-Porat shows that this simultaneous activation of two texts is influenced by several factors. First, the form of the linguistic token itself has a large influence over the understanding of what it alludes to. Its aesthetic manifestation, then, can be said to modulate the conceptual structures which will be acquired by the reader. Second, the context in which the alluding token(s) appears also influences the correct interpretation of such an allusion, and thus the overall understanding of the text. This contextual approach, once again hints at the change of scale that Ricoeur points in his shift from the word to the sentence, and demands that we focus on the whole, rather than single out isolated instances of linguistic beauty. Finally, a third factor is the personal baggage (a personal encyclopedia) brought by the reader. Such a baggage consists of varying experience levels, of quality of the know-how that is to be activated during the reading process, and of the cognitive schemas that readers carry with them. Imagination in literary interpretation, builds on these various aspect, from the very concrete form and choice of the words used, to the unspoken knowledge structures held in the reader's mind, themselves depending on varied experience levels. By allowing the reader to project themselves into potential scenarios, imagination allows us to test out possibilities and crystallize the most useful ones to continue building our conception of the fictional world.

The work of imagination also relies on how the written word can elicit the recall of sensations. This takes place through the re-creation, the evokation of sensory phenomena in linguistic terms, such as the \emph{perceptual modeling} of literary works, which can be defined as (linguistic) simulations relying on the senses to communicate situations, concepts, and potential realities, something at work in the process of creating a fiction. This connects back to the modelling complexities evoked in \ref{subsubsec:modelling-complexity}: both source code and literature have at least the overlap of helping to form mental models in the reader.

This attention to the sense calls for an approach of literary criticism as seen through embodied cognition, starting from the postulate that human cognition is grounded in sensorimotricity, i.e., the ability to feel, perceive, and move. Specifically, pervading cognitive process called perceptual simulation, which is activated when we cognitively process a gesture in a real-life situation, is also recruited when we read about actions, movements, and gestures in texts.

Depiciting movement, vision, tactility and other embodied sensations allows us to crystallize and verify the work of the imaginative process. As such, literature unleashes our imaginary by recreating sensual experiences—Lakoff even goes as far as saying that we can only imagine abstract concepts if we can represent them in space\footnote{Geoff Hinton, pioneer of modern deep-learning, has reportedly said that, to visualize 100-dimensional spaces, one should first visualize a 3-dimensional, and then "shout 100 really really loud, over and over again", cited in \citep{akten_journey_2016}}. It seems that the imaginative process depends in part on visual and spatial projections, and suggests a certain fitness of the conceptual structures depicted. By describing situations which, while fictional, nonetheless are possible in a reality often very similar to the one we live in, it is easy for the reader to connect and understand the point being made by the author. So if literature is an object of knowledge, both sensual and conceptual, offering an interplay between rational and imaginative thought, it still relies on the depiction of mostly familiar situations (the protagonists physiologies, the rules of gravity, the fundamental social norms are rarely challenged).

A first issue that we encounter here, in trying to connect source code and computing to this line of thought, is that source code has close to no perceptible sensual existence, beyond its textual form. In trying to communicate concepts, states and processes related to code and computing, and in being unable to depict them by their own material and sensual properties, we once again resort to linguistic processes which enable the bringing-into-thinking of the program text.

\begin{listing}
    \inputminted[]{java}{./corpus/unhandled_love.java}
    \caption{Unhandled Love, by Daniel Bezera, published in \{code poems\} \citep{bertram_code_2012}}
    \label{code:unhandled-love}
\end{listing}

The code poem listed in \ref{code:unhandled-love} suggests a similar phenomenon when it comes to perceiving motions and sensations through words. The key part of the poem here is the use of the keyword \lstinline{throw}: as a reserved keyword in some of the most popular programming languages, it is known and has been encountered by multiple programmers, as opposed to a word defined in a specific program (such as a variable name). This previous encounters build up a feeling of familiarity and of dread—indeed, the act of the throwing in programming is as dynamic and as violent as in human prose. To throw an object in programming, is to interrupt the smooth execution flow of the program, because something unexpected has happened,—that is, an exception. Additionally, the title of the poem hints at a supplemental implication of the poems motion; any exception that is thrown should be caught, or handled, by another part of the program, in order to gracefully recover from the mishap and proceed as expected. If it's not handled—as is the case in the poem—the program terminates and the source code itself aborts all function.

Vilem Flusser considers poetic thinking as a means to bring concepts into the thinkable, and to crystallize thoughts which are not immediately available to us\footnote{"\emph{In this sense we may say that the intellect expands intuitively. We may, however, define the intuition that results in the production of proper names better, since it is a productive intuition. We may call it “poetic intuition.” The proper names are taken, through this ntuitive activity, from the chaos of becoming in order to be put here (hergestellt), that is, in order to be brought into the intellect.}" \citep{flusser_doubt_2014}}; through various linguistic techniques, poetry allows us to formulate new concepts and ideas, and to shift perspectives. Rendered meaningful via this code poem, a certain conception of love is therefore depicted here as an exception that must be handled (with care) , and the use of a particularly dynamic keyword elicits such a feeling in a reader who previously had to throw and handle exceptions.

Another example of how source code can communicate concepts can be seen in \ref{code:binary-search}. In this case, we can see in the relation between the name of the function, \lstinline{find} and the three local variables \lstinline{high}, \lstinline{low} and \lstinline{probe}, that the act of finding is going to imply some sort of \emph{search space}. The search space is going to be traversed in an alternating way, called the \emph{binary search} in computer science terms\footnote{The author of \ref{code:binary-search} said of the difference between concept and implementation: "\emph{Nothing could be simpler, conceptually, than binary search. You divide your search space in two and see whether you should be looking in the top or bottom half; then you repeat the exercise until done. Instructively, there are a great many ways to code this algorithm incorrectly, and several widely published versions contain bugs.}" \citep{bray_finding_2007}}.

\begin{listing}
    \inputminted[]{java}{./corpus/binary.java}
    \caption{Binary search, implemented by Tim Bray in \emph{Beautiful Code} highlights variable names as an indicator of the spatial component of the function's performance \citep{bray_finding_2007}.}
    \label{code:binary-search}
\end{listing}

Here, we thus have two indicators, syntactical and structural. First, \lstinline{high} and \lstinline{low}, imply the space in-between, a space to be explored via \lstinline{probe}\footnote{Conversely these variables could have been named \lstinline{start}, \lstinline{end} and \lstinline{current}, with similar purpose, but a different denotation}. Second, the use of only two statements inside the \lstinline{while} loop represents the simplicity of the search process itself, a search process which, as \lstinline{(high - low > 1)} tells us, implies a shrinking search space\footnote{Rather than expliciting checking if the target has been found inside the loop, the code's simplicity relies on the fact that another definition for finding is that of reducing search space: "\emph{Some look at my binary-search algorithm and ask why the loop always runs to the end without checking whether it's found the target. In fact, this is the correct behavior; the math is beyond the scope of this chapter, but with a little work, you should be able to get an intuitive feeling for it—and this is the kind of intuition I've observed in some of the great programmers I've worked with. [\dots] You could do the math to figure out when the probability of hitting the target approaches 50 percent, but qualitatively, ask yourself: does it make sense to add extra complexity to each step of an O(log2 N) algorithm when the chances are it will save only a small number of steps at the end?  The take-away lesson is that binary search, done properly, is a two-step process. First,write an efficient loop that positions your low and high bounds properly, then add a simple check to see whether you hit or missed.}" \citep{bray_finding_2007}}.

By paying attention to the spatial and embodied implicit meanings held in the syntactic structures used in both literature and source code, we can start to see how a certain sense of understanding being extracted from reading either kind of texts depends on embodiment. In the case of program texts, the point is to reduce computational space into humanly embodied space; similarly, literature engages in communicating different kinds of space.

\subsection{Words in space}
\label{subsec:spatial-literature}

Beyond the use of metaphor, literature allows the reader to engage cognitively with the world of the work, and the interrelated web of concepts that can then be grasped once they are put into words. This process of putting down intention, through language and into written words, is also the process of transforming a time-based continuum (speech) into a space-based discreete sequence; a process called grammatization, explored further in \citep{bouchardon_valeur_2014}. This is valid both for human prose and machine languages: the unfathomably fast execution of sequential instructions is manifested as static space in source code.

Literary theory also engages with the concept of space. We have seen in the subsection above that there is a particular attention being given to movement in space, through embodied cognition; in that case, the use of a specific syntax can elicit a kinetic reaction in the incarnated reader. We now pay attention to how spatiality interplays with meaning in literature, looking at the spatial form of the text in general, and to spatio(-temporal) markers in the text in specific.

First, we leave behind some traditional concepts in literary theory. We have seen that, due to source code's non-linearity and collaborative aspect, concepts such as narrative and authorship are somewhat complicated to map across fields.

We have mentioned above that the fictionality of a text provides a kind of text-based simulation for a combination of events, characters and situations. While soure code, by its actual execution, might tend to be classified rather as non-fiction, we nonetheless show here that, by evoking interconnected entities, it also participates to the construction of mental models.

Here, we pay particular attention to fictional space: the web of relationships, connotations and suggestions that hint at a broader world than the one immediately at hand in a work of literature. This fictional space, or \emph{storyworld} is not to be equated to what we have denotated as the problem domain. Rather, it is what exists through, yet beyond, the text itself; we refer to it as the \emph{world of reference}.

To focus on the specific tokens denoting space, we rely on the distinction operated by Marie-Laure Ryan on the topic \citep{ryan_space_2009}. The starting point she offers is to consider how the spatial extension of the text, its existence in a certain number of dimensions\footnote{An oral narrative exists in zero dimensions, a live TV news ticker exists in one dimension, a printed or digital page exists in two dimensions, while a theater play exists in three dimensions.} impacts the readers' perception of the narrative.

\begin{listing}
    \inputminted[]{python}{./corpus/spatial_extension.py}
    \caption{This snippet shows how the spatial extension of the text corresponds to the structural semantics of the code, in the Python programming language.}
    \label{code:spatial_extension}
\end{listing}

At the simplest level, we see this illustrated in \ref{code:spatial_extension}. In this listing, we can see how the most direct spatial perceptions of the program text, its indentation, actually represents semantic properties: the indent on \lstinline{class_space} is related to it existing at a different level (scope) than the variables \lstinline{dimensions} and \lstinline{alone}, just like the indent before \lstinline{def __init__} differs from the one before \lstinline{def new_space} also signify changes in lexical scope.

Moving beyond this immediately visual spatial component, Ryan shifts to the spatial form of the text. Rather than looking at the space in which it is deployed, it is considering

\begin{quote}
    a type of narrative organization characteristic of modernism that deemphasizes temporality and causality through compositional devices such as fragmentation, montage of disparate elements, and juxtaposition of parallel plot lines. \citep{ryan_space_2009}.
\end{quote}

Narrative, in its traditional sense of coherent, sequential events whose developments involve plot and characters, is seldom mentioned in writing source code. In source code, narrative is already deemphasized and the spatial form of the text mentioned above is therefore better suited to match the material of the code. Indeed, Ryan continues:

\begin{quote}
    The notion of spatial form can be extended to any kind of design formed by networks of semantic, phonetic or more broadly thematic relations between non-adjacent textual units. When the notion of space refers to a formal pattern, it is taken in a metaphorical sense, since it is not a system of dimensions that determines physical position, but a network of analogical or oppositional relations perceived by the mind. \citep{ryan_space_2009}
\end{quote}

Space, along with interactivity, is a core feature of the digital medium\footnote{As N. Katherine Hayles states in her eponymous essay, \emph{"print is flat, code is deep"} \citep{hayles_print_2004}}. Janet Murray also puts spatiality as one of the core distinguishing features of digital media, at the forefront of which are digital games\footnote{\emph{"The computer's spatial quality is created by the interactive process of navigation. We know that we are in a particular location because when we enter a keyboard or mouse command the (text or graphic) screen display changes appropriately.} \citep{murray_hamlet_1998}}.

An example of this intertwining of flat textual screen and spatial depth is the overall genre of interactive fiction, which displays prompts for textual interaction on a screen, accompanied with the description of where the reader is currently standing in the fictional world. Exploration can only be done in a linear fashion, entering one space at a time; and yet the system reveals itself to contain spaces in multiple dimensions, connected by complex pathways and relationships. The listing in \ref{code:mac_sched} shows how the execution processes of a program text can be expressed spatially in the comments, and then textually in the rest of the file. Since comments are ignored by the computer, this depiction is only to help the human reader in their spatial representation of the executed program.

\begin{listing}
    \inputminted[]{c}{./corpus/mac_sched.c}
    \caption{This listing includes an execution flow diagram inside the program text itself, testifying to the inherently fragmented and non-linear execution of source code. \citep{mustacchi_mac_2019}}
    \label{code:mac_sched}
\end{listing}

As Murray mentions, these features are not limited to those playful interactive systems presented as works to be explored (be it e-literature or digital games), but are rather a core component of digitality. Beyond the realm of fiction, one can see instances of this in the syntax used in both programming languages and programming environments (see \ref{subsec:tools-cognition} and our overview of IDEs). For instance, the use of the \lstinline{GOTO} statement in BASIC, of the \lstinline{JMP} and \lstinline{MOV} instructions in x86 Assembly, or the use of the \lstinline{return} in the C family of programming languages all hint at movement, at going places and coming back, representing the non-linear perception of program execution\footnote{In the meantime, program execution is still considered to be linear by the machine, since instructions are executed one after the other. The use of multi-core architecture and parallel processing does complicate this picture, but programmers rarely engage directly with the specification of which CPU core executes which instruction. What they do engage with, is parallel programming, in which things happen simultaneously, thus presenting cognitive complexity insofar as two processes being run in parallel imply some sort of distinct semantic spaces to be reflected in the mental model of the programmer.}.

And yet, Ryan hints at anothe aspect of spatial form specifically in the digital medium:

\begin{quote}
    But an even more medium-specific type of spatial form resides in the architecture of the underlying code that controls the navigation of the user through a digital text. \citep{ryan_four_2021}
\end{quote}

As writers and readers of this architecture, of which source code is the blueprint, we gather information through syntax about developments in space and time into a cognitive map or mental model of narrative space\footnote{The term \emph{narrative} is used here to describe the effective behaviour of the program, once executed. Since source code appreciation is subject to its function, following the narrative of source code would then amount to following its correct execution path(s), even though \emph{description} fits better to most program texts since, from the machine perspective, it describes exactly what it is doing.}.

Mental maps are therefore dynamically constructed in the course of reading and consulted by the reader to orient herself in the program. A very simple example of spatialization of meaning, both visually and conceptually, can be seen in \ref{code:nested}. There, the spatial component is rendered specifically through the syntax of HTML. HTML, as a markup language, has a specific ontology: it is fundamentally made up of elements who contain other elements, or are self-contained. When an element is contained into another, a specific semantic relationship occurs, where the container influences the contained, and vice-versa. Therefore, what we see at first is layout spatialization, which leads to this specific triangle shape. By using the semantics of the language, in which certain elements can only exist in the context of others, this layout spatialization\footnote{While not functionally necessary, the indents added to the listing further highlight the computational concept of nestedness through visual cues.} also comes to delimit certain semantic areas. This explicitly poetic example takes religion, and the representation of God as its problem domain; its expressive force comes by describing it as both the all-including and the all-included, and thus escaping the implicit rules of everyday spatiality, that a thing cannot contain itself.

\begin{listing}
    \inputminted{html}{./corpus/nested.html}
    \caption{Nested, by Dan Brown and published in \{code poems\} \citep{bertram_code_2012}}
    \label{code:nested}
\end{listing}

A more concrete example can be seen in \ref{code:shutdown}. Written in the style of software engineers, rather than poets, this listing describes a function which gracefully shuts down a HTTP server. Essentitlly, the function \lstinline{Shutdown()} regularly checks if the number of connections to the server is zero. If it reaches zero, it considers the process completed without errors; it waits until it receives an error from the context, or if it receives a tick from a timer setup in advance.

\begin{listing}
    \inputminted[]{go}{./corpus/shutdown.go}
    \caption{This listing represents the various steps taken in order to shutdown a HTTP server, and shows multiple aspects of spatio-temporal complexities \citep{weidideng_caddyserver_2023}}
    \label{code:shutdown}
\end{listing}

The first reference we can look at is mostly spatial, and takes place at the declaration of \lstinline{nextPollInterval}. By being another function declaration, it is both self-contained, but also has access to variables in its declarative environment, such as \lstinline{pollIntervalBase}. A long, dynamic series of statements which double a timer interval everytime it is called is thus compressed into a single token, \lstinline{nextPollInterval}, and can then be passed as an argument to timer functions. Here, the space of the timer interval's calculation is compressed and abstracted away.

Interestingly, we can note the comment \lstinline{// Add 10% jitter}, which explains the calculation of the subsequent interval. The word jitter usually refers to a quicky, jumpy movement, but is here used to facilitate the understanding of adding a random number to the previous one, effectively deviating the timer from its linear increase. Here, using the word \emph{jitter} immediately evokes feeling of small, unpredictable change.

The second reference is primarily temporal. The keyword \lstinline{defer} in the line \lstinline{defer timer.Stop()} specifically marks the deferred execution of this particular function to the specific moment at which the current function (\lstinline{Shutdown()}) returns. This reference is not absolute (as is the timer on the line above, even though it might not be determinate), but rather relative, itself dependent on when the current function will return. Here, the programming language itself makes it simple to express this relative temporal operation.

Finally, we can take a look at both the last \lstinline{select} statement of the function to see a more complex interplay of both space and time. There are two things happening there. With the specific \lstinline{<-} arrow, the pictorial representation shows how a message is incoming, either from \lstinline{ctx.Done()}, which itself comes from outside the current function, given as an argument, or from \lstinline{timer.C}, which comes from the timer that has just been declared in the current function. Both of these messages come from different places, one very distant, and the other very local, and might arrive at different moments. Here, the \lstinline{<-} denotes the movement of an incoming message, expliciting where the messages come from, and in which order they should be treated, and thus facilitates the handling of event with varying spatio-temporal properties.

The listing \ref{code:shutdown} shows not only different spaces of executions, nor only different moments of execution, but very much the intertwining of space and time. One of the earlier approaches to the specific tokens which represent space in the traditional novel has also related it to time: the chronotope is described by Mikhail Bakhtin as the tight entanglement of temporal and spatial relationships that are artistically expressed in literature. Those markers execute a double function, as they allow for the reification of temporal events and spatial settings during the unfolding of narrative events\footnote{"\emph{Time, as it were, thickens, takes on flesh, becomes artistically visible; likewise, space becomes charged and responsive to the movements of time, plot and history.}" \citep{bakhtin_dialogic_1981}}. 

While Bakhtin introduces the concept from a marxist-historical point of view, analyzing notions of history, ideal, epics and folklore through that lense, it is nonetheless useful for our purposes. Chronotopes are a kind of marker which enable the understanding of where something comes from (such as an explicit module declarations in header files, or inline before a function call), or when something should happen (such as the \lstinline{async/await} keyword pair in ECMAscript denoting the synchronicity of an operation or the \lstinline{defer} keyword indicating that a specificied function will only be called \emph{when} the current function returns).

Thus, the chronotopes give flesh to the events described in (and then executed from) source code. As such, they function as the primary means of materializing time in space. From a network of these chronotopes, along with metaphors and other devices that are explicited in \ref{sec:cognitive-aesthetics}, emerges a concretization of representation which the reader can use to constitue a mental model of the program text.

\spacer

Syntactical literary devices allow readers to engage cognitively with a particular content; they enable the construction of mental models a particular narrative, through a network of metaphors, allusions, ambiguous interpretations and markers of space and time. We have shown that these literary devices also apply to source code, especially how the use of machine tokens and human interpretation suggest an aesthetic experience through metaphors, and with particular markers that are needed to make sense of the time and space of a computer program, which differs radically from that of a printed text. This making sense of a foreign time and space is indeed essential in creating a mental map of the storyworld (in fiction) or the world of reference (in non-fiction).

The use of the term map also implies a specific kind of territory, enabled by the digital. As a hybrid between the print's flatness and code's depth, Ryan and Murray—among many others—identify the digital narrative as a highly spatialized one. This feature, Ryan argues, is but a reflection of the inner architecture of source code. Pushing this line of thought further, we now turn to architecture as a discipline to investigate how the built environment elicits understanding, and how such possibilities might translate in the space of program texts.

\section{Architecture and understanding}
\label{sec:arch-understanding}

At its most common denominator, architecture is concerned with the gross structure of a system. At its best, architecture can support the understanding of a system by addressing the same problem as cognitive mapping does: simplifying our ability to grasp large system. This phrase appears in Kevin Lynch's work on \emph{The Image of the City}, in which he highlighted that our understanding of an urban environment relies on combinations of patterns (node, edge, area, limit, landmark) to which personal, imagined identities are ascribed. The process is once again that of abstraction, but goes beyond that, and includes a subjective perspective \citep{lynch_image_1959}. Moving from the urban planner's perspective to the architects, we see how each individual component contributes to the overall legibility of the system. This section considers how individual structures, through their assessed beauty, offer a cognitive involvement to their participants.

Beauty in architecture is one of the discipline's fundamental components, dating back to Vitruvius's maxim that a building should exhibit \emph{firmitas, utilitas, venustas}—solidity, usefulness, beauty. And yet in practice, beauty, or the abillity to elicit an aesthetic experience, is not sufficient, and sometimes not even required, for a building to be considered architectural. Even though architecture is usually considered as an art, it is also a product of engineering, and thus a hybrid field, one where function and publicness modulate what could be otherwise a "pure" aesthetic judgment.

This sections looks at architecture through its multiple aspects, to highlight to which extent some of these are reflected in source code\footnote{Recall how, in \ref{subsubsec:crafting-software}, programmers tended to refer extensively to themselves as architects, engineers or craftspeople.}. Through an investigation of the tensions and overlaps of form, function, context and materiality in the built space, we identify similarities in the programmed space. Particularly, we will look at how an understanding of patterns translates across both domains, in response to both architecture and programming's material constraints, due to the physical instantiation of buildings and programs in a situated context.

\subsection{Form and Function}
\label{subsec:form-function}

Particularly, our interest here is with the cognitive involvement in the architectural work. What is there to be understood in a building, and how do buildings make it intelligible? The early theoretical answers to this question is to be found in the work of Italian architects, such as Andrea Palladio, whose conception of its discipline came from ideal platonic form, and mathematical relation between facade and inner elements, as well as Leon Battista Alberti, whose consideration of beauty in architecture, as such an organization of parts that nothing can be changed without detriment to the whole \citep{scruton_aesthetics_2013}\footnote{Such a definition is a reminiscent of how Vladimir Nabokov defines beauty in literature: "\emph{A really good sentence in prose should be like a good line in poetry, something you cannot change, and just as rhythmic and sonorous}" \citep{nabokov_lectures_1980}}.

While structure is meant to stand the test of time and natural forces\footnote{A purpose exemplified by the still standing structures of Roman and Greek antiquities, resulting from a particular mixing process of concrete.}, utility can be assessed by the extent to which a building fulfills its intended function. How the beauty of a building relates to its function, whether it can be completely dissociated from it, or if it is dependent on the fulfillment of its function, is still a matter of debate between formalists and functionalists. Nonetheless, the position we take here is in line with Parsons and Carlson, in that fitness of an object is a core component of how it is appreciated aesthetically \citep{parsons_functional_2012}, and that form is hardly separable from function.

In some way, then, form should be able to communicate the function of a building. Roger Scruton, in his philosophical investigation of architecture, brings up the question of language—if buildings are to be cognitively engaged with, then one should be able to grasp what they communicate, what they stand for, what they express. To do so, he starts from the fact that architectural works are often composed of interconnected, coherent sub-parts, which then contribute to the whole, in a form of \emph{gestaltung}.

\begin{quote}
    Architecture seems, in fact, to display a kind of 'syntax': the parts of a building seem to be fitted together in such a way that the meaningfulness of the whole will reflect and depend upon the manner of combination of its parts. \citep{scruton_aesthetics_2013}
\end{quote}

Yet, he develops an argumentation which suggests that architecture is not so much articulated as a language, than as a set of conventions and rules, and that it is not a representative medium (which would imply valid and invalid syntax, as well as intent), but rather an expressive one. Architectural significance, then, relies on the presence and arrangement of those evolving conventions—that is, a style—rather than on the depiction of a subject through an exact syntax. While architecture might not represent content the same way literature does, it is nonetheless expressive, and relies on particular styles—recurring formal patterns and ways of doing—to express a tone, a feeling, or a \emph{stimmung} in their dwellers.

As identified in \ref{subsec:beauty-architecture}, the similarities between software and architecture can be mapped as symmetrical approaches: as top-down or bottom-up, from an architect's perspective, or from a craftperson's. Since we focus on what a building expresses, we need to consider the source of such an expression. First, we look at how modernism, and the conventions that make up this architectural thought, are the top-down result of the intersection of function, form and industry, and reveal the influence of functional design on the aesthetic appreciation of a work.

The central modern architectural standard is Louis Sullivan's maxim that \emph{form follows function}, devised as he was constructing the early office buildings in North America. Sullivan's statement is thus that what the building enables its inhabitants to do, inevitably translates into concrete, visible, and sensual consequences.

\begin{quote}
    All things in nature have a shape, that is to say, a form, an outward semblance, that tells us what they are, that distinguishes them from ourselves and from each other
    \dots
    It is the pervading law of all things organic and inorganic, of all things physical and metaphysical, of all things human and all things superhuman, of all true manifestations of the head, of the heart, of the soul, that the life is recognizable in its expression, that form ever follows function. This is the law. \citep{sullivan_tall_1896}
\end{quote}

The value of the building is therefore derived from what it allows the individuals to do: the office building allows them to work, the school to learn, the church to pray and the house to live. To do so, modernist architecture rejects any superfluous decoration, or extraneous addition, as a corruption of the purity of the building's function. In a similar vein, Le Corbusier, another fundamental actor of modern architecture, equates the building with its function, advocating for the suppression of decorative clutter and unnecessary furnishings and possessions, and hailing transparency and simplicity as architectural virtues \citep{lecorbusier_vers_1923}, and culminating in Le Corbusier's assessment that the architectural plan as a generator, and the house as a machine to be lived in.

From this perspective, architectural works are a kind of system, in that they constitute sets of interrelated structural components, where the parts are connected by distinctive structural and behavioral relations; and yet the set of conventions to which Le Corbusier contributes is an abstract representation of this systemic nature. He focuses on the plan as the primary source of architectural quality. For software developers, the equivalent of an architectural plan would be a modelling system such as UML: a language to describe structural relationships between software components, with an example shown in \ref{graphic:uml}. From a modernist angle, the aesthetic value of a building is thus directly dependent on how well it performs an abstractly defined function for its users, assessed at a structural level.

\begin{figure}
    \includegraphics[width=0.8\textwidth,height=\textheight,keepaspectratio,center]{Policy_Admin_Component_Diagram.png}
    \caption{Description of a software component and its inner relations in the Universal Modelling Language, \citep{wikipedia_unified_2023}}
    \label{graphic:uml}
\end{figure}

Just as a two-dimensional floorplan and a three-dimensional building are different, a diagram and a program text are also different. This difference is highlighted throught the process of construction in architecture, and implementation in software development, involving respectively engineers and programmers to realize the work that has been designed by the architect.

It is clear the modernists thought of function as engineering function, and aligned it with engineering aesthetics\footnote{\emph{Esthétique de l'ingénieur} is the title of one of the chapters of Le Corbusier's manifesto, \emph{Vers une Architecture} \citep{lecorbusier_vers_1923}}. Nonetheless, such a conception of function is definitely machinic, consisting of airflow, heat exchange or drainage, expressing a particular feeling of progress and achievement through industrial manufacturing techniques allowing for new material capabilities against contextual understandings. Here again, the human is but a small part in a dynamic system.

Jacques Rancière, in his study of the Werkbund and the Bauhaus-inspired architecture, offers an alternative approach, away from the strcit functionality laid out by Sullivan and Le Corbusier before him. The simplification of forms and processes, he writes of the AEG Turbinenhalle in Berlin, which is normally associated with the reign of the machine, finds itself, on the contrary, related to art, the only thing able to spiritualize industrial work and common life \citep{ranciere_aisthesis_2013}.

By paying attention to the role of a detail, and of the human subjectivity and situatedness of the people inhabiting the building, departs form the strict function of an object or of a building, to its actual use. Such a shift moves the aesthetic judgment from a structure-centric perspective (such as Le Corbusier's ideal dimensions), to a human-centric perspective (such as Lacaton \& Vassal's practical extension of space and light). Peter Downton reiterates this point, when he states that "\emph{buildings and design are often judged from artistic perspectives that bear no relation to how the building’s occupants perceive or occupy the building.}" \citep{downton_knowledge_1998}; his conception of the artistic here, is one that aligns with Kant's definition of a work that is purposive in itself, and not based on a function that it should fulfill.

One can see a translation of such a self-referential conception of art in the class of building which encompass follies and pavillions. These kinds of buildings are constructed first and foremost for their decorative properties, and only secondarily for its structural and functional properties. Follies, for instance, are individual buildings built on the demand of one specific individual's desire. They aim to represent something else than what they are, with no other purpose than ornament and the display of wealth. Pavilions, in the modern acceptation of the term, are rather displays of architectural and engineer prowess, demonstrating the use of new techniques and materials. By focusing only on design and technical feat, it is this prowess itself that is being represented: the function of the building is only to represent the skill of its builders. For instance, Junya Ishigami's pavillion at the Venice Biennale in 2008, shown in \ref{graphic:pavillion} consisted in a very elegant and aerial structure, but whose function was depending on the fact that no living being interacted with it\footnote{Indeed, the structure collapsed due to a cat's playfulness: "\emph{The Barbican says that the 37-year-old Ishigami is "internationally acclaimed", and there is certainly a buzz and fascination around him. Last year he won the Golden Lion, the highest prize at the Venice Architecture Biennale, for a structure that collapsed almost as soon as it was built, following an accident with a cat. Little was left but a scrawled note saying "Scusate, si è rotto. I'm sorry It's broken." } \citep{moore_junya_2011}"}.

\begin{figure}
    \includegraphics[width=0.8\textwidth,height=\textheight,keepaspectratio,center]{ishigami.jpg}
    \caption{Pavillion built by Junya Ishigami + associates, showing a focus on appearance and structural feat, rather than habitability. Picture courtesy of Iwan Baan, 2008.}
    \label{graphic:pavillion}
\end{figure}

As an artform, architecture provides an immersive and systemic physical environment, and thus shapes human psychology and agency within it, in turn forcing the dweller to acknowledge and engage with their environment. This suggests that, from a formal, top-down approach which considers architecture as possessing a systematic language to be realized exactly at a structural level, there exists a complementary, bottom-up approach, centered around human construction and function.

\subsection{Patterns and structures}
\label{subsec:patterns-structures}

A counterpoint to this modernist approach of master planning is that of Christopher Alexander. Along with other city planners in the United States, such as William H. Whythe or Jane Jacobs, Alexander belongs to an empirical tradition of determining what makes a built environment \emph{good} or not, by examining its uses and the feelings it elicits in the people who tread its grounds. He elaborates an approach to architecture which does not exclusively rely on abstract design and technological efficiency, but rather takes into account the multiple layers and factors that go into making

\begin{quote}
    [...] beautiful places, places where you feel yourself, places where you feel alive \citep{alexander_timeless_1979} [...]
\end{quote}

In \emph{The Timeless Way of Building}, he focuses on how beauty is involved in moving from disorganized to organized complexity, a design process which is not, in itself, the essence of beauty, but rather the condition for such beauty to arise. Alexander's conception of beauty, while very present throughout his work, is however not immediately concerned with the specifics of aesthetics, but rather with the existence of such objects. This existence, in turn, does require to be experienced sensually, including visually.

In this process of achieving organized complexity, he highlights the paradoxical interplay between symmetry and asymmetry, and pinpoints beauty as the "deep interlock and ambiguity" of the two, a beauty he also finds the the relationship between static structures of the built environment, and the flow of living individuals in their midst. In his perspective, then, architecture should take into account the role of tension between opposite elements, rather than the combination of rational and abstract design elements. Such an approach echoes other considerations of tension as a source for stimulating human engagement,such as Ricoeur's analysis of the metaphor (see \ref{subsec:literary-metaphors}), and the resolution of the riddles presented in works of obfuscated source code (see \ref{subsec:hackers}).

He therefore considers a possible aesthetic experience as a consequence of qualities such as appropriateness, rightness to fit, not-simplistic and wholeness. All of these have in common the subsequent need for a purpose, a purpose which he calls the \emph{Quality Without a Name} \citep{alexander_timeless_1979}. This quality, he says, is semantically elusive, but nonetheless exists; it is, ultimately, the quality which sustains life, a conclusion which he reached after extensive empirical research: no one can name it precisely, but everyone knows what it refers to. It is the quality which makes one feel at home, which makes one feel like things make sense in a deep, unexplicable way\footnote{"\emph{It is always looking at two entities of some kind and comparing them as to which one has more life. It appears to be a rank bit of subjectivity. [\dots] It turns out that these kind of measurements do correlate with real structural features in the thing and with the presence of life in the thing measured by other methods, so that it isn't just some sort of subjected I groove to this, and I don't groove to that and so on. But it is a way of measuring a real deep condition in the particular things that are being compared or looked at.}" \citep{alexander_keynote_1996}}. This reluctance to being linguistically explicited is echoed in the work of the craftsman, where a practitioner often finds herself showing rather than telling \citep{pye_nature_2008}, another domain with which software developers identify, explicited in \ref{subsubsec:crafting-software}.

Among the adjectives used to circle around this quality are whole, comfortable, free, exact, egoless, eternal \citep{alexander_timeless_1979}. Some of these qualities can also be found in software development, particularly wholeness and comfort. A whole program is a program which is not missing any features, whose encounter (or lack thereof) might cause a crash. If if a function implies a systematic design, such systematic design is not compromised by the lacking of some parts. Conversely, it is also a program which does not have extraneous—useless—features.

A comfortable program text being is a program which might be modified without fear of some unintended side-effects, without inivisible dependencies which might then compromise the whole. There is enough separation of concerns to ensure a somewhat safe working area, in which one can engage in epistemic probing assuming that things will not be breaking in unexpected ways; being whole, it also provides a higher sense of meaning by realizing how one's work relates to the rest of the construction. The implication here is that comfort derives from a certain kind of knowledge, a knowledge of how things (spatial arrangements, technical specifications, human functions) are arranged, how they relate to each other, how they can be used and modified.

To complement this theoretical pendant, Alexander conducted empirical research to find examples of such qualities, in a study led at the University of Berkley which resulted in his most popular book, \emph{A Pattern Language} \citep{alexander_pattern_1977}. In it, he and his team lists 253 patterns which are presented as to form a kind of language, akin to a Chomskian generative grammar, re-usable and extendable in a very concrete way, but without a normative, quasi-biological component. It turns it out that such a documentation, of re-usable configuration and solutions for contextual problem-solving, had a significant echo with computer scientists.

A whole field of research developed around the idea expressed in \emph{A Pattern Language}, at the crossover between computer science and architecture\footnote{See, for instance, the \emph{Beautiful Software Initiative} as an organized effort to develop Alexander's theses on growth, order, artefact and computation \citep{bryant_beautiful_2022}.} of distinct, self-contained but nevertheless composable components. In Alexandrian terms, they are a triad, which expresses a relation between a certain context, a problem, and a solution. Similarly to architectural patterns, these emerged in a bottom-up fashion: individual software developers found that particular ways of writing and organizing code were in fact extensible and reusable solutions to common problems which could be formalized enough to be shared with others. Patterns enable a cognitive engagement based on a feeling of familiarity, and of recognizing affordances.

Extending from the similarities of structure and function between software and architecture mentioned above, it is the lack of learning from practical successes and failures in the field which prompted interest in Alexander's work, along with the development of Object-Oriented Programming, first through the Smalltalk language\footnote{For an extensive history of the design and development of the Smalltalk hardware and software, see Alan Kay's \emph{Early History of Smalltalk} \citep{kay_early_1993}.}, then with C++, until today, as most of the programming languages in 2023 include some sort of object-orientation and encapsulation. What object-orientation does, is that it provides a semantic structure to the program, reflected in the syntactic structure: objects are conceptual entities, with states and actions, as discussed in \ref{subsubsec:modelling-complexity} and shown in \ref{code:representation}. This enables such objects to be re-used within a program text, and even across program texts.

The similarities between a pattern and an object, insofar as they are self-contained solutions to contextual situations that emerged through practice, and resulting from empirical deductions, caught on with software developers as a technical solution with a social inflection, rather than a computational focus. Writing in \emph{Patterns of Software}, with a foreword by Alexander, Richard P. Gabriel addresses this shift from the machine to the human:

\begin{quote}
    The promise of object-oriented programming—and of programming languages themselves—has yet to be fulfilled. That promise is to make plain to computers and to other programmers the communication of the computational intentions of a programmer or a team of programmers, throughout the long and change-plagued life of the program. The failure of programming languages to do this is the result of a variety of failures of some of us as researchers and the rest of us as practitioners to take seriously the needs of people in programming rather than the needs of the computer and the compiler writer. \citep{gabriel_patterns_1998}
\end{quote}

The real issue raised here in programming seems to be, again, not to speak to the machine, but to speak to other humans. The programming paradigm of object-orientation aims at solving such complexity in communication. While understanding software is hard, creating, identifying, and formalizing patterns into re-usable solutions turns out to be at least as hard \citep{taylor_patterns_2001}. Part of this comes from a lack of visibility of code bases (most of them being closed source), but also from the series of various economic and time-sensitive constraints to which developers are subject to (and echoes those in the field of architecture), and which result in moving from making something great to making something good enough to ship. The promise of software patterns seemed to offer a way out by—laboriously—codifying know-how. Interestingly, while the increase in software quality has been found to result from the application of engineering practices \citep{hoare_how_1996}, the discovery and formalization of the software patterns takes place through the format of writers' workshops\footnote{As taken from the website of the 2022 Pattern Languages of Programming conference: "\emph{At PLoP, we focus on improving the written expression of patterns through writers' workshops. You will have opportunities to refine and extend your patterns with the assistance of knowledgeable and sympathetic patterns enthusiasts and to work with others to develop pattern languages.}" \citep{guerra_plop_2022}.}, presenting a different mode of knowledge transmission.

Throughout his work, Gabriel draws from the work of an architect to weave parallels between his experience as a software developer and as a poetry writer, drawing concepts from the latter field into the former, and inspecting it through the lens of a pattern languages of built concrete or abstract structures. We develop further two concepts in particular, and show how \emph{habitability} and \emph{compression} enable an understanding of such structures.

\subsubsection{Compression and habitability in functional structures}
\label{subsubsec:compression-habitability}

We have seen how source code is an inherently spatial medium, with entrypoints, extracted packages, parallel threads of executions, relative folders and directories and endless jump between files. Reading a program text therefore matches more closely an excursion into a foreign territory whose map might be misleading, than reading a book from start to finish. For instance, \ref{graphic:code-city} builds on a longer history of using the city as a metaphor for large code bases, and visualizes classes, packages and version in three dimensions.

\begin{figure}
    \includegraphics[width=0.8\textwidth,height=\textheight,keepaspectratio,center]{codecity_screenshot.png}
    \caption{CodeCity is an integrated environment for software analysis, in which software systems are visualized as interactive, navigable 3D cities. The classes are represented as buildings in the city, while the packages are depicted as the districts in which the buildings reside. \citep{wettel_codecity_2008}}
    \label{graphic:code-city}
\end{figure}

Given this somewhat literal mapping of source code structure onto urban structure, and given the abstract structure of object-oriented code, a reader of source code will need to find their bearings and orient themselves\footnote{"\emph{Exploring a source code repository always starts with finding out what the OS will select as the entry point. 99\% of the time it means finding the `int main(int,char**)` function}" says Fabien Sanglard on the topic of reverse-engineering code-bases \citep{sanglard_game_2018}.}. Once the entrypoint is found, the programmer starts to explore the programmed maze and attempts to make sense of their surroundings, as a step towards the construction of mental models.

Both inhabitants in a building and programmers in a code base have a tendency to be there to \emph{accomplish something}, whether it might be living, working or eating for the former, or fixing, learning or modifying for the latter.s Particularly in software, one of the correlated functions of a program text is to be maintainable; that is, it must be made under the assumption that others will want to modify and extend source code. Other pieces of code might just be satisfying in being read or deciphered (as we've seen in source code poetry in \ref{subsubsec:code-poetry} or with hackers in \ref{subsec:hackers}) but this assumption of interaction with the code brings in another concept, that of \emph{habitability}. In Gabriel's terms, it is

\begin{quote}
    the characteristic of source code that enables programmers, coders, bug-fixers, and people coming to the code later in its life to understand its construction and intentions and to change it comfortably and confidently. \citep{gabriel_patterns_1998}
\end{quote}

In a sense, then, beautiful code is also code that is clear enough to inform action and, well-organized enough to warrant actually taking that action. For instance, writing in the ACM Queue, an anonymous programmer discusses the beauty in a code where the separation between which sections of the source are hardware-dependent and which are not, as seen in \ref{code:hardware-separation}. In that example, it is clear to the programmer what the problem-domain is: counter incrementation, high-performance computation, or a specific Intel chip.

\begin{listing}
    \inputminted{c}{./corpus/hardware_separation.h}
    \caption{This header file defines the structure of a program, both in its human use, in its interaction with hardware components, and its decoupling of hardware and software elements.}
    \label{code:hardware-separation}
\end{listing}

There are several things which we can identify here. First, the three lines at the top of the listing indicate version numbers, which do not hold any computational functionality, but rather a human functionality: it communicates that this software considers change and evolution as core part of its source code, inviting the programmer reader to further modify it\footnote{From the anonymous programmer: "\emph{The engineer clearly knew his software would be modified not only by himself but also by others, and he has specifically allowed for that by having major, minor, and patch version numbers. Simple? Yes. Found often? No.}" \citep{vicious_beautiful_2008}.}

Second, the line defining the types of CPUs supported by the software is written in human-intelligible way, rather than a cryptic hexadecimal notation\footnote{"\emph{Nothing is more frustrating when working on a piece of software than having to remember yet another stupid, usually hex, constant. I am not impressed by programmers who can remember they numbered things from 0x100 and that 0x105 happens to be significant. Who cares? I don’t. What I want is code that uses descriptive names. Also note the constants in the code aren’t very long, but are just long enough to make it easy to know in the code which chip we’re talking about.}" \citep{vicious_beautiful_2008}.}. While the CPUs are ultimately represented in hexadecimal notation, the effort is made to render things intelligible to and quickly retrievable from the programmer's memory.

Finally, the \lstinline{struct pmc_mdep} is a shorthand notation for "machine-dependent". In an era in which software can theoretically be executed on different hardware architectures, it is welcome to make the difference between the variables themselves, which apply across platform, and the values of these variables, which need to be changed per platform\footnote{"\emph{It would seem obvious that you want to separate the bits of data that are specific to a certain type of CPU or device from data that is independent, but what seems obvious is rarely done in practice. The fact that the engineer thought about which bits should go where indicates a high level of quality in the code.}" \citep{vicious_beautiful_2008}.}. This is a good example of a separation of concerns: it is made clear which parts of the program text the programmer needs to pay attention to, and can change, and which parts of the program texts she needs not be concerned with. For a further example of separation of concerns, one could point a beautiful commit is a commit which adds a significant feature, and yet only change the lines of the code that are within well-defined boundaries (e.g. a single function), leaving the rest of the codebase untouched, and yet affecting it in a fundamental way.

Habitability, then, is a combination of acknowledgment by the writer(s) to the reader(s) of the source, by referring to the evolution over time of the software, along with the use of intelligible names and separation of concerns. This distinction relates to Alexander's property of comfort, by affording involvement instead of estrangement. Still, such a feature of habitability, of supporting life, doesn't specify at all what it could, or should, look like. Rather, we get from Alexander a negative definition:

\begin{quote}
    The details of a building cannot be made alive when they are made from modular parts\dots And for the same reason, the details of a building cannot be made alive when they are drawn at a drawing board. \citep{alexander_timeless_1979}
\end{quote}

If modularity itself is at odds with making good (software) constructions, then its implementation under the terms of an object-oriented programming paradigm becomes complicated. Indeed, the technical formalization of the field came with the release of the \emph{Design Patterns: Elements of Reusable Object-Oriented Software} book, which lists 23 design patterns implementable in software \citep{gamma_design_1994}. Its influence, in terms of copies sold, and in terms of papers, conferences and working groups created in its wake, is undeniable, with Alexander himself giving a keynote address at the ACM two years after the release. It has, however, been met with some criticism.

Some of this criticism is that patterns are "external", they look like they come from somewhere else, and are not adapted to the code. In this sense, this corroborates Alexander in being wary of constructions which do not integrate fully within their environments, which do not, in an organic sense, allow for a piecemeal growth\footnote{Addressing this concern, the failure of strict top-down hierarchies in software development resulted in the \emph{agile} methodology for business teams, now one of the most popular ways of building software products.}. If patterns express relations between contexts, problems and solutions, then it seems that one of the main complaints of developers is that they might, one day, look at the code they were working on and see chunks of foreign snippets dumped in the middle to fix some generic problem, with no understanding for specifics, nor fit in the existing structure. This is judged negatively due to its lack of understanding of context offered by those proposed solutions. In this, blindly applying patterns from a textbook might be a solution, but it's not an elegant one. This criticism also finds its echoes in the Scruton's analysis of architectural styles; rules and conventions, while present in architecture, are often adopted only to be departed from—re-interpreted and adapted to the context of the building \citep{scruton_aesthetics_2013}.

One aspect that has been eluded so far is therefore that of the programming languages used by the programmer. Indeed, one doesn't write Ruby like one writes Java,  C++, or Lisp. If materiality is a core component of eliciting an aesthetic experience in an architectural context, then programming languages are the material of source code, and offer a specific context to the writing and reading of the program text.

A final criticism to software patterns is that they are language-independent. As such, they are often workarounds for features that a particular programming language doesn't allow from the get-go, or offers simpler implementations than the pattern's\footnote{For instance, Peter Norvig highlights that most patterns in the original book have much simpler implementations in Lisp than in C++ or Smalltalk \citep{norvig_design_1998}}.

While patterns might operate at a more structural level, hinting at different parts of code, and its overall organization, one can also turn to a more micro-level. What can a detail do in our understanding of structures? Sometimes decried, sometimes praised in architecture, the detail fulfills mutliple roles: acting as a meaningful interface, compressing meaning and testifying for materiality.

Both Scruton and Rancière mention the detail as an essential architectural element. Without contributing to the structural soundness of the construction, it nonetheless contributes to its expressiveness. A blend of the cognitive and sensual is also characteristic of Scruton's "imaginative perception", understood as the perception of the details of built structures, and their extrapolation into the imaginary. Indeed, the experience of the user is based on the points at which it sensually grasps its environment: the detail is therefore the point of interaction between the human and the structure. This imagination depends on the interpretative choices in parsing ambiguous or multiform aspects of the built environment. The detail contributes to the stylistic convention of the creation:

\begin{quote}
    Convention, by limiting choice, makes it possible to 'read' the meaning in the choices that are made \dots for style is used to 'root' the meanings which are suggested to the aesthetic understanding, to attach them to the appearance from which they are derived. \citep{scruton_aesthetics_2013}
\end{quote}

With many external constraints, due to both context and function, the architect or builder does not have much room for personal expression, and it is through details that their intent and their style are being shown. The significance of a detail can be in explaining which conventions the structure adopts, as well as communicating the intent of the creator. A significant detail manages to compress meaning into a restricted physical surface.

Compression is a concept introduced by Gabriel in response to pattern design. In narrative and poetic text, it is the process through which a word is given additional meaning by the rest of the sentence. In a sentence such as "\emph{Last night I dreamt I went to Manderley again.}" \citep{dumaurier_rebecca_1938}, the reader is unlikely to be familiar with the exact meaning of \emph{Manderley}, since this is the first sentence of the novel. However, we can infer some of the properties of Manderley from the rest of the sentence: it is most likely a place, and it most likely had something to do with the narrator's past, since it is being returned to. A similar phenomenon happens in source code, in which the meaning of a particular expression or statement can be derived from itself, or from a larger context. In object-oriented programming, the process of inheritance across classes allows for the meaning of a particular subclass to be mostly defined in terms of the fields and methods of its subclasses—its meaning is compressed by relying on a semantic environment, which might or not be immediately visible.

This, Gabriel says, induces a tension between extendability (to create a new subclass, one must only extend the parent, and only add the differentiating aspects) and context-awareness (one has to keep in mind the whole chain of properties in order to know exactly what the definition of an interface that is being extended really is). Resolving such a tension, by including enough information to hint at the context, while not over-reaching into idiosyncracy, is a thin line of being self-explanatory without being verbose.

For instance, Casey Muratori explores the process of compression in refactoring a program text, first by distinguishing semantic compression from syntactic compression\footnote{"\emph{Like, literally, pretend you were a really great version of PKZip, running continuously on your code, looking for ways to make it (semantically) smaller. And just to be clear, I mean semantically smaller, as in less duplicated or similar code, not physically smaller, as in less text, although the two often go hand-in-hand.}" \citep{muratori_semantic_2014}}, and then honing in on what makes a compression successful\footnote{"\emph{Ah! It's like a breath of fresh air compared to the original, isn't it? Look at how nice that looks! It's getting close to the minimum amount of information necessary to actually define the unique UI of the movement panel, which is how we know we're doing a good job of compressing.} \citep{muratori_semantic_2014}}. Transitioning from uncompressed code, shown in \ref{code:uncompressed} to compressed code, shown in\ref{code:compressed}, allows the programmer to understand broad patterns about the overall architecture of the program text—here, the function is to display a clickable panel on a user interface.

% - Listing 44: again, the point of these examples as an illustration of refactoring is also not clear, should be commented a lot more
\begin{listing}
    \inputminted[]{c}{./corpus/uncompressed.c}
    \caption{genalloc.c, Basic general purpose allocator for managing special purpose memory from the Linux Kernel, displaying examples of source-code spatiality \citep{muratori_semantic_2014}}
    \label{code:uncompressed}
\end{listing}

The difference we can see between the compressed and uncompressed goes beyond the number of lines used for the same functionality. A first clue in terms of semantics is the use of strictly syntactic block markers: \{ and \}. There are here stricly to delimitate a code block, with no semantic meaning to the computer. While the uncompressed listing shows all the separate elements needed for a button to exist (such as \lstinline{x0}, \lstinline{y0}, \lstinline{my_height}, etc.), while the compressed listings as encapsulated them into an object called \lstinline{Panel_Layout}, thus abstracting away from the programmer's mind the details of such a panel. This encapsulation then enables a further compression of the program: by adding the \lstinline{push_button()} method on the \lstinline{layout}, the compressed code realizes the same functionality of checking for button presses for each button, but ties it to a specific object and, due to the implementation, includes the name of the button being pressed on the same line as the check happens, rather than a line apart in the uncompressed example.

\begin{listing}
    \inputminted[]{c}{./corpus/compressed.c}
    \caption{genalloc.c, Basic general purpose allocator for managing special purpose memory from the Linux Kernel, displaying examples of source-code spatiality \citep{muratori_semantic_2014}}
    \label{code:compressed}
\end{listing}

By compressing the source code and abstracting some concepts, such as the button, one can also gain understanding about the rest of the program text. By showing details of practices and styles, a programmer can extrapolate from a small fragment to a larger structure. Gabriel calls this idea \emph{locality}: it is

\begin{quote}
    that characteristic of source code that enables a programmer to understand that source by looking at only a small portion of it. \citep{gabriel_patterns_1998}
\end{quote}

In poetry, compression presents a different problem since, ultimately, the definitions of each words are not limited to the poet's own mind but also exist in the broad conceptual structures which readers hold. However, since all aspects of a program is always explicitly defined, programmers thus have the ultimate say on the definition of most of the data and functions described in code. As such, they create their own semantic contexts while, at the same time, having to take into account the context of the machine, the context of the problem, and the context(s) that their reader(s) might be coming from.

We now see that, within the same need for the appreciation of function, architecture can take opposite approaches: seeing a building as an abstract design, or as a concrete construction. In his 1951 lecture, "Building, Dwelling, Thinking", Martin Heidegger offers a perspective on these two forms of architecture. He equates top-down and bottom-up to, respectively, building as erecting, and building as cultivating. Ultimately, both of these approaches relate to human dwelling in a given location. To dwell is an engagement of thought and of action, one which leads to the construction of buildings in particular locations, arguing for a contextual adequacy of human structures to their environment\footnote{Speaking of a farmhouse in the Schwarzwald, he describes the chain of creation as such: \emph{" A craft which, itself sprung from dwelling, still uses its tools and frames as things, built the farmhouse.}} \citep{heidegger_building_1975}. This active existence in time and space, allowing for deliberate thought and action and resulting in a better structure also equates to Gabriel's concept of habitability:

\begin{quote}
    Habitability is the characteristic of source code that enables programmers coming to the code later in its life to understand its construction and intentions and to change it comfortably and confidently \dots Software needs to be habitable because it always has to change \dots What is important is that it be easy for programmers to come up to speed with the code, to be able to navigate through it effectively, to be able to understand what changes to make, and to be able to make them safely and correctly. \citep{gabriel_patterns_1998}
\end{quote}

As Heidegger returns to the etymological root of dwelling (\emph{bauern}) in order to connect it to the possible experience of the world humans can have through language, he grounds our experience in context. His though, between earth, man, \emph{techne} and construction, hints at the essence that human construction—craft—as a consequence of thought and as a precedence to construction. Taking into account context and materiality, a final connection between software and architecture is actually with the field that predated, and complemented, architecture: craftsmanship.

\subsection{Material knowledge}
\label{subsec:material-knoweldge}

Architecture as a field and the architect as a role have been solidified during the Renaissance, consecrating a separation of abstract design and concrete work. This shift obfuscates the figure of the craftsman, who is relegated to the role of executioner, until the arrival of civil engineering and blueprints overwhelmingly formalized the discipline \citep{pevsner_term_1942}. While computer science, through its abstract designs, echoes the modernist architect with its pure plans, the programmer, identifying itself with the craftsman, offers different avenues for knowing artefacts.

The architect emerged from centuries of hands-on work, while the computer scientist (formerly known as mathematician in the 1940s and 1950s) was first to a whole field of practitioners as programmers, followed by a need to regulate and structure those practices. Different sequences of events, perhaps, but nonetheless mirroring each other. On one side, construction work without an explicit architect, under the supervision of bishops and clerks, did indeed result in significant achievement, such as Notre Dame de Paris or the Sienna Cathedral. On the other side, letting go of structured and restricted modes of working characterizing computer programming up to the 1980s resulted in a comparison described in the aptly-named \emph{The Cathedral and the Bazaar}. This essay described the Linux project, the open-source philosophy it propelled into the limelight, and how the quantity of self-motivated workers without rigid working structures (which is not to say without clear designs) can result in better work than if made by a few, select, highly-skilled individuals \citep{raymond_cathedral_2001,henningsen_joys_2020}.

What we see, then, is a similar result: individuals can cooperate on a long-term basis out of intrinsic motivation, and without clear, individual ownership of the result; a parallel seen in the similar concepts of \emph{collective craftsmanship} in the Middle-Ages and the \emph{egoless programming} of today \citep{brooksjr_mythical_1975}. Building complex structures through horizontal networks and practical knowledge is therefore possible, with consequences in terms aesthetic apprecitations.

Craftsmanship in our contemporary discourse seems most tied to a retrospective approach: it is often qualified as that which was \emph{before} manufacture, and the mechanical automation of production \citep{thompson_study_1934}, preferring materials and context to technological automation. Following Sennett's work on craftsmanship as a cultural practice, we will use his definition of craftsmanship as \emph{hand-held, tool-based, intrinsically-motivated work which produces functional artefacts, and in the process of which is held the possibility for unique mistakes} \citep{sennett_craftsman_2009}.

At the heart of knowledge transmission and acquisition of the craftsman stands the \emph{practice}, and inherent in the practice is the \emph{good practice}, the one leading to a beautiful result. The existence of an aesthetic experience of code, and the adjectives used to qualify it (smelly, spaghetti, muddy), pointed at in \ref{subsec:lexical-fields}, already hints at an appreciation of code beyond its formalisms and rationalisms, and towards its materiality.

A traditional perspective is that motor skills, with dexterity, care and experience, are an essential feature of a craftsman's ability to realize something beautiful \citep{osborne_aesthetic_1977}, along with self-assigned standards of quality \citep{pye_nature_2008,sennett_craftsman_2009}. These qualitative standards which, when pushed to their extreme, result in a craftsperson's \emph{style}, gained through practice and experience, rather than by explicit measurements \citep{pye_nature_2008} \footnote{See Pye's account of craftsmanship, and his intent to make explicit the question of quality craftsmanship and \emph{"answer factually rather than with a series of emotive noises such as protagonists of craftsmanship have too often made instead of answering it."} \citep{pye_nature_2008}}. Two things are concerned here, supporting the final result: tools and materials \citep{pye_nature_2008}. According to Pye, a craftsperson should have a deep, implicit knowledge of both, what they use to manipulate (chisels, hammers, ovens, etc.) as well as what they manipulate (stone, wood, steel, etc).

The knowledge that the craftsman derives, while being tacit (see \ref{subsec:knowing-what-how}), is directed at its tools, its materials, and the function ascribed to the artefact being constructed, and such knowledge is derived from a direct engagement with the first two, and a constant relation to the third. Finally, any aesthetic decoration is here to attest to the care and engagement of the individual in what is being constructed—its dwelling, in Heideggerian terms.

This relationship to tools and materials is expected to have a relationship to \emph{the hand}, and at first seems to exclude the keyboard-based practice of programming. But even within a world in which automated machines have replaced hand-held tools, Osborne writes:

\begin{quote}
    In modern machine production judgement, experience, ingenuity, dexterity, artistry, skill are all concentrated in the programming before actual production starts. \citep{osborne_aesthetic_1977}
\end{quote}

He opens here up a solution to the paradox of the hand-made and the computer-automated, as programming emerges from the latter as a new skill. This very rise of automation has been criticized for the rise of a Osborne's "soulless society" \citep{osborne_aesthetic_1977}, and has triggered debates about authorship, creativity and humanity at the cross-roads between artificial intelligence and artistic practice \citep{mazzone_art_2019}. One avenue out of this debate is human-machine cooperation, first envisioned by Licklider and proposed throughout the development of Human-Computer Interaction \citep{licklider_mancomputer_1960,grudin_tool_2016}. If machines, more and more driven by computing systems, have replaced traditional craftsmanship's skills and dexterity, this replacement can nonetheless suggest programming as a distinctly 21st-century craftsmanship, as well as other forms of cratsmanship-based work in an information economy.

Beautiful code, code well-written, is an integral part of software craftsmanship \citep{oram_beautiful_2007}. More than just function for itself, code among programmers is held to certain standards which turn out to hold another relationship with traditional craftsmanship—specifically, a different take on form following function.

A craftsman's material consciousness is recognized by the anthropomorphic qualities ascribed by the craftsman to the material \citep{sennett_craftsman_2009}, the personalities and qualities that are being ascribed to it beyond the immediate one it posseses. Clean code, elegant code, are indicators not just of the awareness of code as a raw material that should be worked on, but also of the necessities for code to exist in a social world, echoing Scruton's analysis that architectural aesthetics cannot be decoupled from a social sense\footnote{"\emph{it is the aesthetic sense which can transform the architetct's task from the blind pursuit of an uncomprehended function into a true exercise of practical common sense.}" \citep{scruton_aesthetics_2013}}. As software craftsmen assemble in loose hierarchies to construct software, the aesthetic standard is \emph{the respect of others}, as mentioned in computer science textbooks \citep{abelson_structure_1979}.

Another unique feature of software craftsmanship is its blending between tools and material: code, indeed, is both. This is, for instance, represented at its extreme by languages like LISP, in which functions and data are treated in the same way \citep{mccarthy_lisp_1965}. In that sense, source code is a material which can be almost seamlessly converted from information to information-\emph{processing}, and vice-versa; code as a material is perhaps the only non-finite material that craftspeople can work with—along with words\footnote{This disregards the impact of programming languages, the hardware they run on, and the data they process on the environment; see \citep{kurp_green_2008}}.

Code, from the perspective of craft, is not just an overarching, theoretical concept which can only be reckoned with in the abstract, but also the very material foundation from which the reality of software craftsmanship evolves. An analysis of computing phenomena, from software studies to platform studies, should therefore take into account the close relationship to their material that software developers can have. As Fred Brooks put it,

\begin{quote}
    The programmer, like the poet, works only slightly removed from pure thought-stuff. He builds his castles in the air, from air, creating by exertion of the imagination. Few media of creation are so flexible, so easy to polish and rework, so readily capable of realizing grand conceptual structures. \citep{brooksjr_mythical_1975}
\end{quote}

So while there are arguments for developing a more rigorous, engineering conception of software development \citep{ensmenger_computer_2012}, a crafts ethos based on a materiality of code holds some implications both for programmers and for society at large: engagement with code-as-material opens up possibilities for the acknowledgement of a different moral standard\footnote{Writing about resilient web development, Jeremy Keith echoes this need for material honesty: "\emph{The world of architecture has accrued its own set of design values over the years. One of those values is the principle of material honesty. One material should not be used as a substitute for another. Otherwise the end result is deceptive \citep{keith_resilient_2016}. }"}. As Pye puts it,

\begin{quote}
    [\dots] the quality of the result is clear evidence of competence and assurance, and it is an ingredient of civilization to be continually faced with that evidence, even if it is taken for granted and unremarked. \citep{pye_nature_2008}
\end{quote}

Code well-done is a display of excellence, in a discipline in which excellence has not been made explicit. If most commentators on the history of craftsmanship lament the disappearance of a better, long-gone way of doing things, before computers came to automate everything, locating software as a contemporary iteration of the age-old ethos of craftsmanship nonetheless situates it in a longer tradition of intuitive, concrete creation.

\spacersmall

To conclude this section, we have seen that architecture can offer us some heuristics when looking for aesthetic features which code can exhibit. Starting from the naïve understanding that form should follow function, we've examined how Alexander's theory of patterns, and its significant influence on the programming community\footnote{This theory has even spawned short-lived debates about his quality without a name on stackoverflow \citep{interstar_quality_2017}.}, points not just to an explicit conditioning of form to its function (in which case we would all write hand-made Assembly code), but rather to an elusive, yet present quality, which is both problem- and context-dependent.

Along with the function of the program as an essential component of aesthetic judgment, our inquiry has also shown that program texts can present a quality that is aware of the context that the writer and reader bring with them, and of the context that it provides them, making it habitable. Software architecture and patterns are not, however, explicitly praised for their beauty, perhaps because they disregard these contexts, since they are higher-level abstractions; this implies that generic solutions are rarely elegant solutions. And yet, there is an undeniable connection between the beautiful and the universal. Departing from our investigation of software as craftsmanship, and moving through towards a more abstract discipline, we turn to scientific aesthetics.

\section{Forms of scientific activity}
\label{sec:aesthetic-scientific}

As programmers learned their craft from practice and immediate engagement with their material, computer science was concomittantly developing from a seemingly more abstract discipline. Mathematicians such as Alan Turing, John Von Neumann and Grace Hopper can be seen, not just as the foreparents of the discipline of computing, but also as standing on the shoulders of a long tradition of mathematicians. Computation is one of the many branches of contemporary mathematics and, as it turns out, this discipline also has reccuringly included references to aesthetics. After the metaphors of literature and the patterned structures of architecture, we conclude our analysis of the aesthetic relation of domains contingent to source code by looking at how mathematics integrate formal presentation.

This section approaches the topic of aesthetics and mathematics in three different steps. First, we look at the objective or status of beauty in mathematics: are mathematical objects eliciting an aesthetic experience in and of themselves, or do they rely on the observer's perception? Considering the difference between abstract objects and their representation: is aesthetic representation ascribed to either, or to both? And what is the place of the observer in this judgment? Having  established a particular focus on the representations of abstract objects, we then turn to the epistemic value of aesthetics, and how positive aesthetic representations in mathematics can enable insight and understanding. Finally, we complement this relation between knowledge and presentation and depart from the ends of a proof, and an evaluative appraisal of aesthetics in mathematics, by investigating the actual process of doing mathematics, concluding with topics of pedagogy and ethics.

\subsection{Beauty in mathematics}
\label{subsec:beauty-mathematics}

The object of mathematics is to deal first and foremost with abstract entities, such as the circle, the number zero or the derivative, which can find their applications in fields like engineering, physics or computer science. Because of this historical separation from the practical world through the use and development of symbols, one of the dominant discourses in the field tended to consider mathematical beauty as something intrisic to itself, and independent from time, culture, observer, or representation itself. Indeed, a circle remains a circle in any culture, and its aesthetic properties—uniformity, symmetry—do not, at first glance, seem to be changing across time or space.

According to the Western tradition, mathematics are perhaps the first art. Aristotle, in his \emph{Metaphysics}, wrote of beauty and mathematics as the former being most purely represented by the latter, through properties such as order, symmetry and definiteness\footnote{"\emph{the supreme forms of beauty are order, symmetry, and definiteness, which the mathematical sciences demonstrate in a special degree. And since these (e.g. order and definiteness) are obviously causes of many things, evidently these sciences must treat this sort of causative principles also (i.e. the beautiful) as in some sense a cause.}" \citep{aristotle_metaphysics_2006}}. By offering insight into the harmonious arrangement of parts, it was thought that mathematics could, through beauty, provide knowledge of the nature of things, resulting in an understanding of how things generally fit together. Beauty then naturally emerges from mathematics, and mathematics can, in turn, provide an example of beauty. At this intersection, it also becomes a source of intellectual pleasure, since gaining mathematical knowledge exercises the human being's best power—that of the mind.

Arguing for this position of objective quality being revealed through beautiful manifestation, Godfrey H. Hardy writes, in his \emph{Mathematician's Apology}, that beauty is constitutive of the objects that compose the field; their abstract quality is what removes them from the contextuality of human judgment.

\begin{quote}
    A mathematician, like a painter or a poet, is a maker of patterns. If his patterns are more permanent than theirs, it is because they are made with \emph{ideas}. A painter makes patterns with shapes and colours, a poet with words.
    \dots
    The mathematician's patterns, like the painter's or the poet's must be \emph{beautiful}; the ideas like the colours or the words, must fit together in a harmonious way. Beauty is the first test: there is no permanent place for ugly mathematics. \citep{hardy_mathematician_2012}
\end{quote}

Here, Hartman posits that it is the arrangement of ideas that possess aesthetic value, and not the arrangement of the representation of ideas. In this, he follows the position of other influential mathematicians, such as Poincaré \citep{poincare_science_1908}, or Dirac \citep{kragh_paul_2002}, who rely on beauty as a property of the mathematical object in itself. For instance, Dirac states that a physical law must necessarily stem from a beautiful mathematical theory, thus asserting that the epistemic content of the theory and its aesthetic judgment thereof are inseparable; a good mathematical theory is therefore intrinsically beautiful. Summing up these positions, Carlo Cellucci establishes proportion, order, symmetry, definiteness, harmony, unexpectedness, inevitablity, economy, simplicity, specificity,  and integrations as the different properties inherent to mathematical objects, as mentioned from an essentialist perspective \citep{cellucci_mathematical_2015}. Ironically, this rather seems to hint at the multiplicity of appreciations of beauty within mathematics, with mathematicians concurring on the existence of beauty, but not agreeing on what kind of beauty pertains to mathematics. Nonetheless, they do agree that beauty is connected to understanding and epistemic acquisition. John Von Neumann, writing in 1947, states that:

\begin{quote}
    One expects a mathematical theorem or a mathematical theory not only to describe and to classify in a simple and elegant way numerous and a priori disparate special cases. One also expects "'elegance" in its "architectural," structural makeup. Ease in stating the problem, great difficulty in getting hold of it and in all attempts at approaching it, then again some very surprising twist by which the approach, or some part of the approach, becomes easy, etc\dots \citep{vonneumann_mathematician_1947}
\end{quote}

The point that Von Neuman makes here is a difference between the content of the mathematical object and its structural form. Such a structural form, by organizing the connection of separate parts into a meaningful whole, makes it easy to grasp the problem. In this sense, it is both the crux of aesthetics and the crux of understanding.

Similarly, François Le Lionnais, a founding member of the Oulipo literary movement in postwar France, wrote an essay on the aesthetic of mathematics, paying attention to both the mathematical objects in and of themselves, such as e or π, but also to mathematical methods, and how they compare to traditional artistic domains such as classicism or romanticism. Without getting into the intricacies of this argumentation, we can nonetheless note that his descriptions of mathematical beauty find echoes in source code beauty. For instance, his appraisal of the proof by recurrence\footnote{"\emph{It seems to us that a method earns the epithet of classic when it permits the attainment of powerful effects by moderate means. A proof by recurrence is one such method. What wonderful power this procedure possesses! In one leap it can move to the end of a chain of conclusions composed of an infinite number of links, with the same ease and the same infailliability as would enter into deriving the conclusion in a trite three-part syllogism."}" \citep{lelionnais_great_1971}} reflects similar lines of praise given by programmers to the elegance of recursive functions, which are sharing the same mathematical device (for instance, see \ref{code:recursion_iteration_csharp} and \ref{code:scheme_interpreter} for examples of recursion as an aesthetic property). A proof by recurrence is indeed a kind of structure, which can be adapted to demonstrate different kinds of mathematical objects.

To understand is to grasp how each elements fits with others within a greater structure (either in a poem, a symphony or a theorem), with some or all of these elements being rendered sensible to the observer \citep{cellucci_mathematical_2015}. The beauty of a mathematical object can then be ascribed in its display of the definite relation between its elements. For instance, the equation representing Euler's identity (see \ref{graphic:euler}) demonstrates the relation between geometry, algebra and numerical analysis through a restrained set of syntactic symbols, where e is Euler's number, the base of natural logarithms, i is the imaginary unit, which by definition satisfies i\^{2} =-1, and π is the ratio of the circumference of a circle to its diameter. Each of the symbols is necessary, definite, and establishes clear relations between each other, revealing a deep interlock of simplicity within complexity.

\begin{figure}
    \includegraphics[width=0.8\textwidth,height=\textheight,keepaspectratio,center]{euler.png}
    \caption{Euler's identity demonstrates the relation between geometry, algebra and numerical analysis through a restrained set of syntactic symbols.}
    \label{graphic:euler}
\end{figure}

There is also empirical grounding for such a statement. This equation ranked first in a column in the \emph{Mathematician Intelligencer} about the beauty of mathematical objects; the columnist, David Wells, had asked readers to rank given theorems, on a linear scale from 0 to 10, according to how beautiful they were considered \citep{wells_are_1990}\footnote{Along with, for instance, the infinite prime theorem, and Fermat's "two squares" theorem.}. Again, while this assessment does show that there can be consensus, and thus some aspect of objectivity, in a mathematician's judgment of beauty in a mathematical object, it also showed that mathematical beauty also depends on the observer, since mathematicians provided varying accounts.

Rather than focusing on the beauty of the mathematical entities themselves, then, another perspective is to consider beauty to be found in the \emph{representation} of mathematical , since conceptual entities can only graspable through their manifestation.

A first approach is to consider that that the beauty ascribed to mathematics and the beauty ascribed to mathematical representation are unrelated. This disjunctive view, that aesthetics and mathematics can be decoupled (e.g. there can be ugly proofs of insightful theorems, and elegant proofs of boring theorems), was first touched upon by Kant. As Starikova highlights, the philosopher operates a distinction between perceptual, disinterested beauty, and intellectual, vested beauty. Perceptual beauty, the one which can be found in the visual representations of mathematical entities, is the only beauty graspable, while intellectual beauty, that of the objects themselves, simply does not exist, "mathematics by itself being nothing but rules" \citep{starikova_aesthetic_2018}.

Such manifestation of perceptual beauty, connected to mathematical entities themselves, can nonetheless be found in the phenomenon of re-proving in existing proofs, in order to make them more beautiful. Rota, for instance, associates the beauty of a piece of mathematics with the shortness of its proof, as well as with the knowledge of the existence of other, clumsier proofs\footnote{"\emph{"The beauty of a piece of mathematics is frequently associated with shortness of statement or of proof.}" and \emph{"A proof is viewed as beautiful only after one is made aware of previous, clumsier proofs.}" \citep{rota_phenomenology_1997}} \citep{rota_phenomenology_1997}. Thus, it is not so much the content of the proof itself, nor the abstract mathematical object being proven that is the focus of aesthetic attention, but rather the process of establishing the epistemic validity of such an object.

What is useful here is technique, the demonstration of the knowledge from the prover to the observer, through the proof. As such, the asssessment of aesthetics in mathematics, both as a producer and as an observer, depends on the expertise of each individual, and on the previous knowledge that this indivual has of mathematics\footnote{"\emph{Mathematical creation is not so free, hence the contrasting analogy of the landscape gardener, who needs a good grasp of the topography before getting down to creating something beautiful which needs to be based on that topography.}" \citep{thomas_beauty_2017}} (an assessment of the aesthetics of mathematics for non-expert is discussed in \ref{subsec:aesthetics-heuristics} below). It seems that the way that the mathematical object is presented does matter for the assessment.

If beauty is not intrinsic to the mathematical object, but rather connected to the representation of the mathematician's knowledge, there remains the question of why is beauty taken into account in the doing of mathematics. Looking at the lexical fields used by mathematicians to qualify their aesthetic experience, as reported in \citep{inglis_beauty_2015} provides us with a clue: amongst the most used terms are "ingenious", "striking", "inspired", "creative", "beautiful", "profound" and "elegant". Some of these terms have a connection to the epistemic: for instance, something ingenious enables previously unseen connections between concepts, implying the resourcefulness and the cleverness of the originator of the idea. The next question is therefore that of the relationship between the aesthetic and the epistemic in mathematics; and in how this relation can manifest itself in source code.

\subsection{Epistemic value of aesthetics}
\label{subsec:epistemic-aesthetics}

Caroline Jullien offers an alternative to the perception of mathematics as an autotelic aesthetic object, by retracing the definitions of beauty given by Aristotle and establishing a cognitive connection through a cross-reading of the \emph{Metaphysics} and the \emph{Poetics}, highlighting that "\emph{the characteristics of beauty are thus useful properties that yield an optimal perception of the object they apply to. [...] Men can understand what is ordered, measured and delineated far better than what is chaotic, without clear boundaries, etc.}" \citep{jullien_languages_2012}. She then develops this point further, building on Poincaré's assessment of mathematical entities which fulfill aesthetic requirements and are, at the same time, an assistance towards understanding the whole of the mathematical object presented. Aesthetics, then, might not exist exclusively as intrinsic properties of a mathematical object, but rather as an epistemic device.

Her argument focuses on considering mathematics as a language of art in the Goodmanian sense of the term, investigating how mathematical notation relates to Goodman's criteria of syntactic density, semantic repleteness, semantic density, exemplification and multiple references \citep{jullien_languages_2012}. She shows that, while mathematical notation might not seem to satisfy all criteria (for instance, syntactic density is only fulfilled if one takes into account graphical representations), a mathematical reasoning can present symptoms of the aesthetic, particularly through the ability to exemplify and refer to abstract entities.

However, to do that, she also includes different representations of mathematical systems, beyond typographical characters. Taking into account diagrams and graphs, it becomes easier to see how a more traditionally artistic representation of mathematics is possible. The thickness of a line, the color-coding or the spatial relationship can all express a particular class of mathematical objects; for instance, the commutative property in arithmetic can be represented in geometry through the aesthetic property of symmetry. In this work, we focus on the textual representation of source code, eluding any graph or diagram (such as the one we've seen in architectural descriptions of software systems in \ref{graphic:uml}). Nonetheless, we have argued in \ref{sec:aesthetic-cognition} that source code qualifies as a language of art: while the syntactic repleteness does not match that of, say, painting, the unlimited typographical combinations, paired with the artificial design of programming languages as working medium enables the kinds of subtle distinctions necessary for symptoms of the aesthetic to be present.

Following Jullien, if a piece of mathematics is eliciting an aesthetic experience, or presenting positive aesthetic properties, it might then be a support for the understanding by the viewer of this very piece of mathematics. Such a support is itself manifested in this ability to show a harmonious correspondence of parts in relation to a whole. A beautiful presentation is a cognitively encouraging presentation. The subsequent question then regards the nature of that understanding: if it does not happen as an instant stroke of enlightenment, how does it take place as a gradual process of deciphering \citep{rota_phenomenology_1997}?

Addressing this question, Carlo Celluci hints at the concept of fitness, meaning the appropriateness of a symbol in its denotation of a concept, and the appropriateness of concepts in their demonstration of a theorem. Only through this dual level can fitness enable understanding rather than explanation \citep{cellucci_mathematical_2015}. This gradual conception of understanding fits the context of proofs and demonstration; when confronted with source code—that is, with the result of a thought process of one or more programmers—the processual conception of understanding seems to find its limits.

To illustrate the relation between presentation and understanding of defined conceptual entities, we can look at how the linked list, a data structure that is fundamental in computer science, can be represented in an elegant way. The linked list allows for the retrieval and manipulation of connected items, as well as for the re-arrangement of the list itself, and thus holds within it thoughtful implications in terms of organizing and accessing sequential data. To do so, each item on the list contains both its value, and the address of the next item on the list, except for the last item, which points to \lstinline{null}; a last component, the \lstinline{head} points to the current element of the list. A graphical representation is provided in \ref{graphic:linked-list}.

\begin{figure}
    \includegraphics[width=0.8\textwidth,height=\textheight,keepaspectratio,center]{linked_list.png}
    \caption{The linked list is an abstract data structure which acts as a fundamental conceptual entitiy in computer science. It is here represented as a graph, and implementations can be seen in \ref{code:linked_list} and \ref{code:linked_list_remove}.}
    \label{graphic:linked-list}
\end{figure}

The linked list implementation shown in \ref{code:linked_list} establishes a source code representation of such data structure. This comparison between a graphical representation and a textual one highlights that the graphical representation is not only artistic in the traditional sense of the term, but rather that it  operates different expressive choices, and calls for attention on different parts of the same concept (e.g. the \lstinline{head} of the list). On the other side, the textual representation also makes attentional choices, but to different aspects (here, the structure of the \lstinline{list_item}); in this case, it seems less explanatory than the graphical representation, and limited in communicating why this is a canonical example of such a computational entity, or how did one reach this conclusion among other possible entities. The preference of graphical demosntrations over textual ones might indeed rely on the fact that our visual perception is the most developed of our system, and that our reasoning takes place through the manipulation of visual cues \citep{wallen_form_1990}.

\begin{listing}
    \inputminted{c}{./corpus/linked_list.h}
    \caption{A textbook example of a fundamental construct in computer science, the linked list. This header file shows all the parts which compose the concept \citep{kirchner_linked_2022}.}
    \label{code:linked_list}
\end{listing}

Looking at \ref{code:linked_list}, one can view the different relationships between parts and wholes: the list item composing the list itself, the head pointer being a specific instance of the next pointer, and the different methods to access or modify the list itself. Seeing all of these together suggest an understanding of the whole through the parts which is nowhere explicitly described but everywhere suggested. On the contrary, the diagram provided in \ref{graphic:linked-list} provides a much more immediate understanding of how the linked list is structured. As such, its aesthetic properties (spacing, weight, color) contribute to highlighting the smae parts as defined in \ref{code:linked_list}.

Rather than in the static description of the structure, we can look at the operations which can be performed on it in order to suggest implicit qualities of the object at hand. The linked list example (see \ref{code:linked_list}) might be considered aesthetically pleasing only at a particular level of skill. However, once we start manipulate the concept, we can grasp its underlying properties. In \ref{code:linked_list_remove}, we have reproduced two functions which perform the same operation: given a list and an element, they remove the element from the list.

The distinction is made clear via the function names between a beginner level (\lstinline{remove_cs101}, labelled "CS101" for the course number of introduction to computer science) and an expert level (\lstinline{remove_elegant}). In the first one, we see two main statements, \lstinline{while} and \lstinline{if}. The first statment looks for the element to be removed by looping over the list. Once it has found it, it hands it over the next statement, which checks for the particular edge case where the target element might be the first one, which needs to be handled differently than for all other cases. In this case, the representation of this operation does not quite reach into the generic.

The second function is more complex, yet more enlightening. In order to get rid of the particular edge case which does not symbolize the universality of such a procedure, the author\footnote{The author of this particular elegant re-write is Linus Torvalds, orinal author of the Linux kernel \citep{torvalds_linus_2016}.} makes a heavy use of the pointer notation, which allows a program to refer to either the content of a variable, or to the address at which the content is stored.

This use of pointers implies a change in the mental model when considering a linked list. While the first implementation operates on \lstinline{list_item} elements, and then individually deals with the sub parts (such as the \lstinline{next} field), the second implementation considers the list mostly as a series of pointers, thus reducing the conceptual overload, and increasing the functional efficiency at the price of initially more cryptic notation.

\begin{listing}
    \inputminted{c}{./corpus/linked_list.c}
    \caption{A comparison of how to remove an element from a list, with elegance depending on the skill level of the author \citep{kirchner_linked_2022a}.}
    \label{code:linked_list_remove}
\end{listing}

Subtle notational changes can therefore flip the representation of a conceptual entity. Rather than being separated from purpose (in the case of mathematics, that function being proving a theorem), aesthetics is integrated into it by facilitating the understanding of the connection between, and the reasoning for mental or computational operations. For instance, writing about elegant mathematical proofs, John Barker argues that aesthetics are involved in implicit understanding:

\begin{quote}
    Grasping a proof, understanding its gist, seeing why it works, is an important further step, and an essential step if one is to become a competent mathematician. However, by simply following each move in a proof, one has learned everything that is explicitly stated in the proof. Therefore, in really understanding a proof, one must be learning something that is not explicitly stated in it. \citep{barker_mathematical_2009}
\end{quote}

Still, whether an aesthetic judgment relies on perceived qualities, or if it only relies on the quality of an idea, is still up for debate. For instance, Starikova that "\emph{[A]lthough visual representations are involved and understanding does rely on them, it is clearly non-perceptual beauty that initiates aesthetic judgment}" \citep{starikova_aesthetic_2018}, pointing back to the distinction above as to whether beauty is perceived as intrinsic to the mathematical object, or intrinsic to its representation.

Here, we argue that, when it comes to source code, adequate representation of the idea is necessary in order to elicit an aesthetic experience, following our conception of understanding through an embodied lens. However, aesthetic judgments also depend on the nature of background knowledge that the reader holds when engaging with a program text. As we saw in \ref{code:linked_list_remove}, a beginner might appreciate the conceptual beauty of the linked list, while an expert might appreciate the beautiful representation of the linked list.

On one side, the lack of pre-existing knowledge involves the deciphering of symbols and thus immediate attention to form. On the other side, the pre-existence of knowledge allows one to focus on the quality and details of the presentation, such as when mathematicians decide to find more beautiful proofs to an existing theorem. In this case, the knowledge of the theorem, and how its intellectually-perceived simplicity can be translated into a sensually-perceived simplicity and an aesthetic judgment on the form. Here, the aesthetic judgment precedes the intellectual judgment, all the while not guaranteeing a positive intellectual judgment (e.g. the abstract object, whether program function or mathematical theorem, is presented in an aestheticlly-pleasing manner, but remains shallow, boring, non-sensical or wrong).

We argue here that both intellectual pleasure and aesthetic pleasure happen in a dialogic fashion, considering the symbols and the meanings reciprocally, until intellectual and aesthetic judgments have been given. This is in line with Rota's critique of the term "enlightenment" or "insight" in his phenomenological account of beauty in mathematics. The process of discovery and understanding is a much longer one than a simple stroke of genius experienced by the receiver \citep{rota_phenomenology_1997}.

An aesthetic experience in mathematics involves uncovering the connections between aesthetic and epistemic value being represented through a mathematical symbol system. However, such a conception seems to take place as a gradual process of discovery, both from the writer and from the reader, rather than intrinsic aesthetic value existing in a given mathematical concept. Seen in the light of skill-based aesthetic judgment, this chronological unfolding points towards a final aspect of aesthetics in mathematics specifically, and in the sciences in general: aesthetics as heuristics for knowledge acquisition.

\subsection{Aesthetics as heuristics}
\label{subsec:aesthetics-heuristics}

So far, we had been looking at how aesthetics are evaluated in a finished state—that is, once the form of the object (whether a proof or a program text) has stabilized. In doing so, we have left aside a significant aspect of the matter. Aesthetics in mathematics do not need to be seen exclusively as an end, but also as a mean, as a part of the cognitive process engaged to achieve a result. As such, we will see how they also serve as a useful heuristic, both from a personal and social perspectives. Since the ultimate purpose of mathematics specifically, and scientific activities in general, is the establishment of truths, one can only follow that beauty has but a secondary role to play—though that is not to say superfluous.

Complementing the opinions of mathematicians at the turn of the 20\^{th} century, Nathalie Sinclair offers a typology of the multiple roles that beauty plays in mathematics. Beyond the one that we have investigated in the previous sections, which she calls the \emph{evaluative} role of beauty, in determining the epistemic value of a conceptual object, she also proposes to look at a \emph{generative} role and at a \emph{motivational} role \citep{sinclair_aesthetic_2011}. The latter helps the mathematician direct their attention to worthy problems—something which is of limited equivalence in source code, since programming mostly derives from external constraints. The former holds a guiding role during the inquiry itself, once the domain of inquiry has been chosen. It helps in generating new ideas and insights as one works through a problem. This aesthetic sense can be productive both in its positive evaluations—implying one might be treading a fruitful path—as well as negative—hinting that something might not be conceptually well-formed because it is not formally well-presented\footnote{"\emph{The realization that we recognize problems through our anti-aesthetic response to them provides an important clue as to how we go about defining the nature of the problem and recognize its solution. The nature of the disjuncture between our aesthetic expectations and what we observe or think we know reveals the detailed characteristics of the specific problem that presents itself.}" \citep{root-bernstein_aesthetic_2002}}. According to Root-Bernstein, the informal insights of aesthetic intuition precede formal logic. Only when we explicitly recognize that the “tools of thinking” and the “tools of communication” are distinct can we understand the intimate, yet tenuous, connection between thought and language, imagination and logic \citep{root-bernstein_aesthetic_2002}.

This is echoed in Norbert Wiener's perception of aesthetics in mathematics as a way to structure a knowledge that is still in the process of being formed, in order to optimize short-term memory as the mental model of the conceptual object being grasped is still being built\footnote{"\emph{The mathematician's power to operate with temporary emotional symbols and to organize out of them a semipermanent, recallable language. If one is not able to do this, one is likely to find that his ideas evaporate due to the sheer difficulty of preserving them in an as yet unformulated shape.}" quoted by Sinclair in \citep{sinclair_roles_2004}}. This description of a sort of landmark item, in the geographical sense, echoes the role of beacons described by Détienne \citep{detienne_software_2001} and mentioned in \ref{subsec:psychology-programming}, One can therefore consider an aesthetically pleasing element to serve as a sort of beacon used by programmers to construct a mental representation of the program text they are reading or writing.

A representation does not need to be of an effective proof in order to be nonetheless considered functional. A sketch of a concept might even evoke more in certain readers than a fully detailed implementation, offering a direction into which further fruitful inquiry.

For instance, the listing \ref{code:regex} shows such a sketch, as it represents the essential components of a regular expression matcher, as featured in the \emph{Beautiful Code} edited volume. Regular expressions are a form of linguistic formal pattern that serve as an input to a regular expression matcher in order to find particular patterns of text in an input string. Given a input text file, a regular expression matcher could find a pattern such as "\emph{any consecutive list of characters, starting with any number of alphanumeric characters, followed by a dot, followed by at least one character and at most seven characters }"—in essence a rough description of a file name and extension. Building such a system is not a trivial endeavour.

In this case, the essential components of the matcher are implemented, in a clear and concise way. It highlights the overall structure (a general \lstinline{match} function, with \lstinline{matchhere} and \lstinline{matchstar} handling separate cases), the process of looping over an input string, the fundamentals of handling different patterns, and within those the fundamentals of handling different characters in relation to the current pattern. Each part is clearly delineated (and fit for its separate purposes) and contributes to an understanding of the whole, by limiting itself to displaying its essence.

\begin{listing}
    \inputminted{c}{./corpus/regex.c}
    \caption{A regular expression matcher by Rob Pike, praised for its elegance and conciseness, but not for its practical utility \citep{oram_beautiful_2007}}
    \label{code:regex}
\end{listing}

It must be pointed out here that the regular expression is functional at its core: in less than 50 lines, it performs the core operations of the system it represents, while a fully-functional implementation, such as the one in Python's \lstinline{re} module, is more than 2000 lines \citep{secretlabsab_parser_2001}. The beauty found by Brian Kernighan in this program text is that the core of the idea is represented elegantly, while leaving avenues for exploration to the reader\foonote{Brian Kernighan concludes his analysis of this piece of code as such: "\emph{I don’t know of another piece of code that does so much in so few lines while providing such a rich source of insight and further ideas.}" \citep{kernighan_regular_2007}}.

Mathematics, like source code, therefore pay close attention to how formal presentation facilitates the cognitive grasping of abstract concepts. Reducing and organizing literal tokens into conceptually coherent units, and meaningful relations to other units—for instance, having the code in \ref{code:regex} reversed, with the \lstinline{match()} function at the bottom of the document would represent a different level of importance of that entrypoint function, which would complicate the understanding of how the source code functions.

One of the virtues of the listing is that it is particularly beneficial to students, helping them grasp the important parts without being overloaded with too many technical details. Nathalie Sinclair further develops the importance of aesthetically pleasing textual objects representing mathematical concepts in order to facilitate learning. She positions her argument as a response to the strict focus of the studies in mathematics on the perceptions and reports of highly successful individuals. If individuals like Poincaré, Hardy and Dirac can self-report their experiences, she inquiries into the ability for individuals of a different skill level to experience generative aesthetic experiences, experiences where the encountering of an aesthetic symptom generates new directions for ideas. In a subsequent work, she describes the perception of mathematics students as such:

\begin{quote}
    The aesthetic capacity of the student relates to her sensibility in combining information and imagination when making purposeful decisions, regarding meaning and pleasure. \citep{sinclair_aesthetic_2011}
\end{quote}

Emotion and intellect are no longer antitheses, and can be reported by students as well. From her investigations, then, it seems that the heuristic value works across skill levels, whether one holds a Fields medal or a high-school degree. Doing similar work, Seymour Papert aimed at evaluating the functional role of aesthetics by documenting a group of non-experts working through a proof that the square root of 2 is an irrational number. After a series of transformative steps, the subjects of the study manged to eliminate the square root symbol by elevating the two other variables to the power of two\foonote{Incidentally, this process of elevating to a power to get rid of a square root is the same heuristic used in the highly-optimized piece of hacker code calculating the inverse square root of a number, listed in \ref{code:fast_sqrt_c} and discussed in \ref{subsec:hackers}}, as in \ref{graphic:irrational-proof}.

\begin{figure}
    \includegraphics[width=0.8\textwidth,height=\textheight,keepaspectratio,center]{irrational.png}
    \caption{Steps of transformation to approach an epistemic value in finding whether or not the square root of 2 is an rational number.}
    \label{graphic:irrational-proof}
\end{figure}

Papert conceptualizes such an observation as a phase of play, a phase of playing which is aesthetic insofar as the person doing mathematics is delimitating an area of exploration, qualitatively trying to fit things together, and seeking patterns that connect or integrate \citep{papert_mathematical_1978}, and thus looking to identify parts which would seem to fit a yet-to-be-determined whole. This also seems to confirm the perspective that there are some structures that are meaningul to the mathematician.

An interesting aspect of this conception of aesthetics by both Papert and Sinclair is their temporal component. While, for evaluative aesthetics, one can grasp the formal representation of the mathematical object in one sweep, this generative role hints at a more important need to develop over time. This opens up a new similarity with source code, by shifting from the reader to the writer. On one side, Sinclair connects this unfolding over time with Dewey's theory of inquiry and with Polanyi's personal knowledge theory, connecting further the psychological perception with the role of aesthetics. Both Dewey and Polanyi offer a conception of knowledge creation which relies particularly on a step-by step development rather than immediate enlightenment \citep{polanyi_knowing_1969,sinclair_roles_2004}; it is precisely this distinction which Papert addresses with his comparison of aesthetics as \emph{gestalt} (evaluative) or \emph{sequential} (generative).

Taking from Dewey's proposal of what an aesthetic experience is\footnote{Dewey presents it as having first and foremost a temporal structure, something that is dynamic, because it takes a certain time to complete, time to overcome obstacles and accumulate sense perceptions and knowledge, following a certain direction, a teleology hopefully concluding in a certain sense of pleasure and fulfillment. \citep{leddy_dewey_2021}}, we can connect it back to a sequential aesthetic perception in Papert's term, one of learning and discovery, but also to the practice of writing good source code.

In programming practice, the process of working through the establishment of a valuable epistemic object through the sequential change of representations is called \emph{refactoring}. As described by Martin Fowler, author of an eponymous book, refactoring consists in improving the textual design of an existing program text, while retaining an identical function. The crux of the process lies in applying a series of small syntactical transformations, each of which help to sharpen the fitness of the parts to which these transformations are applied. Ultimately, the cumulative effect of each of these syntactical transformations ends up being significant in terms of program comprehension, bringing it closer to a sense of elegance \citep{fowler_refactoring_1999}\footnote{We have described an instance of this process in \ref{subsubsec:compression-habitability}, with a starting point in \ref{code:verbose_c}, and a conclusion in \ref{code:verbose_refactored_c}}.

Finally, extending from this personal and psychological perspective on the development of epistemic value through the pursuit of aesthetic perceptions, we can note a final dimension to aesthetics in mathematics: a social component. Shifting our attention away from the modes of mathematical inquiry of individual mathematician, Sinclair and Primm have highlighted the practices of the community as a whole, including how truths are named, manipulated and negotiated. \citep{sinclair_aesthetic_2011}.

On one side, this amounts to uncovering the fact that scientific problems are being decided upon and researched based on particular values and conventions, conventions which then trickle down into the presentation of results, highlighting trends and social formations both in terms of content of research and style of research \citep{depaz_stylistique_2023}. The interpretation provided by Pimm and Sinclair is that aesthetics, through "good taste" subtly reify a power relationship and exclude practitioners by delimiting what is proper writing and proper research \citep{sinclair_many_2010}.

While one could argue for a similar power dynamic when it comes to programming style, one notable difference we see with programming is the highly interactive collaborative environment in which the productive work can be done. Particularly in the case of software engineering, the fact that a given program text is being worked on by different individuals of different skill levels and at different moments also suggests a social function of aesthetics as a means to harmonize social processes. The evaluative posture of the reader in giving a positive value judgment on a given program text or mathematical proof also implies that this positive judgment was given as a generative role; that is, the aesthetic symptoms are made visible by a writer in search of elegant function and epistemic communication, and appreciated by the reader as an indicator of a work well-done \citep{tomov_role_2016a}.

This implies a certain sense of care that was being given to the program text, or to the mathematical proof, which in turn suggests a certain functional quality in the finished object. Beautiful mathematics, as beautiful code, can therefore be seen as a sign that someone cared for others to understand it clearly.

Aesthetics, then, complement more traditional notions of scientific thinking, from representing a mathematical object, enabling access to the conceptual nature and implications of this object, as well as providing useful heuristics in establishing a new object. What remains, and what will be taken up in the next chapter, is to "\emph{reify this meta-logic as a set of rules, axioms, or practices.}" \citep{root-bernstein_aesthetic_2002}, by establishing which mathematical approaches fit with source code aesthetics.

\spacer

In this chapter, we have established a more thorough connection between aesthetics and cognition. First at the philosophical level, we established how source code fits within Nelson Goodman's conception of what is a language of art, before complementing this ability for an aesthetic experience to communicate complex concepts with more contemporary research, including contribution from cognitive sciences.

We then moved to more specific domains, examining both how their aesthetic properties engage with cognition, but also how these might relate to those held by source code. Looking at literature, we paid attention to how metaphors, embodied cognition and spatial representations are all devices allowing for the evokation of complex world spaces and cultural references, facilitating the comprehension of (electronic) poetry and prose.

Turning to architecture, we acknowledged the role of function in the conception of modernist aesthetics, one which focuses on the plan rather than on the building, before contrasting this approach with the theories of Christopher Alexander. His concepts of patterns and habitability have been widely transposed in programming practice, highlighting a tension between top-down, abstract design, with bottom-up, hands-on engagement. This notion of direct material engagement led us to further examine how craft folds ties to architecture, and how it facilitates a particular kind of knowledge production and value judgment.

Finally, turning to mathematics, we distinguished two main approaches: an evaluative aesthetics, where the representation of a mathematical object has an epistemic function, and a generative aesthetics, which works as a heuristic from a writer's perspective, and often remains unseen to the reader, as it is presented in its final form, without the multiple steps of formal transformations that led to the result.

Throughout, we compared how these specific aesthetic approaches related to source code. Since source code is presented by programmers as existing along these domains of practice, this has allowed us to further refine a specific aesthetics of source code. The next chapter brings the concepts identified in these domains into a dialogue in order to constitute a coherent view. To do so, we will start from source code's material: programming languages.
