\chapter{Conclusion} %40k

The point of this work was to show two main things: that aesthetics facilitate understanding in a functional context.

The emphasis is on the \emph{functional}, in that there is something that a software should achieve, but there is also something that aesthetics should achieve: i.e. the communication of complex ideas.

The emphasis is also on the \emph{context}, in that, while there are conventions that have emerged, and psychological studies that have confirmed that particular kinds of layout and presentation are beneficial to program understanding, aesthetics are also situated, depending on the relation with the program, the machine, and the audience of the program, as well as the intent of the use.

We've also shown that, due to software's ambiguous nature as an \emph{abstract artifact}, a variety of domains are summoned to make sense of software beauty, each connecting the surface-level to the deep-level in their own way.

\section{Function} %9k

\section{Context} %9k

\section{Abstract artifact} %9k

\section{Opening into poetics} %9k