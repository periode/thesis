\chapter{Conclusion} %40k

The point of this work was to show that aesthetics can be used to facilitate understanding in a functional context.

The emphasis is on the \emph{functional}, in that there is something that a software should achieve, but there is also something that aesthetics should achieve: i.e. the communication of complex ideas.

The emphasis is also on the \emph{context}, in that, while there are conventions that have emerged, and psychological studies that have confirmed that particular kinds of layout and presentation are beneficial to program understanding, aesthetics are also situated, depending on the relation with the program, the machine, and the audience of the program, as well as the intent of the use.

We've also shown that, due to software's ambiguous nature as an \emph{abstract artifact}, a variety of domains are summoned to make sense of software beauty, each connecting the surface-level to the deep-level in their own way.

In the end, this allows us to think through the concept of aesthetics: not exclusively as an end to all things, but as means to represent \emph{something} to \emph{someone}, meaning that it acts as an interface between a concept, an idea, a person, and is presented to another person. Aesthetics, in this sense, have a clear communicative role. This clear communicative role, before it is even located within a particular environment (a turing machine), implies some sort of success. A successful communication is a communication which is correctly interpreted (here, the interpretation is, at minima, what the program does, and what the program intents to do, things that might not always be aligned).

We get a scale for exercising a value judgment: how much does this help/deter understanding?

\section{Abstract artifact} %9k

\section{Function} %9k

\section{Context} %9k

And convention (style)

And technology (tools, idiomaticity)

\section{Opening into poetics} %9k