\chapter{Conclusion} %30k - 5-7% of total

%
% BITCH BE HUMBLE
%
% this is supposed to specify what we've found, why it's valuable, how it can be applied, what is new and exciting
% his will show how it has made a valid, and useful, contribution to knowledge.

% refresh the memory with the questions laid out in the introduction - 2k
A piece of source code, as the lexical representation of computational processes, themselves representations of worldly matters, is an ambiguous object. Such an object exists at the overlap of both human and machine comprehension, and is often hidden away by the executed processes of which it is the source. And yet, source code practioners, those who write and read code, agree on the existence of a certain sense of \emph{beauty} in source code.

The research aims of this thesis were to highlight the origins and implications of the aesthetic properties exhibited by varieties of source code. That is, we intended to show how aesthetic properties relate to other aesthetic fields, and what role they played in the existence and purpose of source code—with a particular focus on its relationship to function. Our methodology started from an empirical approach of looking at specific instances of source code, and on analyzing of the discourses surrounding and commenting these instances. From this initial study, we identified several lexical fields that programmers refer to when they evoke the aesthetic properties of source code—literature, architecture, mathematics and craft.

Along with this first research axis, we also noted how the aesthetic judgement in source code is closely tied to its fuctionality. Indeed, any aesthetic value is dependent on the correct behaviour of the source code; ugly code is related to its apparent bugginess and difficulty in engaging with it, while beautiful code implies that the actions resulting from the source code are conform to what the programmer had intended, along with being presented in the best possible way\footnote{Such a definition of a \emph{best possible way} is dependent on the social, cultural and economic spheres within which the code is produced.}. The aesthetic properties of source code are therefore those of a semantic representation of computational space-time, whose purpose is the effortless communication of the operations of the computer, the intention of the programmer(s) and the representation of the world.

The rest of this conclusion will address each of our initial research questions' findings, followed by an examination of the limitations and contributions to existing research on source code.

\subsubsection{What does source code have to say about itself?} %5k

Starting from trade literature on the topic, and complementing it by cases of close-reading, we've highlighted both structural and contextual specificities to source code.

% 2k section on the structural properties (conceptual distance, conceptual symmetry, double-meaning (and that this is almost a __TRIPLE-MEANING__ since there is linguistic, mechanic and problem domain))

The key finding is that code is a matter with which we build/represent spaces

% 1k section on idiomaticity, communities of language meeting as communities of practices, also style and tools
Code also supports multiple practices (tools, idiomaticity, craft, style)

% 2k section on how the code is also contextual
And those multiple practices are themselves reflective on varied social contexts.

That it's not single-faceted. We've identified that there isn't one abstract programmer, nor one abstract source code. Code is written in various contexts, for various purposes, and therefore takes various forms.

\subsubsection{How does source code relate to other aesthetic fields?} % 5k

As varieties of spaces. Rather than being opposites of one another, each reference contributes to the purpose of source code aesthetics by clarifying the structure of the code at multiple levels and dimensions.

Literary aesthetics facilitate the comprehension of the scope of variables and of the intent of the programmer in relation with the problem domain.

Architectural aesthetics establish patterns of familiarity and expectations for better navigation of the source code space.

Mathematical aesthetics enable the condensation of knowledge and insight in the least amount of tokens, minimizing noise, and related to poetic condensation.

All in all, those domains are metaphors in the vivid sense of the term, following Ricoeur: they are actively used, not in a punctual fashion, but in a more generic, sustained manner.

It's also interesting to flip it over, and think of how these external aesthetics have in relation to code: structure, consistency, but mostly cognition! There is a need for communicating invisible concepts/aspects which is present in all of those fields.

\subsubsection{How do the aesthetics of source code relate to its functionality?} %5k

We've shown that aesthetics can be used to facilitate understanding in a functional context. More specifically, the aesthetic properties of code are derived from a conception of code as a semantic material which in terms is assembled, and apprehended, as a spatial construct, rather than a strictly literary, mathematical, or architectural one.

This connects back to the approaches of Goodman on the expressive power of symbol systems.

We could also introduce the concept of effective aesthetics, according to which an aesthetic property can be judged as to whether it adequately represents a given concept, behaviour or intent\footnote{These are indeed three different things, but they all involve some sort of invisibility.}. But it also proposes that there is a certain level of expertise needed to appreciate a work of art, and that the novice cannot be expected to provide an informed aesthetic value judgment.

% bigger picture
bigger picture: we've shown that the formal properties of a specific medium have a relationship to episteme.

The emphasis is on the \emph{functional}, in that there is something that a software should achieve, but there is also something that aesthetics should achieve: i.e. the communication of complex ideas.

The emphasis is also on the \emph{context}, in that, while there are conventions that have emerged, and psychological studies that have confirmed that particular kinds of layout and presentation are beneficial to program understanding, aesthetics are also situated, depending on the relation with the program, the machine, and the audience of the program, as well as the intent of the use.

We've also shown that, due to software's ambiguous nature as an \emph{abstract artifact}, a variety of domains are summoned to make sense of software beauty, each connecting the surface-level to the deep-level in their own way.

In the end, this allows us to think through the concept of aesthetics: not exclusively as an end to all things, but as means to represent \emph{something} to \emph{someone}, meaning that it acts as an interface between a concept, an idea, a person, and is presented to another person. Aesthetics, in this sense, have a clear communicative role. This clear communicative role, before it is even located within a particular environment (a turing machine), implies some sort of success. A successful communication is a communication which is correctly interpreted (here, the interpretation is, at minima, what the program does, and what the program intents to do, things that might not always be aligned).

We get a scale for exercising a value judgement: how much does this help/deter understanding?

% contribution to the field
\section{Contribution}  %6k

I gathered a larger corpus of source code examples.

I offered a typology of how to exert aesthetic properties in code.
On a more theoretical level, I connected it to epistemology and the arts.
I proposed an explanation about code as matter, existing in semantic space

QUESTION: Reflecting on the gaps in existing research, relation to existing theories, inscribed in a framework.

Validated theories
\begin{itemize}
    \item hayles for her print surface, code deep
    \item paloque-berges for her double-meaning
    \item pineiro for his instrumental aspect
\end{itemize}

Invalidated theories
\begin{itemize}
    \item cramer, magic, it's not magic, it's skill (but that reflects to Alfred Gell's technology of enchantment)
    \item oram, beautiful code; complemented and formalized the approaches.
    \item fiswhick, aesthetic programming: showed that his approach also already exists within the textual form, and that his suggestion is just a matter of different representation
\end{itemize}

Another important part of the contribution is the \emph{interdisciplinary} aspect. It offered an example of how one can make multiple disciplines dialogue? the focus on the object, on the multiple realities of that object, then to find the Venn diagram (space, matter) and the liminal/side spaces.

%What are the implications of this new knowledge? Who needs to know what you have to say? Why? How could this knowledge be of interest/use to them?

% QUESTION: Concrete actions that can be taken in the real world?

\subsection{Limitations} %3k

The first limitation is that a lot of code is not available.

A second limitation is the expertise level required not just in programming, but in idiomaticity—that is, in knowing specific languages—and in the relevant problem domains. And that, by definition, there is a requirement to have already a solid basis on the technical and problem context in understand to justifiably assess the aesthetic manifestation of source code—since i'm not expert in all, there are obviously parts that i have overlooked.

There is also a broad definition of what aesthetic is, and is different from beautiful, which also elicits an emotional response, or gut feeling. so i don't talk about beauty, but about code which exhbitis positive aesthetic properties—finding beautiful code in an art history sense has not been achieved.

Additionally, the part about the psychology of aesthetics, the cognitive aspect is a burgeoning field, in active research, which means that some of the connections between code and cognition, or beauty and cognition are still a work in progress.

% Has my methodology or anything else affected my interpretation of findings and is this something that needs to be discussed (e.g. biases inbuilt into the research design)?
% in terms of access, i would like to thank alexandra elbakyan for her unvaluable support.

\section{Opening} %2k
% QUESTION What do i hope the outcome of this research will be?

I hope this research will lead into exploring the poetics of code (how software represents the world), or the intricacies of knowledge transfer/knowledge management.

For poetics, source code is modelling the real-world, and could that have an effective impact on the real world itself? Particularly in terms of  time and space. Tthese are unique things when it comes to source code, thus another qustions could be: how does source code affect our perception of time and space?

For knowledge management, what is the role of aesthetics for efficiently communicating concepts and ideas in computer-supported collaborative work?