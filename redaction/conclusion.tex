\chapter{Conclusion} %30k - 5-7% of total

%
% BITCH BE HUMBLE
%

% this is supposed to specify what we've found, why it's valuable, how it can be applied, what is new and exciting
% his will show how it has made a valid, and useful, contribution to knowledge.

% to check: https://patthomson.net/2012/12/19/conclusion-mise-en-place-christmas-present-six/

% _____LIMITATIONS_____
% the main limitation of the study is the explicitness, as opposed to the implicitness of source code beauty appreciations.
% there is also a limitation in the fact that a lot of code is not available
% and that, by definition, there is a requirement to _KNOW_ the technical and problem context in understand to justifiably assess the aesthetic manifestation of source code—since i'm not expert in all, there are obviously parts that i have overlooked.
% there is also a broad definition of what aesthetic is, and is different from beautiful, which also elicits an emotional response, or gut feeling. so i don't talk about beauty, but about code which exhbitis positive aesthetic properties—finding beautiful code in an art history sense has not been achieved.
% additionally, the part about the psychology of aesthetics, the cognitive aspect is a burgeoning field, in active research, which means that some of the connections between code and cognition, or beauty and cognition are still a work in progress.
% Has my methodology or anything else affected my interpretation of findings and is this something that needs to be discussed (e.g. biases inbuilt into the research design)?
% in terms of access, i would like to thank alexandra elbakyan for her unvaluable support.

% refresh the memory with the questions laid out in the introduction
First, our research aims were that __________. % and also we wanted to look at the realities of source code, in the light of some theories

% each of the research questions could make useful subheadings

% the key finding is that code is a matter from which we build spaces
In response, we've shown that aesthetics can be used to facilitate understanding in a functional context. More specifically, the aesthetic properties of code are derived from a conception of code as a semantic material which in terms is assembled, and apprehended, as a spatial construct, rather than a strictly literary, mathematical, or architectural one.
% in addition, what are the key points of the work?

% bigger picture
bigger picture: we've shown that the formal properties of a specific medium have a relationship to episteme.

The emphasis is on the \emph{functional}, in that there is something that a software should achieve, but there is also something that aesthetics should achieve: i.e. the communication of complex ideas.

The emphasis is also on the \emph{context}, in that, while there are conventions that have emerged, and psychological studies that have confirmed that particular kinds of layout and presentation are beneficial to program understanding, aesthetics are also situated, depending on the relation with the program, the machine, and the audience of the program, as well as the intent of the use.

We've also shown that, due to software's ambiguous nature as an \emph{abstract artifact}, a variety of domains are summoned to make sense of software beauty, each connecting the surface-level to the deep-level in their own way.

In the end, this allows us to think through the concept of aesthetics: not exclusively as an end to all things, but as means to represent \emph{something} to \emph{someone}, meaning that it acts as an interface between a concept, an idea, a person, and is presented to another person. Aesthetics, in this sense, have a clear communicative role. This clear communicative role, before it is even located within a particular environment (a turing machine), implies some sort of success. A successful communication is a communication which is correctly interpreted (here, the interpretation is, at minima, what the program does, and what the program intents to do, things that might not always be aligned).

We get a scale for exercising a value judgment: how much does this help/deter understanding?

% this part is a reminder of how we started, and some of the empirical findings, extending previous work
\section{Brief summary of what we've been through}

We've started this thesis by clarifying what it is that we mean by source code. On the one side, we've identified that there isn't one abstract programmer, nor one abstract source code. Code is written in various contexts, for various purposes, and therefore takes various forms.
\subsection{Empirical}

\subsection{Abstract artifact} %9k

\subsection{Function} %9k

\subsection{Context} %9k

And convention (style)

And technology (tools, idiomaticity)

% contribution to the field
\section{Contribution}

% realistically: gathered a larger corpus of source code examples
% realistically: offered a typology of how to exert aesthetic properties in code
% theoretically: connected it to epistemology and the arts
% statefully: proposed an explanation about code as matter, existing in semantic space

% QUESTION: Reflecting on the gaps in existing research, relation to existing theories, inscribed in a frameworks
% - which ones did i validate? (hayles, depth, paloque-berges on double-meaning, pineiro on instrumental action)
% - which ones did i invalidate? (cramer, magic)

% the ____interdisciplinary___ aspect is also an interesting one, tbh: how do we make multiple disciplines dialogue? the focus on the object, on the multiple realities of that object, then to find the venn diagram (space, matter)

% further research
\section{Opening} %9k
% QUESTION What do i hope the outcome of this research will be?

% QUESTION: Concrete actions that can be taken in the real world?

% poetics -> source code modelling the real-world, can that have an effective impact on the real world itself? particularly in terms of  time and space -> these are unique things when it comes to source code -> how does source code affect our perception of these?

% knowledge transfer -> what other conclusions can we draw in the role of aesthetics in knowledge transfer, knowledge management?