\chapter{Machine languages}
\label{chap:programming}

This chapter brings it all together: it develops on the medium itself of source code beauty: the programming languages. They will allow us to highlight two things. First, that there is a tension between human-meaning and machine-meaning, a tension between syntax and semantics. Second, it will allow us to show that what is considered beautiful in one language is not always considered beautiful in another, even though it has the same \emph{intent}, but clearly different syntax, and perhaps even different semantics.

Then, we extract form our case studies a theory/typology/nomenclature of aesthetic manifestations in programming. \emph{Perhaps we should do the theory before, coming out from the programming languages, and then apply it in case studies}.

Once we have shown the consolidated typology/exposé of aesthetic manifestations and devices, we will look at some case studies to illustrate all our previous points: the standards established beforehand (clean, elegant, etc.), the different social contexts, the different language contexts, the lit/arch/math/eng components and most importantly the relationship between surface-level and deep-level of a program-text. These case studies include poems, operating systems, parsers. (it would be nice to find something that is very hacky, or something that is very scientific).

\section{Machine meaning and human meaning}
\label{sec:human-machine-meaning}

This talks more about programming languages, we show how the abstract artifact is tanglibly manifested. We also look at machine semantics and syntax, and finally we look at language-independent features, or language independent features.

The way that programmers understand is through the establishment of mental spaces, as an idealized cognitive model composed of a \emph{base space} (the text) and \emph{built space} (the imagination that the readers has of the text). It's also related to the dialogical imagination: probing the code as constructing a mental model.

\subsection{Idiomaticity}
\label{subsec:idiomaticity}

Genette develops a concept of \emph{diégèse}, which could be equated to idiomaticity.

Ruby: \url{https://rubyglasses.blogspot.com/2007/08/actsasgoodstyle.html} and \url{https://evrone.com/yukihiro-matsumoto-interview}

\section{A Framework for source code aesthetics}
\label{sec:programming-aesthetic-framework}

This is where we outline a basic theory, as seen in the check-in 2.

What are the chronotopes of source code?
- local/global
- encapsulation/abstraction
- sync/async and blocking/non-blocking
- stack/heap
- memory/disk/network

\section{Case studies}
\label{sec:case-studies}

This shows the case studies.

\spacer


In conclusion, we have shown that a lot of the questions we have raised and features we have pointed out are manifested in programming languages. From there on, we have proposed our theory about levels and distances, confirming this idea of mental structures clearly manifested in source code, and of metaphors. This has been followed by case studies which have shown, in a variety of corpuses, how these are present in specific program-texts. And, again, that aesthetics is inseparable from functionality.