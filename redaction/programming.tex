\chapter{Machine languages}
\label{chap:programming}

After analyzing the discourses of programmers with regards to beautiful code, after highlighting the specific cognitive hurdles and bypasses inherent to software, and after having investigated how aesthetics enable various forms of understanding in associated fields, we now offer a framework for the aesthetics of source code. To do this, this chapter develops on the medium itself of source code beauty: the programming languages. Understanding what they are and how they are used will allow us to highlight two important aspects. First, that there is a tension between human-meaning and machine-meaning, a tension between syntax and semantics. Second, it will allow us to highlight yet another contextual aspect of source code aesthetics—just like natural languages, machine languages also act as linguistic communities.

Once we laid this material groundwork, we propose two approaches to the aesthetic manifestations in program texts. First, we build on a close-reading approach to suggest a perspective of various scales. We will therefore see how aesthetic manifestations exist along a linear axis of vocabulary, syntax and structure. Finally, we will take a step back to the standards established in \ref{chap:ideals} (e.g. clean, elegant, etc.) and show how these create different contexts for aesthetic appreciation.

Ultimately, this chapter continues on our investigation of how programmers establish mental spaces in order to understand a program text. Programming languages therefore act as an interface between a \emph{base space} (the text) and \emph{built space} (the imagination that the readers has of the text), resulting in a functional cognitive model.

\section{Semantic interfaces}
\label{sec:semantic-interfaces}

Software is an idea ultimately represented in specific hardware configurations. The immediate medium of this representation, from the programmer's perspective, is the programming language in which the idea is written down. Programming languages have so far been set aside when examining which sensual aspects of source code resulted in what could be deemed a "beautiful" program text. The relationship between semantics (deep-structure)  and its syntactic representation (surface-structure) is framed by programming languages, as they define the legal organization of form.

This section examines the influence of programming languages on the aesthetic manifestations of source code. To do so, we first go over a broad description of programming languages, ending on what makes a programming language expressive. Second, we touch upon the problem of semantics in programming languages, and how they might differ from a human understanding of semantics. We then we assess their fit as an artistic, expressive system by introducing notions to style and idiomaticity in programming language communities. Finally, we highlight a couple of computing-specific concepts which are made explicit by programming language research, and further define the kinds of concepts that are defined and manipulated when writing code.

\subsection{Programming languages}
\label{subsec:programming-languages}

As an introduction to the section, we start by recalling the historical and technical developments of programming languages, relocating them as an interface between hardware and software, before investigating what makes them good from a general, language designer point of view. This will highlight generic qualities such as orthogonality, abstraction and simplicity, which we will then compare to specific implementations.

\subsubsection{History and developments}
\label{subsubsec:history-developments}

A programming language is a strictly-defined set of syntactic rules and symbols for describing instructions to be executed by the processor. The history of programming languages is, in a sense, the history of decoupling the means of creating software from hardware. The earliest programming languages were embedded in hardware itself, such as piano rolls and punched cards for Jacquard looms \citep{sack_software_2019}. Operating on similar principles, the first electric computers—such as the ENIAC, the UNIVAC or the MUC—still required manual re-wiring in order to implement any change in the algorithm being computed. This process then gave way to programming through the stack of cards fed into the machine, a process which nonetheless retained a definite material aspect. It is with the shift to the stored-program model, at the dawn of the 1950s, that the programs could be written, stored, recalled and executed in their electro(-mecha)nical form, essentially freeing the software result from any immediately physical representation.

This tendency to have software gradually separate from hardware saw a parallel in the development of programming languages themselves. Ultimately, any software instruction needs to execute one of the built-in, hardwired instructions of the processor. Also called \emph{machine language}, these instructions set describe the specific implementation of the most common operations executed by a computer (e.g. \lstinline{add, move, read, load}, etc.), and are the oldest and most direct semantic interface to the hardware. While these are represented as binary numbers to the processing unit, a first layer of a family of languages called Assembly, translate those machines instructions into another set of instructions, whose syntax is loosely based on English. Considered today as some of the most low-level code one can write, Assembly languages are machine-dependent, featuring a one-to-one translation from English keywords to the kind of instruction sets known to the processor they are expected to interface with. As such, a program written for a particular architecture of a computer (e.g. x86 or ARM) cannot be executed without any modifications on a another machine.

The first widely acknoweldged high-level language which allowed for a complete decoupling of hardware and software was FORTRAN\footnote{Even though programming languages such as Plankalkül, Short Code and Autocode were partial proposals of such decouppling before FORTRAN.}. At this point, programmers did not need to care about the specifics of the machine that they were running on anymore, and found more freedom in their exploration of what could be done in writing software, expanding beyond scientific and military applications into the commercial world (see \ref{sec:practice-programmers}). Moving away from hand-crafted and platform-specific Assembly code also implied a certain sense of looseness incompatible with the extension of its application domain. As such, FORTRAN\footnote{Literally meaning FORmula TRANslation, thus making clear its role as a mediator.}, and the subsquent COBOL, Lisp and ALGOL 58 also started being concerned with the specific definition of their syntax in a non-ambiguous manner to ensure reliability. Using Backus-Naur Form notation, it became possible to formalize their syntactic rules in order to prevent any unexpected behaviour and support rigorous reasoning for the implementation and research of current and subsequent languages. With such specifications, and with the decoupling from hardware, programming languages became, in a way, context-free.

The context-free grammatical basis for programming allowed for the further development of compilers and interpreters, binary programs which, given a syntactically-valid program text, output their machine code representation. Such a machine-code representation can then be executed by the processor\footnote{The main difference between a compiler and an interpreter is that the compiler parses the whole program text as once, resulting in a binary object, while interpreters parse only one line at a time, which is then immediately executed}. At this point, a defining aspect of programming languages is their theoretical lack of ambiguity. This need for disambiguation was reflected both in the engineering roots of computation\footnote{Punch cards and electrical circuits are ultimately discreete—hole or no hole, voltage or no voltage.} and in their formal mathematic roots notation\footnote{For instance, Plankalkül was based on Frege's \emph{Begriffschrift}, a lineage we've seen in \ref{subsec:knowing-what-how}}, and was a requirement of the further development of functional software engineering.

Nowadays, most programming languages are Turing-complete: that is, their design allows for the implementation of a Turing machine and therefore for the simulation of any possible aspect of computation. This means that any programming language that is Turing-complete is equivalent to any other Turing-complete programming language, creating essentially a chain of equivalency between all programming languages. And yet, programming language history is full of rise and fall of languages, of hypes and dissapointments, of self-claimed beautiful ones and criticized ugly ones, from COBOL to Ada, Delphi and C. This is because, given such a wide, quasi-universal problem set, the decision space requires creative constraints: individual programmers resort to different approaches of writing computational procedures, echoing what Gilles Gaston-Granger undestands as \emph{style}, as a formal way to approach the production and communication of aesthetic, linguistic and scientific works \citep{granger_essai_1988}. We have already seen one difference in approaching the domain of computation: compilation vs. interpretation. While the input and outputs are the same \footnote{a program text goes in, and machine code comes out}, there are pros and cons\footnote{For instance, a compiled binary does not need an extra runtime to be executed on a machine, but cannot be immediately used on a different architecture than the one it was compiled for.} to each approach, which in turn allows programmers to bestow value judgments on which on they consider better than the other. Ultimately all programming languages need to address these basic components of computation, but they can do it in the way they want.

\begin{itemize}
  \item{\emph{data} (what kinds of basic datatypes are built-in the language, e.g. signed integers, classes)}
  \item{\emph{primitive operations} (how can the programmer directly operate on data, e.g. boolean logic, assignments, arithmetic operations)}
  \item{\emph{sequence control} (how the flow of the program can be manipulated and constrained, e.g. if, while statements)}
  \item{\emph{data control} (how the data can be initialized and assigned, e.g. type-safe vs. type-unsafe)}
  \item{\emph{storage management} (how the programming language handles input/output pipelines)}
  \item{\emph{operating environment} (how the program can run, e.g. virtual machine or not)}
\end{itemize}

This decision to change the way of doing something while retaining the same goal is particularly salient in the emergence of programming paradigms. A programming paradigm is an approach to programming based on a coherent set of principles, sometimes involving mathematical theory, and grouped into families. Some of these concepts include encapsulation and interfaces (in object-oriented programming), pure function and lacks of side effects (in functional programming), or mathematical logic (in declarative programming). Each paradigm supports a set of concepts that makes it the best for a certain kind of problem \citep{vanroy_programming_2012}, these concepts in turn act as stances which influence how to approach, represent and prioritize the computational concepts mentioned above.

So while programming languages all stem from and relate to a single commonality—Turing-completeness and data processing—, these linguistic interfaces nonetheless offer many approaches to performing computation, including a diversity and reliability of functions characteristic of tools. Since diversity within equivalence supports qualified preference, we can now examine what it its that makes a programming language good.

\subsubsection{Abstraction, simplicity and orthogonality}
\label{subsubsec:abstraction-simplicity-orthogonality}

Every programming language of practical use takes a particular approach to those basic components, sometimes backed by an extended rationale (e.g. ALGOL 68), or not (e.g. JavaScript). In the case in which we are circumscribed to context-free grammars, it would be possible to optimize a particular language for an objective standard (e.g. compile time, time use, cycles used). Still, computers exist to solve problems, those problems are diverse in nature and therefore necessitate different approaches (as we've seen in our discussion of the limitations of patterns above\footnote{See awk as a kind of fundamentally pattern-based scripting language.}). These different approaches to what is referred to as the \emph{problem domain} in turn influenced the development of those different paradigms, since a problem domain might have different data representations (e.g. objects, text strings, formal rules, dynamic models, etc.). Two of the early programming languages, FORTRAN and Lisp, addressed to very different problem domains: the accounting needs of businesses and the development of formal rules for artificial intelligence, respectively\footnote{For a specific discussion of these differences and how they are manifested aesthetically, see Kernighan on Pascal and C: \url{https://www.lysator.liu.se/c/bwk-on-pascal.html}}. Some of the overarching programming paradigms are imperative (FORTRAN), functional (Lisp), object-oriented (Smalltalk) or logic (Prolog). Without diving deeper in some of the worldmaking assumptions of each of these paradigms, they are here sufficient proof to show that, while there is only one Turing-completeness, there are widely different approaches to it, some being better than others when given a specific problem.

What makes a good programming language is a matter which has been discussed amongst computer scientists, at least since the \lstinline{GOTO} statement has been publicly considered harmful. Some of these discussions include both subjective arguments over preferred languages, as well as objective arguments related to performance and ease-of-use. According to Pratt and Zelkowitz:

\begin{quote}
  The difference among programming languages are not quantitative differences in what can be done, by only qualitative differences in how elegantly, easily and effectively things can be done. \citep{pratt_programming_2000}
\end{quote}

Without  then jumping immediately to aesthetic details of the languages they write in, programmers still express preferences (see \ref{subsec:style-idioms-programming}). One must also keep in mind that there is a difference between considering a programming language good or beautiful \emph{in itself}, and considered the programs written in the programming language. Turing-completeness offers an interesting challenge to the Sapir-Whorf hypothesis—if natural languages might only weakly affect the kinds of cognitive structures speakers of those languages can construct, programming languages are claimed to do so to large extents\footnote{See Alan Perlis's Epigrams on Programming: "A language that doesn't affect the way you think about programming, is not worth knowing.", \url{https://web.archive.org/web/19990117034445/http://www-pu.informatik.uni-tuebingen.de/users/klaeren/epigrams.html}}, even though they can all do the same thing in theory—only \emph{how} they do it matters. These differences in the ways of doing illustrates how, in reality, different programming languages are applicable to different domains, and do so through different kinds of notations—different aesthetic features when it comes to realizing the same task.

Of the two programs presented below, the output result is exactly the same, but the aesthetic differences are obvious:

\begin{listing}
  \inputminted{ruby}{./corpus/hello.rb}
  \caption{A terse example of writing a string to an output in Ruby.}
  \label{code:hello-ruby}

\end{listing}

The code above is written in Go, a language designed by Pike, Thompson and Griesemer in 2009, while the code below is written in Java, designed by James Gosling in 1995. Both are statically typed, compiled languages that are heavily influenced by C/C++ syntax, but Go is significantly younger than Java (relatively to programming languages' short history)

\begin{listing}
  \inputminted{java}{./corpus/hello.java}
  \caption{A verbose approach to writing a string to an output in Java.}
  \label{code:hello-java}
\end{listing}

These two snippets might seem very similar at first glance\footnote{For a radically different approach, see: \lstinline{greeting = "Hey there"; print greeting}, a valid program in Ruby, Python and Perl, all scripting languages.}. From these two snippets, we can explore some of the most important criteria in programming language design: \emph{abstraction}, \emph{simplicity} and \emph{orthogonality} \citep{sebesta_concepts_2018}, and how they underpin the writing of good programs.

Abstraction is the ability of the language to allow for the essential idea of a statement to be expressed without being encumbered by specifics which do not relate directly to the matter at hand, or to no matter at all. Abstract programming languages can lead to more succint code, and tend to hide complexity (of the machine, and of the language), from the programmer. For instance, the Java snippet above explicitly states the usage of the \lstinline{System} object, in order to access its \lstinline{out} attribute, and then call its \lstinline{println()} method. While a lot of code here might seem verbose, or superfluous, it is in part due to it being based on an object-oriented paradigm. However, \lstinline{out} object itself might seem to go particularly contrary to the requirement of programming languages to abstract out unnecessary details: \lstinline{println()} is definitely a system call dealing with I/O, and therefore already implicitly relates to the output; one shouldn't have to specify it explicitly. In contrast, Go abstracts away the system component of the print call inside the import statement \lstinline{import "fmt"}. Printing, in Java, does not abstract the machine, while printing, in Go, abstracts it away in order to focus on the actual appearance of the message ("fmt" stands for "format"). Another abstraction is that of the language name itself from the import statements. When we write in Java, we (hopefully) know that we write in Java, and therefore probably assume that the default imports come from the Java ecosystem—there shouldn't be any need to explicitly redeclare it. For instance, \lstinline{System.out.println()} isn't written \lstinline{java.io.System.out.println()}. In contrast, the Go snippet hides the implicit \lstinline{import "go/fmt"}, allow the programmer to focus, through visual clarity, on the real problem at hand, which is the logic of the program. In this direction, languages which provide more abstraction (such as Python), or which handle errors in an abstract way (such as Perl) tend to have greater readability by focusing on the most import tokens, rather than aggregating visual clutter—also called verbosity.

Related to abstraction, and a topic in itself, is the criteria of \emph{typing}, the process of specifiying the type of a variable or of a return value (such as integer, string, etc.). A strictly-typed language such as C++ might end up being unreadable because of its verbosity, while a type-free language might be simple to read and write (at a small-scale), but dangerous to run in production. The tradeoff here is again between being explicit and safe (because a word cannot usually and intuitively be operated on in a similar way as a floating-point number), and being implicit, subtle, and dangerous (such as JavaScript's very liberal understanding of typing). With experience, typing can usually be inferred by purely aesthetic means: Python's boolean values are capitalized (\lstinline{True, False}) and its difference between string and byte is represented by the use of double-quotes for the former and single-quotes for the latter. In the case above, explicitly having to mention that \lstinline{greeting} is of type \lstinline{String} is again redundant, since it is already hinted at by the double-quotes (and, indeed, single quotes are byte types in Java as well). Go doesn't force programmers to explicitly declare variable types (they can, if they want to), but in this case they let the computers do the heavy lifting of specifying something that is already obvious to the programmer.

A particularly note-worthy example of an elegant solution to the tradeoff between type safety and readability can be found in Go's handling of error values returned by functions:

\begin{listing}
  \inputminted{go}{./corpus/error_handling.go}
  \caption{Go proposes an elegant way of ignoring errors, with the use of the underscore token.}
  \label{code:error-handling}
\end{listing}

The \lstinline{_} character which we see on the first line is the choice made by Go's designers to force the user to both acknowledge and ignore the potential error value that is returned by executing the external command. This particular character, acting as an empty line, \emph{represents absence}, not cluttering the layout of the source, while reminding subtly of the \emph{potential} of this particular statement to go wrong and crash the program. Abstraction is therefore a tradeoff between explicitly highlighting the computer concern (how to operate practically on some data or statement), and hiding anything but the human concern (whether or not that operation is of immediate concern to the problem at hand at all). As such, languages who offer powerful abstractions tend not to stand in the way of the thinking process of the programmer. This particular example of the way in which Go deals with error-handling is a great example of the designer's explicit stylistic choice. In the words of Niklaus Wirth:

\begin{quote}
  Stylistic arguments may appear to many as irrelevant in a technical environment, because they seem to be merely a matter of taste. I oppose this view, and on the contrary claim that stylistic elements are the most visible parts of a language. They mirror the mind and spirit of the designer very directly, and they are reflected in every program written. \citep{wirth_essence_2003}
\end{quote}

Orthogonality, in turn, relates to the ability of a small set of simple syntactics constructs to be recombined in order to achieve greater complexity, while remaining independent from each other. A direct consequence of such a feature is the ease with which the programmer can familiarize themselves with the number of constructs in the language, and therefore their ease in using them without resorting to the language's reference. Relating back to Ricoeur, the orthogonality of a language offers a simple but powerful solution to the polysemy of the word (here, token) as embedded within a broader sentence (here, statement). The expressivity of a statement comes not just from the individual keywords, but rather from their combination. For instance, the example of Lisp treats both data and functions in a similar way, essentially allowing the same construct to be recombined in powerful ways (again, the Lisp interpreter comes to mind), while the Ruby language, and its foundational design choice which makes every type (themselves abstracted away) an object allows for greater creativity, through familiarity, in writing code, making the language itself more habitable. Orthogonality also implies independence, since all constructs operate distinctly from each other, while remaining related, because in cooperation with each other. This offers a solution to the cognitive burden of the \emph{non-atomicity} of computer programs, in which data can end up being tangled in a non-linear program execution, and become unreadable. This unreadability is triggered, not by verbosity, but because of the uncertainty of, and confusion about, the potential side-effects caused by any statement. Such independence in all constructs in turn presents a kind of \emph{symmetry}—a well-accepted aesthetic feature of any artefact—, in that the use of each of the constructs is similar. This similarity eases the cognitive friction in writing and reading code since an orthogonal language allows the programmer to rely on the fact that, deep down, everything is the same, and not a collection of quirks and arbitrary decisions. For example, the below C code is illegal:

\begin{listing}
  \inputminted{go}{./corpus/illegal_return.c}
  \caption{C syntax sometimes behaves arbitrarily.}
  \label{code:illegal-return}
\end{listing}

The above code is a specific instance of one of those quirks: the fact that C cannot return arrays from functions requires both a deep knowledge of the language implementation and a willingness to accept that this is how things are, even though other languages allow for such a feature. In this case, the language exhibits an un-orthogonal property since the two constructs (\lstinline{return} and \lstinline{int[]}) interact with each other in non-independent ways.

Finally, one of the consequences of such a feature is the shift from computer semantic interpretation (usually connected to strongly-typed languages) to human-interpretation (and weakly-typed languages). Non-orthogonality implies that the compiler (as a procedural representation of the language) has the final say in what can be expressed, while orthogonal languages leave more leeway to the writer in keeping track of which value and statement does what, allowing for both more creativity and more uncertainty in the interpretation and execution of the program.

Both of these features, abstraction and orthogonality, ultimately relate to simplicity:

\begin{quote}
  Simplicity enters in four guises: uniformity (rules are few and simple), generality (a small number of general functions provide as special cases a host of more specialized functions, orthogonality), familiarity (familiar symbols and usages are adopted whenever possible), and brevity (economy of expression is sought).\citep{stansifer_study_1994}
\end{quote}

The point of a simple programming language is a programming language which does not stand in the way of the program being written, or of the problem being addressed. Such a goal is achieved in part by having accurate conceptual mappings between computer expression mapping and human mapping (such as the code block-human sentence mapping\citep{fedorenko_language_2019}). If one is to write a program related to an interactive fiction in which sentences are being input and output in C, then the apparently simple data structure \lstinline{char} of the language reveals itself to be cumbersone and complex when each word and the sentence that the programmer wants to deal with must be present not as setences nor words, but as series of \lstinline{char}\footnote{Hence the origin of the name of the data type \lstinline{string}, as a continuous series of \lstinline{char}}. As we've seen, a simple language does not mean that it is easy (perhaps the simplest language of all being lambda-calculus, is far from an easy construct to grasp, just like Go\footnote{The game, not the language.} is simple without being easy), but that it is just a means to an end, akin to any other tool or instrument\footnote{For a further parallel on musical instruments, see Rich Hickey's keynote address at RailsConf 2012: \url{https://www.youtube.com/watch?v=rI8tNMsozo0}}.

Programming languages, however, are \emph{symbolic} tools, manipulating \emph{symbolic} matter. As a formal system of symbols which can sustain creations with aesthetic features, programming languages do share commonalities with Goodman's \emph{Languages of Art}.

Goodman develops in his opus a systematic approach to symbols in art, freed from any media-specificity (from pictorial symbols to musical notations and even time marks on clocks and watches). With it, he accomplished two things: he highlights the ways in which symbols systems have expressive and communicative power (through the dyads of denotation and exemplification, description and representation, possession and expression), and what are the kinds of requirement that such a system must have in order to develop these expressive and communicative abilities. These requirements are that of unambiguity, syntactic and semantic disjointedness, and differentiation\citep{goodman_languages_1976}. Looking at programming languages from this perspective argues for their communicative and expressive power. From the perspective of the computer, programming languages are unambiguous insofar as any expression or statement will ultimately result in an unambiguous result by the CPU (if any ambiguity remains, the program crashes). They are also syntactically and semantically disjointed (i.e. clearly distinguishable from one another). The use of formal notations, such as BNF, had for aim to resolve any possible ambiguity in the syntax of the language in a very clear fashion. The semantics of programming languages, as we will see below, also aim at being thoroughly disjointed: a variable cannot be of multiple types at the same time. Finally, programming languages are also differentiated systems since no symbol can refer to two things at the same time.

The tension arises when it comes to the criteria of unambiguity, from a human perspective. The most natural-language-like component of programs, the variable and function names, always have the potential of being ambiguous (e.g. does \lstinline{int numberOfFlowers} refer to the current number of flowers in memory? To the total number of potential of flowers? To a specific kind of number whose denomination is that of a flower?). We consider this ambiguity a productive opportunity for creativity, and a hindrance for program effectiveness. So, given the qualification of programming languages as symbolic systems, we could expand our short analysis above by inspecting how programming languages allow for program texts which denote, label, represent, etc. in order to further argument how source code has the potential, and has examples, of being an artistic means of expression and comprehension, from a cognitive point of view.

If programming languages are aesthetic symbol systems, then they can also elicit emotional responses. As we've seen with software patterns, what also matters to programming languages is not just their design, but their \emph{situated} use:

\begin{quote}
  It must be a pleasure and a joy to work with a language, at least for the orderly mind. The language is the primary, daily tool. If the programmer cannot love his tool, he cannot love his work, and he cannot identify himself with it. \citep{wirth_essence_2003}
\end{quote}

Bringing it back to architecture, the language designer bruce McLennan further presses the point:

\begin{quote}
  There are other reasons that elegance is relevant to a well-engineered programming language. The programming language is something the professional programmer will live with - even live in. It should feel comfortable and safe, like a well-designed home or office; in this way it can contribute to the quality of the activities that take place within it. Would you work better in an oriental garden or a sweatshop? \citep{mclennan_who_1997}
\end{quote}

\subsubsection{Idiosyncracies and implementations}
\label{subsubsec:idiosyncracies-implementations}

Computation, in its philosophical sense, is a complex and debated concept \citep{rapaport_philosophy_2005,smith_origin_1998}. In short, software isn't simple, it's a cognitive artefact which can be understood at the physical, design and intentional levels\citep{moor_three_1978}. With modern programming languages allowing us to safely ignore the first level, it is at the interaction of the design (programming) and intentional (human) level that things get complicated; the question "what does a Turing machine do?" has \lstinline{n+1} answers, \lstinline{1} syntactic answer, and \lstinline{n} semantic ones, based on however many interpretations.

Without diving into the depths of the philosophy of computation, we highlight two programming concepts which tend to be evident to the computer (evident in the sense some see them as emergent properties of computation), and yet quite complex to deal with for humans: \emph{referencing} and \emph{threading}.

Referencing is a surface-level consequence of the \emph{use-mention} problem referred to above, the separation between a name and its value, with the two being bound together by the address of the physical location in memory. As somewhat independent entities, it is possible to manipulate them separately, with consequences that are not intuitive to grasp. Some programming languages allow for this direct manipulation, through something called \emph{pointer arithmetic}\footnote{For better or worse, C is very liberal with what can be done with pointers.}. Indeed, the possibility to add and substract memory locations independent of the values held in these locations, as well as the ability to do arithmetic operations between an address and its value isn't a process whose meaning comes from a purely experiential or social perspective, but rather exists meaningfully for humans only through logical grounding, by understanding the theoretical architecture of the computer. What also transpires from these operations is another dimension of the non-linearity of programming languages, demanding complex mental models to be constructed and updated to anticipate what the program will ultimately result in when executed. Notation attempts at remediating those issues by offering symbols to represent these differences, such as:

\begin{listing}
  \inputminted{python}{./corpus/references.c}
  \caption{Pointers are crazy.}
  \label{code:references-c}
\end{listing}

The characters \lstinline{*} and \lstinline{&} are used to signal that one is dealing with a variable of type pointer, and that one is accessing the pointed location of a variable, respectively. Line 2 of the snippet above is an expression called \emph{dereferencing}, a neologism which is perhaps indicative of the lack of existing words for referring to that concept. In turns, this hints at a lack of conventional conceptual structures to which we can map such a phenomenon.

Threading is the ability to do multiple things at the same time. While the concept itself is simple, to the point that we take it for granted in modern computer applications since the advent of time-sharing systems. However, the proper handling of threading when writing and reading software is itself complex. This involves the ability to demultiply the behaviour of routines (already non-linear) to keep track of what \emph{could} be going on at any point in the execution of the program, including use and modification of shared resources, the scheduling of thread start and end, as well as synchronization of race conditions (e.g. if two things happen at the same time, which one happens first, such that the consistence of the global state is preserved?). As Edward A. Lee put it:

\begin{quote}
  Although threads seem to be a small step from sequential computation, in fact, they represent a huge step. They discard the most essential and appealing properties of sequential computation: understandability, predictability, and determinism. Threads, as a model of computation, are wildly non-deterministic, and the job of the programmer becomes one of pruning that nondeterminism\citep{lee_problem_2006}.
\end{quote}

Threading shows how the complexity of a deep-structure needs to be adequately represented in the surface. Once again, aesthetically-satisfying (simple, concise, expressive) notation can help programmers in understanding what is going on in a multi-threaded program, by removing additional cognitive overload generated by verbosity\footnote{For an example of code that looks good on the surface, but is deeply wrong, see p.105 of \url{https://mirrors.edge.kernel.org/pub/linux/kernel/people/paulmck/perfbook/perfbook-e2.pdf}.}. Here are two of the simplest examples, in C and in Go:

\begin{listing}
  \inputminted{python}{./corpus/thread.c}
  \caption{Complex way to do threads in C.}
  \label{code:threading-c}
\end{listing}

\begin{listing}
  \inputminted{python}{./corpus/thread.go}
  \caption{Nice way to do threads in Go.}
  \label{code:threading-go}
\end{listing}

Once again, we see how the abstraction provided by some language constructs in Go result in a simpler and more expressive program text. In this case, the non-essential properties of the thread are abstracted away from programmer concern. The \emph{double-meaning} embedded in the \lstinline{go} keyword even uses a sensual evokation of moving away (from the main thread) in order to stimulate implicit understanding of what is going on.

Since programming languages are meant to help programmers solve semantic issues (in the problem domain) through elegant syntactical means, and since they act as interfaces between the programmer and the machine, we now turn to the question of how semantics are represented programmatically, and whether machine understanding exists.

\subsection{Machine semantics and human semantics}
\label{subsec:machine-semantics-human-semantics}

% semantics are essentially operative semantics

One of the reasonings behind the formal approach to programming languages, besides the very material machine requirements of a circuit design based on discreete distinctions, is, according to the designers of ALGOL 58, the dissatisfaction with the fact that subtle semantic questions remained unanswered due to a lack of clear description \citep{sethi_programming_1996}. If the goal of a program text is to be syntactically and semantically clear, and if programming languages are syntactically unambiguous, we examine here under what form do semantics exist as computer representations, and what kind of specific semantic issues are at stake when writing program texts. The very requirement for semantic representation in program language design is first and foremost due to the fact that:

\begin{quote}
  The first and most obvious point is that whenever someone writes a program, it is a program about something. \citep{winograd_language_1982}
\end{quote}

A statement which is itself followed by the tension between semantics and syntax. Semantics have the properties of aboutness and directedness (they point towards something external to them), and syntax has the property of (local) consistency and combination (they function as a mostly closed system). Looking at programing languages as applied mathematics, in the sense that it is the art and science of manipulating formal tokens, tokens which in turn represent elements in the world of some kind, we arrive at the issue of defining semantics in strictly computer-understandable terms.

Meaning is created by an active reading, in which the linguistic form enables interpretation, rather than exclusively conveying information. Winograd states that interpretation happens through \emph{grounding}, essentially contextualizing information in order to interpret it and extract meaning. He identifies three different kinds of grounding: experiential, formal, and social. The \emph{experiential} grounding, in which verification is made by direct observation, related to the role of the senses in the constitution of the conceptual structures that enable our understanding of the world—also known as the material implementation of knowledge. The \emph{formal} grounding relies on logical and logical statements to deduce meaning from previous, given statements that are known. Finally, \emph{social} grounding relies on a community of individuals sharing similar conceptual structures in order to qualify for meaning to be confirmed. Of these three groundings, programming languages rely on the second.

The reason for the bypassing of experiential and social grounding can be found in one of the foundations of computer science, as well as information science: Claude Shannon's mathematical theory of communication. In it, he postulates the separation of meaning from information, making only the distinction between signal and noise. Only formal manipulation of signal can then reconstitute meaning\footnote{An affordance that is shared with literature, according to Peter Suber\citep{suber_what_1988}}. Indeed, according to Brian Cantwell-Smith, computing is \emph{meaning mechanically realized}, due to the fact that the machine comes from non-mechanical origins\footnote{Retrieved from: \url{https://web.archive.org/web/20160826234606/http://ageofsignificance.org/aos/en/aos-v1c0.html}}. We think of computers as digital but they can be seen as only the digital implementation of the phenomenon of computation, with its roots in formal logic. It is therefore through formal logic that one can recreate meaning through the exclusive use of the computer.

A computer is actually a collection of layers, each defining different levels of machines, with different semantic capabilities. First, it is a physical machine, dealing with voltage differences. These voltage differences are then quantized into binary symbols, in order to become manipulable by a logical machine. From this logical machine is built an abstract machine, which uses logical grounding in order to execute specific, pre-determined commands. The interpretation of which commands to execute, however, leaves no room for the kind of semantic room for error that humans exhibits (particularly in hermeneutics). It is a strictly defined mapping of an input to an output, whose first manifestation can be found in the symbols table in Turing's seminal paper. The abstract machine, in turn, allows for high-level machines (or, more precisely, high-level languages which can implement any other abstract machine). These languages themselves have linguistic constructs which allow the development of representational schemes for data (i.e. data structures such as \lstinline{structs, lists, tuples, objects}, etc.). Finally, the last frontier, so to speak, is the subject domain: the things that the programmer is talking about. These are then represented in data structures, manipulated through high-level languages, processed by an abstract machine and executed by a logical machine which turns these representations into voltage variations.

The subject domain is akin to a semantic domain, a specific conceptual place that shares a set of meanings, or a language that holds its meaning, within the given context of this place. And there is only one context which the computer provides: itself. Within this unique context, semantics still hold a place in any programming language textbook, and is addressed regularly in programming language research. Concretely, \emph{semantics in computer programming focuses on how variables and functions should behave} \citep{sethi_programming_1996}. Given the statement \lstinline{l := j + p}, the goal of programming language semantics is to deduce what is the correct way to process such a statement; there will be different ways to do so depending on the value and the type of the \lstinline{j} and \lstinline{p} variables. If they are strings, then the value of \lstinline{j} will be their concatenation. If they are numbers, it will be their addition, and so on.

This problem is called the \emph{use-mention} problem, which requires the reconciliation of the name of entities, tokens in source code, with the entities themselves, composed of a value and a type. The way this is achieved is actually quite similar to how syntax is dealt with. The compiler (or interpreter), after lexical analysis, constructs an abstract syntax tree representation of the statement, separating it, in the above case, in the tokens: \lstinline{l}, \lstinline{:=}, \lstinline{j}, \lstinline{+} and \lstinline{p}. Among these, \lstinline{:=} and \lstinline{+} are considered terminal nodes, or leaves, while the other values still need to be determined. The second pass represents a second abstract syntax tree through a so-called semantic analysis, which then \emph{decorates} the first tree, assigning specific values (attributes) and types to the non-terminal nodes, given the working environment (e.g. production, development, test). This process is called \emph{binding}, as it associates (binds) the name of a variable with its value and its type. Semantics is thus the decoration of parsed ASTs, evaluating attribute—which can be either synthesized or inherited. Since decoration is the addition of a new layer (a semantic layer) on top of a base layer (a syntactic one), but of a similar tree form, this leads to the use of what can be described as a \emph{meta-syntax tree}.

In terms when the values are being bound, there are multiple different binding times, such as language-design time (when the meaning of \lstinline{+} is defined), compile time, linker time, and program-writing time. It is only during the last one of these times, that the programmer inserts their interpretation of a particular meaning (e.g. \lstinline{j := "jouer"}, meaning one of the four possible actions to be taken from the start screen of a hypothetical video game). Such a specific meaning is then shadowed by its literal representation (the five consecutive characters which form the string) and its pre-defined type (\lstinline{strings}, here in Go). This process does show that the meaning of a formal expression can, with significant difficulty and clumsiness, nonetheless be explained; but the conceptual content still eludes the computer, varying from the mundane (e.g. a simple counter) to the almost-esoteric (e.g. a playful activity). Even the most human-beautiful code cannot force the computer to deal with new environments in which meaning has, imperceptibly, changed. Indeed,

\begin{quote}
  In programming languages, variables are truly variable, whereas variables in mathematics are actually constant \citep{wirth_essence_2003}.
\end{quote}

From this perspective, the only thing that the computer does know that the programmer doesn't, and which would "make its life easier", the same way that the programmer's life can be made easier through beautiful code, is how the code is represented in an AST, and where in physical memory is located the data required to give meaning to that tree\citep{stansifer_study_1994}. We might hypothesize that beautiful code, from the computer's perspective, is code which is tailored to its physical architecture, a feat which might only be realistically available when writing in Assembly\footnote{For a mythical telling of such a process, see the story of Mel, A Real Programmer: \url{https://www.cs.utah.edu/~elb/folklore/mel.html}.}. Before we turn to how such a code is written by the particular group of humans referred to as hackers, there are nevertheless some concepts in programming which do not have simple meaning for humans, re-iterating the need of aesthetics to make these concepts graspable.

\subsection{Styles and idioms in programming}
\label{subsec:style-idioms-programming}


Concrete use of programming languages operate on a different level of formality: if programming paradigms are top-down strategies specified by the language designers, they find their mirror in the bottom-up practices of softare developers (to borrow Michel De Certeau's terminology). Such practices crystallize, for instance, in \emph{idiomatic writing}. Idiomaticity refers, in traditional linguistics, to the \emph{realized} way in which a given language is used, in contrast with its possible, syntactically-correct and semantically-equivalent, alternatives. For instance, it is idiomatic to say "The hungry dog" in English, but not "The hungered dog" (a correct sentence, whose equivalent is idiomatic in French and German). It therefore refers to the way in which a language is a social, experiential construct, relying on intersubjective communication \citep{voloshinov_marxism_1986}. Idiomaticity is therefore not a purely theoretical feature, but first and foremost a social one. This social component in programming languages is often reliant on knowledge of said language, and of its quirks. In this sense, programming language communities are akin to hobbyists clubs, with their meetups, mascots, conferences and inside-jokes\footnote{For an example of such joke, see Gary Bernhardt's talk on JavaScript: \url{https://www.destroyallsoftware.com/talks/wat}}.

So an idiom in a programming language depends on the human interpretation of the formal programming paradigms (since, in most programming languages today and especially in scripting languages, paradigms are blended and no language is purely single-paradigmatic). Such an interpretation is also manifested in community-created and community-owned documents, such as \emph{The Zen of Python} \citep{peters_code_1999}.

% \emph{The Zen of Python} shows how the philosophy of a programming language relates to the practice of programming in it. Without particular explicit directives, it nonetheless highlights \emph{attitudes} that one should keep in mind and exhibit when writing Python code. Such a document sets the mood and the priorities of the Python community at large (being included in its official guidelines in 2004), and highlights a very perspective on the priorities of theoretical language design. For instance, the first Zen is clearly states the priorities of idiomatic Python

\begin{quote}
  Beautiful is better than ugly.
\end{quote}

This epigram sets the focus on a specific aspect of the code, rather than on a specific implementation. With such broad statements, it also contributes to strengthening the community bonds by creating shared, folk knowledge. In practice, writing idiomatic code requires not only the awareness of the community standards around such an idiomaticity, but also knowledge of the language construct themselves which differentiate it from different programming languages. For instance, in Python, distinct syntactical operators are semantically equivalent:

\begin{listing}
  \inputminted{python}{./corpus/range.py}
  \caption{These two range operators are semantically equivalent in Python, but the first is more idiomatic than the second.}
  \label{code:range-operator}
\end{listing}

but only the second example is considered idiomatic Python, partly because it is \emph{specific} to Python, and because it is more performing than the first example, due to the desire of the developers of Python to encourage idiomaticity (i.e. what they consider good Python to be). Beautiful code, then seems to be a function of knowledge, not just of what the intent of the programmer is, but knowledge of the language itself as a differentiated idiom. Another example\footnote{From \url{https://www.quora.com/What-makes-some-code-beautiful}} of beautiful, because idiomatic, Python code is:

\begin{listing}
  \inputminted{python}{./corpus/fibonacci.py}
  \caption{The decorator is the idiotmatic way to calculate the sum of the Fibonacci sequence.}
  \label{code:fibonacci}
\end{listing}

This function calculates the Fibonacci sequence (a classic exercise in computer programming), but makes an idiomatic (and clever) use of decorators in Python. The \lstinline{@lru_cache(3)} line caches the last 3 results in the least-recently used order, closely mirroring the fact that the Fibonacci sequence only ever needs to compute the terms n, n-1 and n-2, thus closely aligning the language domain and the problem-domain. Through this, the programmer uses a key, advanced feature of the language in order to make the final program more terse, more precise, and mirroring more faithfully the problem than other implementations, to the detriment of a decrease in readability for non-Pythonistas.

Idiomaticity reflects what the social and aesthetic intent of the language designers and implementers. Notation matters, and designers want to encourage good practices through good notations, assuming that programmers would gravitate towards what is both the most efficient and the best-looking solution. For example, one of the biggest differences between object-oriented and non-object-oriented programming is the possibility to identify the actor of an action using purely syntactic means\citep{sustrik_250bpm_2021}. Another way to encourage writing good code is through the addition of \emph{syntactic sugar}. Syntactic sugar describes the aesthetic features of the language who are variants of a similar feature, and where the only difference between them is their appearance—i.e. visual, semantic shortcuts. The looping examples above are good instances of syntactic sugar, albeit with performance differences. The Ruby language is riddled with syntactic sugar, and highlights how syntactic sugar can "sweeten" the reading process, aiming for more clarity, conciseness, and proximity to natural languages. In Ruby, to access a boolean value on an attribute of an object, one would write it as in any other language. The syntactic sugar in Ruby comes in the form of the question mark in control flow statements, as shown in \ref{code:ruby-alive}.

\begin{listing}
  \inputminted{python}{./corpus/alive.rb}
  \caption{Ruby features a lot of syntactic sugar.}
  \label{code:ruby-alive}
\end{listing}

There is absolutely no functional differences in the statements above, and the question mark is just here to make the code seem more natural and intuitive to humans. Checking for a boolean (or non-nil value) in an if statement is, in the end, the equivalent of asking a question about that value Here, Ruby makes that explicit, therefore making it easier to read with the most minimal amount of additional visual noise (i.e. one character).

We've seen how programming languages can be subjected to the criteria of goodness, but how those criteria are only there to ultimately support the writing of good (i.e. functional and beautiful) code. Such a support exists via design choices (abstraction, orthogonality, simplicity), but also through the practical uses of programming languages, notably in terms of idiomaticity and of syntactic sugar, allowing some languages more readability than others (e.g. APL vs. Ruby). Like all tools, it is the (knowledgeable) use of programming languages which matters, rather than their design, and it is the problems that they are used to deal with, and the way in which they are dealt with which ultimately informs whether or not a program text in that language will exhibit aesthetic features.

Idiomaticity allows us to cross-examine what programming languages have to do with appropriateness, material honesty linguistic communitarianism

Genette develops a concept of \emph{diégèse}, which could be equated to idiomaticity.

add something about language communities (pythonistas, gophers, rustlings, etc.)

Rust: \url{https://www.ncbi.nlm.nih.gov/pmc/articles/PMC7959618/}

Ruby: \url{https://rubyglasses.blogspot.com/2007/08/actsasgoodstyle.html} and \url{https://evrone.com/yukihiro-matsumoto-interview}, \url{http://www.randomhacks.net/2005/12/03/why-ruby-is-an-acceptable-lisp/}

Java: \url{https://steve-yegge.blogspot.com/2006/03/execution-in-kingdom-of-nouns.html}

Go: \url{https://dave.cheney.net/practical-go/presentations/gophercon-singapore-2019.html#_footnotedef_1}

C: \url{https://www.lysator.liu.se/c/ten-commandments.html}

language design \url{https://dl.acm.org/doi/10.1145/390016.808420}, \url{http://rigaux.org/language-study/syntax-across-languages/}

INTERCAL (and esolangs in general) \url{https://cabinetmagazine.org/issues/36/bratishenko.php}

LISP \url{https://www.defmacro.org/ramblings/lisp.html}

PHP \url{http://phpsadness.com/}

\begin{quote}
  And it's not hard to "prove" it: If two people write code to solve the same problem and one makes a terrible spaghetti monster in COBOL while the other goes for super-elegant and highly abstracted solution in Haskell, does it really matter to the computer? As long as the two are compiled to the same machine code, the machine does not care. All the clever constructs used, all the elegance, they are there only to guide our intuition about the code. \url{https://250bpm.com/blog:152/index.html}
\end{quote}

In the end, the beauty and the nightmare is that there are multiple ways to do the same thing: \url{https://hachibu.net/posts/2019/7-ways-to-loop-in-javascript/}

\url{https://pointersgonewild.com/2022/05/23/minimalism-in-programming-language-design/}


\spacer

In conclusion, programming languages, as aesthetic symbol systems, are essential in allowing for aesthetic properties to emerge during the writing process of program texts. They present affordances for the abstraction and combination of otherwise-complex programming concepts, for the development of familiarity through their idiomatic uses and for ease of readability—to the point that it might become transparent to experienced readers. Still, since understanding is at stake, there is a kind of beautiful code which bypasses any semantic representation embedded in these languages by speaking directly to the machine. This is the kind of beauty which we look at in the following section.

\section{A Framework for source code aesthetics}
\label{sec:programming-aesthetic-framework}

\subsection{Matters of scale}
\label{subsec:matters-of-scale}

As mentioned above, the corpus of textbooks, essays, online blog posts and comments addressing what makes source code beautiful or aesthetically pleasing has been analyzed according to three main categories: structure, syntax and vocabulary, as all related to formal manifestations in source code, both functioning at different scales. *Structure* is defined by the relative location of a particular statement within the broader context of the code-text, as well as the groupings of particular statements in relation to each other and in relation to the other groups of statements within the code-text. This also includes questions of formatting, indenting and linting as purely pattern-based formal arrangements. *Syntax* is defined by the local arrangement of tokens within a statement, including control-flow statements (and therefore not restricted to single-line statements). It also includes language-specific choices, referred to as idioms, and generally the type of statements needed to best represent the task required (e.g. using an `Array` or a `struct` as a particular data structure for a collection of things). Finally, the *vocabulary* refers to the user-defined elements of the source code, in the form of variables, functions, classes and interfaces. Unlike the two precedent categories, this is the only where the writer can come up with new tokens, and is the closest to metaphors in traditional literature.

\subsubsection{Structure}
\label{subsubsec:framework-structure}

Structure, as the highest-level group of criteria, is both easy to grasp and somewhat superficial: most of the criteria which compose it are indicators and not proof of beautiful code, indeed necessary,  but not sufficient. Structure itself can be further separated between surface-structure, and deep-structure. The criteria for beauty in surface-structure is layout, as the spatial organization of statements, through the use of line breaks and indentations. While serving additional ends towards understanding, proper layout (whether according to conventions, or deliberately positioning themselves against these conventions) seems to be the first requirement for beautiful code. In terms of aiding understanding, blank space creates semantic groupings which enable the reader to grasp, at a glance, what are the decisive moments \citep{sennett_craftsman_2009} in the code's execution, and presented by some as akin to paragraphs in litterature \citep{oram_beautiful_2007} (here, cite the actual chapter by matz). Any cursory reading of source code always first and foremost judges layout.

This aid to understanding is further highlighted by a deep-structure criteria of \emph{conceptual distancing}; statements that have to do with each other are located close to each other. As such, visual appearance reflects the conceptual structure of the code (and, indeed, some argue that the data which the code processes predates the code itself in dictating its layout). While an over-arching principle, it is vague enough to be open to interpretation by practitioners and is therefore unable to act as a strict normative criteria (e.g. should every code-text follow the stepdown rule of function declaration or alphabetical rule of function declaration when writing in a language which doesn't enforce it? should local variables all be declared at the beginning of the highest scope at which they belong, or at the closest location of their next use? should all data be prepared, and then processed, or should each data be prepared and processed in each of their contexts?).

The related criteria of \emph{local coherence} (what is next to each other is related to each other) echoes similar questions about the coherence and consistence in Goodman's aesthetic theory. Local coherence enables what Goodman calls semantic density, in which tokens grouped together obtain a greater denotative power. Local coherence seems at first to stand at odds with the undesirable but unavoidable entanglement of code. Indeed, proponents of local coherence in source code imply that a beautiful piece of code should not have to rely on input and output (i.e. not be entangled) and therefore be entirely autotelic. Such an assumption runs contrary to the reality of software development as a practice, and as an object embedded in the world, and thus not "usable" by software developers. Such an isolated approach to code doesn't match the claims above that code should be useful, and so is probably intended to be understood in a more flexible manner (see in vocabulary below: function arguments).

A correlate to conceptual distancing is \emph{conceptual symmetry}, which states that groups of statement which do the same thing should look the same. It then becomes possible to catch a glimpse of patterns, in which readers know what does what according to a brief overview. Conceptual distancing can be further improved by \emph{conceptual uniqueness}, which demands that all the statements that are grouped together only refer to one single action: complex enough to be useful, and simple enough to be graspable (possibly the start of a definition of *elegance*). Following this, then, beautiful code is "the code that does the job while using the least amount of different ideas", which, according to the DRY principle, implies a linear relationship between the number of lines of code and the amount to be understood. This is possibly an aesthetic standard, but it is unclear to what extent it is a sublime standard.

Interestingly, this last statement contradicts another aesthetic that exists among different software practicioners—hackers. In their case, beautiful code is the code which manages to pack the maximum number of ideas in a minimal amount of lines of code, both in obfuscation practices and in the writing of one-liners. This apparent conflict between clarity and complexity can be resolved in certain codebases, in which the lines of code are few, but the implications are many.

\subsubsection{Syntax}
\label{subsubsec:framework-syntax}

Syntax, as the mid-level group of criteria, deals most specifically with the two main components of the implementation: the algorithm and the programming language. Beautiful syntax seems to denote a conceptual understanding of the tools at hands to solve a particular problem (Knuth states that these understandings are the ones that make writing code an art, and has devoted his magnum opus to the study and communication of algorithms). Both algorithms and languages are seen as tools since, again, it is through its use and implementation that a piece of code is considered functional, and can thus be examined for aesthetic purposes. Due to this need for implementation, I will argue that algorithms exist independently from languages, but that their aesthetic value in the context of this research cannot be separated from the way they are written, and the language they are written in. Indeed, most algorithms are expressed first as pseudo-code and then implemented in the language that is most suited to a variety of factors (speed, familiarity of the author, suitability of the syntax); this seems to be a contemporary version of the 1950s, when computer scientists would devise those algorithms through pencil and paper, and then leave their implementation at the hands of entirely different individuals—computers.

Beautiful syntax in code responds to this limitation. Since algorithms must be implemented in a certain context, with a certain language, it is the task of the writer to best do so with respect to the language that she is currently working in. In this case, knowledge of the language-as-tool and its relevant application makes beautiful syntax an *idiomatic syntax*. This involves knowing the possibilities that a given language offers and, in the spirit of the craftsmanship ethos noted previously, working with the language rather than against it. These sets of aesthetic criteria thus become entirely dependent on the syntactical context of the language itself, and can only be established with regards to each languages (e.g. knowing which keywords shouldn't be used, such as \lstinline{unless} in perl, or \lstinline{*} in C, knowing when to use decorators in python, the spread \lstinline{...} operator in ECMAScript, etc.). Here, syntax also follows the idea of conciseness that has been touched upon at the structure-level: a writer can only be concise if she knows how the language enables her to be concise, and knowing the algorithm and the problem domain will not help to match this criteria. To what extent a syntax is idiomatic (see above) syntax is therefore a good indicator of the aesthetic value of a code-text, while refraining from being too idiomatic (often referred to as "clever code" and generally frowned upon).

It is difficult to establish a hierarchy between separate idioms, since they operate under different paradigms and assumptions. A developer who finds that she can best communicate her ideas according to Java will find Java beautiful. A developer who finds that she can best communicate her ideas while writing in Go will find Go beautiful. This state of affairs seems to be part of the reason as to why online platforms are full of "which language is better?" endless discussions. A syntactical criteria which acts as a response to these discusssions is \emph{consistency}. While there might be specific reasons as to why one would want to be writing code one way or another (e.g. calling functions on objects rather than calling functions from objects in order to prevent output arguments), this minor increase in aesthetic value—through display of skill and personal knowledge— doesn't compensate for the possible increase in cognitive noise if those different ways of writing are used alternatively in an arbitrary manner. In this context, consistency prevails over efficacy, and nonetheless hints at the fact that aesthetics in source code in this context is a game of tradeoffs.

Beyond the state of syntactic consistency, the question of \emph{linguistic reference}, bringing heuristics from one language to another is yet another aesthetic criteria. Being able to implicitly reference another language in a code-text (e.g. "this is how we do it now that we have C++, but the current code is written in C, so one can bring in ideas and syntax that are native to C++" or "since Ruby can qualify as a Lisp-like language, one can write lambda functions in an otherwise object-oriented language"), a code-switching of sorts, can both communicate a deep understanding of not just a language, but an ecosystem of languages while satisfying the purpose of maintaining clarity, assuming a certain skill level in the reader. This communicates a feeling of higher-understanding, akin to perceiving all programming languages as ultimately just "tools for the job" and whose purpose is always to get a concept across minds as fully and clearly as possible. However, a misguided intention of switching between two languages, or a mis-handled implementation can propel a code-text further down the gradient of ugliness. The concept communicated would in such a case be obscured by the conflicting idioms (e.g. writing p5.js—as a JavaScript implementation of a Java-based syntax—within an HTML document forces the co-existence of two distinct syntaxes which are made to cohabit more for purposes of platform-distribution rather than code clarity), reveal of lack of mastery of the unique aspects of the working language(s), and therefore fail to fulfill the aesthetic criterion of idiomaticity.

Finally, a syntax with high aesthetic value is a syntax which favors \emph{natural language reading flow}. For instance, of the two alternatives in Ruby: \lstinline{if people.include? person} vs. \lstinline{if person.in? people}, the second one is going to be considered more beautiful than the first one, since it adapts to the reader's habit of reading human languages. However, the essential succintness and clarity of source code is not to be sacrificed for the sake of human-like reading qualities, such as when writers tend to be overly explicit in their writing. Indeed, a criteria for ugliness in code-text is verbosity, or useless addition of statements without equivalent addition of functionality. This testifies to the need for a \emph{balance between machine idioms over human idioms}, with the sweet spot seemingly being the point at which machine idioms are presented as human-readable.

\subsubsection{Vocabulary}
\label{subsubsec:framework-vocabulary}

Vocabulary, as the only component in this framework which involves words that can be (almost) entirely invented by the writers, is often the most looked at in the literature regarding beautiful code among professional software developers. Aesthetics here deal mostly with beautiful names, and respect for conventional knowledge. It is the level of aesthetic standards which takes into account first and foremost the readership of a given code-text.

Of the two big problems of programming, the most frequent one is naming\footnote{(attributed to Phil Karlton), \url{http://karlton.hamilton.com/quotes/index.cgi}}. One reason as to why that is might be that naming (as language) is an inherently social activity \citep{voloshinov_marxism_1986} and therefore a name is an utterance which only makes sense when done in the expectation of someone else's comprehension of that name. This is supported by the fact that the process of creating a variable or function name on one's own is often more time-consuming when done alone, as reported by developers. Naming, furthermore, aims not just at describing, but at capturing the essence of an object, or of a concept. This is a process that is already familiar in literary studies, particularly in the role of poetry in naming the elusive. For instance, Vilém Flusser sees poetry as the briging-forth that which is conceivable but not yet speakable through its essence in order to make it speakable through prose, using the process of naming through poetry in order to allow for its use and function in prose. In this light, good, efficient and beautiful names in code are those who can communicate the essence of the concept that is being communicated (or parts thereof).

On a purely sensory level (visual and auditory), surface-level aesthetic criteria related to naming are that of \emph{character length} and \emph{pronounceability}. Visually, character length can indicate the relative importance of a named concept within the greater structure of the code-text. Variables with longer names are variables that are more important, demand more cognitive attention, offer greater intelligibility in comparison with shorter variable names, which only need to be "stored in memory" of the reader for a smaller amount of time. These visual cues, again, alleviate cognitive pain when trying to understand code, and therefore hold greater aesthetic value when respected. Pronounceability, meanwhile, take into account the basic human action of "speaking into one's head" and therefore participates in the requirement for communicability of source code amongst human readers. Similar to proper indentation and typographic consistency (see above), this particular criterion exists in the category of aesthetic criteria which are required, but not sufficient, for beautiful code.

Equally visual, but aesthetically pleasing for typographical reasons, is the \emph{casing} of names. Dealing with the constraint that variable names cannot have whitespace characters as part of them, casing has resulted into the establishment of conventions which pre-exists the precise understanding of what a word denotes, by first bringing that word into a category (all-caps denotes a constant, camelCasing denotes a multi-word variable and first-capitalized words indicate classes or interfaces). By using multiple cues (here, typographical, then semantical, as explicited below), casing again helps with understandability. Furthermore, casing, by its existence as a convention, implies that it exists within a social community of writers and readers, and acknowledges the mutual belonging of both writer and reader to such a community, and turns the code-text from a readerly text further into a writerly one \citep{barthes_bruissement_1984}.

Following these visual, auditory and typographical criteria, an aesthetically-pleasing vocabulary is a vocabulary which strictly names \emph{functions as verbs and variables as nouns}. In the vein of making a correspondance between machine language and human language, there is here a clear mapping between the two: functions do things and variables are things. If it's the other way around, while respecting the criteria for consistency, functions as nouns and variables as verbs hints at what it is not, are counter-intuitive and ultimately confusing—confusion which brings ugliness. The noun given to a variable should be a hint towards the concept addressed, and ideally address what it is, how it is used, and why it is present. Each of these three aims aren't necessarily easily achieved at the same time, but finding one word which, through multiple means, points to the same end, is an aesthetic goal of source code writers. Particularly, limiting the naming to be the answer to only one of those questions (only what, only what, or only why) could potentially confuse the reader more than it would enlighten her. A beautiful name is a name which differentiates between value (obvious, decontextualized, and therefore unhelpful, as seen by the general frowning-upon of using magic numbers) and intention, informing the reader not just about the current use, but also about future possible use, in code that is written or yet to be written. We see here a paradox between direct conceptual relationship between a name and what it denotes, and the multiple meanings that it embodies (its description, its desired immediate behaviour, and its purpose). Such a paradox is however overcome in the community of code poets.

While, in the community of software developers, variable names should then have a 1:1 mapping with the object or concept they denote, this isn't the case in other communities, whether those that rely on obfuscation, in which confusion becomes beautiful, or in poetic code, in which \emph{double-meaning} brings an additional, different understanding which ultimately enriches the complexity of the reading \citep{knuth_literate_1984}. Indeed, it is the easiest way for writers to offer metaphors, and provides an entry point in to the possibility that all source code is itself a practical metaphor for the task—and therefore the problem—at hand. This aesthetic criteria of double-meaning comes from poetry in human languages, in which layered meanings are aesthetically pleasing, because they point to the un-utterable, and as such, perhaps, the sublime. The way that this community (code poets and artists) address the aesthetic problem of naming and, more generally, how source code and literature make use of metaphors, is part of the next steps that will be taken in this research.

A final aesthetic criterion for vocabularies is the \emph{limitation of function arguments}, according to which arguments given to a function should be either few, or grouped under another name. Going back to the structural criterion above of limiting input/output and keeping groups of statements conceputally independent, function arguments solves this requirement at the level of vocabulary, demonstrating in passing the relative porosity of those categories. Indeed, the naming of variables also reveals the pick of \emph{adequate data-structures}, echoing those who claim that the data on which the code operates can never be ignored, and that beautiful code is code which takes into account that data and communicates it, and its mutations, in the clearest, most intelligible, possible way.

\subsubsection{Comments}
\label{subsubsec:framework-comments}

Comments in code do not seem to fall clearly in any of the three categories above. By definition ignored by the compiler/interpreter, comments can be erroneous statements which will persist in an otherwise functional codebase, and are therefore not trusted by experienced, professional software practicioners. In this configuration, comments seem to exist as a compensation for a lack of functional aesthetic exchange. By functional aesthetic exchange I mean an exchange in which a skilled writer is able to be understood by a skilled reader with regards to what is being done and how. If any of these conditions fail (the writer isn't skilled enough and relies on comments to explain what is going on and how it is happening, or the reader isn't skilled enough to understand it without comments), then comments are here to remedy to that failure, and therefore are an indicator (but, again, not a proof) of non-beautiful code. For instance, referencing a variable name in a comment is a sure indicator of a message which refers to the what/how of a group of statements and is on the verge of stating the obvious (if not already stating the obivous).

The situation in which comments seem to be tolerated is when they provide contextual information, therefore (re-)anchoring the code in a broader world. For instance, this is achieved by offering an indication as to why such an action is being taken at a particular moment of the code, called contractual comments, pointing at the social existence of source code. This particular use of comments seems to bypass the aesthetic criteria of code being self-explanatory, but nonetheless integrates the criteria of code being writable, a piece of code which, by its appearance, invites the reader to contribute to it, to modify it. As such, in an educational setting (from a classroom to an open-source project), comments are welcome, but rarely quoted as criteria for beautiful code, which seems to indicate that the appreciation of beautiful code does require a certain level of skill.

\subsection{Conceptual specificities and aesthetic manifestations} %20k
\label{subsec:conceptual-specificities}

Starting from the concept of chronotopes, of markers, we show where these need to be aesthetically manifested in order to facilitate understanding.

What are the chronotopes of source code?
- local/global
- encapsulation/abstraction
- sync/async and blocking/non-blocking
- stack/heap
- memory/disk/network

more prosaic: \url{https://cacm.acm.org/magazines/2011/12/142527-coding-guidelines-finding-the-art-in-the-science/fulltext}

And also the section about the double-meaning/double-coding/procedural-rhetoric

\section{Contextual aesthetics} %40k
\label{sec:contextual-aesthetics}

This concludes on the contextual component. The point is that aesthetics is conditional to function. And, while the machine might be one in order to execute that function, it exists in a layer of endless hardwares, softwares, concerns and people which then \emph{modulates} the way this function is described and, ultimately represented.

\subsection{Technical environments}
\label{subsec:technical-environments}

The technical environments impact what is considered nice or beautiful: hardware to software, concrete to abstract.

\subsection{Social environments}
\label{subsec:social-environments}

\subsection{Functional environments}
\label{subsec:functional-environments}

Ultimately, this does relate to function: the requirement for things to function differently will impact the way they look

\spacer


In conclusion, we have shown that programming languages embody multiple dynamics of aesthetics in source code: technical environment, social environment and problem domain. From there on, we have proposed our theory about levels and distances, confirming this idea of mental structures clearly manifested in source code, and of metaphors. This has been followed by case studies which have shown, in a variety of corpuses, how these are present in specific program-texts. And, again, that aesthetics is inseparable from functionality.