\chapter{Understanding source code}

In the previous chapter, we've seen that there is a focus on understanding when it comes to aesthetic standards: whether obfuscating or illuminating, understanding is a key determinant in the value judgment.

The point of this chapter is then to investigate what is it that makes code complicated to understand. This will have us deal with the nature of computation (what do computers and source code do?), with the nature of the world (how does one translate a problem domain into a software domain, through process of modelling and abstraction) and with the nature of the human mind (through the requirement to communicate with other humans.)

We will see what roles metaphors play; and, if linguistics is a key component in the writing of clear source code, we should also look at programming languages.

The question is now: how does one organise all this?

% first the computer
\section{Computation as a technical phenomenon}

% then the human mind
\section{Computation as a mental phenomenon}

% then the communcation between both
\section{Metaphors of computation}